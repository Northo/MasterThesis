\documentclass{article}

\usepackage{braket}
\usepackage{bm}
\usepackage{amsmath}
\renewcommand\vec\bm

\title{Problem with the energy-momentum tensor}
\date\empty
\begin{document}
\maketitle
\vspace{-1cm}
We have a response function
\begin{equation}
  \label{eq:1}
  \chi(t, \vec{r}) = \int \mathrm{d}t' \mathrm{d} \vec{r}'
  \left\{
  \Braket{\left[\vec{J}(t, \vec{r}, T^{00}(t', \vec{r}')\right] } \psi(t', \vec{r}')
\right\},
\end{equation}
where \( \vec{J} \) is a charge current, \( T^{\mu\nu} \) is the energy-momentum tensor, and \( \psi \) is a gravitational potential \footnote{Yes, condensed matter physisists have a hack where they use gravitational potential to ``simulate'' temperature.}.
Using the continuity relation of the energy-momentum tensor
\[
\partial_0 T^{00} + \partial_i T^{0i} = 0 \implies T^{00}(t, \vec{r}) = -\int_{-\infty}^t \mathrm{d}t' \partial_iT^{0i}(t', \vec{r}),
\]
and integration by parts
\[
\int u v' = uv - \int u' v
\]
we get
\begin{equation}
  \chi = \int \mathrm{d}t' \mathrm{d} \vec{r}'
  \int \mathrm{d}t''
  \left\{
  \Braket{\left[\vec{J}(t, \vec{r}, T^{0j}(t'', \vec{r}')\right] } \partial_j'\psi(t', \vec{r}')
\right\}.
\end{equation}

Now, our response function, an observable, depends on the value of \( T^{0j} \) itself, not the divergence.

\end{document}
