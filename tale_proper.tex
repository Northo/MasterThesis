% Created 2022-06-02 Thu 12:46
% Intended LaTeX compiler: pdflatex

\documentclass[12pt, a4paper]{scrarticle}
\KOMAoptions{DIV=3}
% \usepackage[top=1cm, left=1.3cm, right=1.3cm]{geometry}
\usepackage{parskip}
\usepackage{microtype}
% \setlength{\parskip}{2em}
\begin{document}
``I denne forsamlingen er det en kolossal spennvidde, ikke minst i faglig dyktighet'',
som vår kjære professor Iver Brevik sa i sin tale under Nablas 75-årsjubileum da vi gikk i førsteklasse.
(vi lot oss ikke skremme vekk så lett, så her står vi)
Når man ber folk om tips til hvordan man skal skrive en god tale, så går det gjerne igjen at man burde spille på nostalgi.
Trekk frem noe som man sammen har lidd seg gjennom sammen, som man kan samles om.
Da vi pratet sammen om denne talen innså vi at det ikke nødvendigvis var lett, for vi har for det meste hatt det ganske fint sammen.
Både på og utenfor universitetets vegger har vi hatt et fint samhold og gode opplevelser, og det tenker vi å mimre om litt idag.
(pause)
\ldots{}

(Elsie)
Men så kom vi på MekFys-, Elmag-, og Matte 4-eksamen, og innså at vi har hatt det litt kjipt sammen også.
(korona? Korona har sørget for at vi har hatt det litt kjipt uten å være sammen også)

Vårt første møte med Trondheim og hverandre var fadderperioden.
Noen av dere som faller under familie, venner og foresatte her i salen er vel kjent med hva fadderperioden er, men det er nok også mange som ikke kjenner så godt til dette fenomenet.
Når man kommer som ny student til ny by, med mange nye folk, er det kjekt at det er tilrettelagt med opplegg for at man raskt kan bli kjent med sin nye tilværelse og nye studievenner.
Linjeforeningene og NTNU har hver sin strategi for dette:
Linjeforeningene løser dette med et to-ukers program med fester, aktiviteter, og turer.
NTNU på sin side, har fulle arbeidsdager med ``teknostart'', der man blant ruller klinkekuler ned renner, skriver labjournal, og hører på forelesninger, på svensk, der stoffet er bevis av hawking radiation ved dimensjonsanalyse\ldots{}
Kombinert utgjør dette to svært hektiske, men også minneverdige, uker, der man løper fra det ene til det andre, med knapt noe tid til å hive i seg en matbit mellom slagene.

Det er i fadderperioden man først opplever hvor fantastisk og anderledes fysmat er:
det er selvsagt fult med fester og sprell, men samtidig er det også memorisering av siffer i Pi og e, greske alfabet og studentviser.
Alle nerdene fra videregående samles innenfor en kritisk tetthet.
(Måtte jo få inn én dårlig fysikkvits her)

Vi kan ikke prate i all evighet om fadderperioden, men må jo raskt nevne:
BTB, L2L, Togafest, NWA
% \begin{itemize}
% \item btb
% \item l2l
% \item Togafest
% \item Nablas egen rappegruppe NWA som opptredde på max gangsta (høyrehåndsregel).
% \begin{itemize}
% \item Jeg husker særlig da vi stod på samfundet, med NWA som sang om fysmat, og vi alle holdt oppe høyrehåndsregelen.
% \end{itemize}
% \end{itemize}

(Elsie sier noe om opptak)

\newpage

Slik vi prater nå, teoriser kanskje noen av dere foresatte i salen at vi bare fester og morer oss, og ikke gjør noe arbeid.
Nok en gang er et sitat fra Iver Brevik passende:
Det er en svært elegant teori, det er bare det at den er feil.
Som Fysmattere jobber vi jo også hardt med studiene.
Fysmat er kjent som et av de mest krevende studiene på NTNU.

Ikke bare gjør Fysmattere det bra i studiene; både akademisk og ikke-akademisk overgås det som er å forvente.
Vi har hatt folk som har jobbet ved CERN, andre har forsket på fusjonsreaktorer, for ikke å nevne alle de talløse vinnere og deltakere av diverse olympiader man finner her.
Selv kombinert med en hektisk studiehverdag, har de fleste her i salen hatt verv enten i Nabla, UKA, Samfundet, eller andre grupper.
Noen har også tatt svært store verv.
Her finner vi leder for Samfundet her, leder for NTNUI og andre medlemmer i NTNUI sitt hovdestyre. sjef for velferdstinget, og personer med verv med store ansvarsområder innen både UKA og Isfit.

(UKA)

Vi husker jo alle 12. mars 2020.
Vi følte oss som kriminelle da vi gikk inn på skolen og hentet bøker fra pulten, og dro hjem.
Der ble vi sittende i de neste to år.
Det var jo kjipt at utvekslingen ble avlyst av korona, noe vi vet mange hadde gledet seg mye til.
Det var heldig at vi hvertfall fikk ekskursjonen som et lite plaster på såret\ldots{} (kanskje elsie sier dette)
Og så trenger vi ikke si mer om det

(avsluting)
\end{document}
