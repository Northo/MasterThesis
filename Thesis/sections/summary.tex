\addchap{Preface}
This thesis constitutes the final part of a ``Sivilingeniør'' (M.Sc.) degree in Physics and Mathematics at the Norwegian University of Science and Technology.
It was written during the spring of 2022, after a five-year study program.

I would like to thank my supervisor, Alireza Qaiumzadeh, for excellent guidance throughout this last year.
This has been a novel topic for both of us, and our weekly meetings have been filled with interesting discussions and conversations.
% Thank you for guiding me through enormous amounts of information, and helping me see what was relevant.
Thank you for being a guiding lantern through an overwhelming maze of information, letting me see what is relevant and what is not.

Furthermore, I want to thank
\href{https://wp.icmm.csic.es/field-theories-in-condensed-matter-physics/vozmediano/}{María A. H. Vozmediano}\footnote{Materials Science Factory, Instituto de Ciencia de Materiales de Madrid, CSIC, Cantoblanco, 28049 Madrid, Spain\label{address-of-spain}.}
and
\href{https://wp.icmm.csic.es/field-theories-in-condensed-matter-physics/alberto-cortijo/}{Alberto Cortijo}\footnotemark[1]
for excellent discussions during the finalization of the thesis.
Through our meetings, the significance and interpretation of the results were made more clear, and interesting questions and continuations discussed.
It was also reassuring to hear that you had struggled with some of the same issues that we have faced.

The work presented in this thesis is currently being worked into a manuscript written primarily by me and Alireza, in collaboration with Maria and Alberto.


\vspace{3cm}
{
  \centering
  \hrulefill\bigskip\\
  \parbox{0.84\textwidth}{
    \centering
  \texttt{This document has been typeset with the intention of printing on B5 paper. Consequently, the text and figures will look too large when printed on A4 paper.}
  }\\
  \bigskip\smallskip\hrulefill
}

\addchap[Summary | Oppsumering]{Summary}
Emergent Dirac equations in topological condensed matter physics may, as opposed to their high energy physics equivalents, have Lorentz-breaking terms.
Several such systems have been discovered both theoretically and experimentally, among them the tilted Dirac and Weyl semimetals.
Non-tilted Dirac and Weyl semimetals have previously been shown to house transverse charge current response to thermal gradients in a magnetic field, a Nernst effect.
The origin of the effect is the conformal anomaly, a quantum anomaly related to non-flat spacetime.
The effect, importantly, is finite even for zero chemical potential and temperature.
Using the Kubo formalism, we have extended the calculation to find the response function for a system with tilt.

Using Luttinger's relation, we introduce an effective gravitational field from the thermal gradient, which couples to the energy density.
By employing the conservation of energy, we can reformulate the response as a response to the derivatives of elements of the energy-momentum tensor.

We find the effect to be tunable by the direction and magnitude of the tilt with respect to the magnetic field.
Several possible candidates for experimental signatures are presented, and thus this may give a possible venue for experimental investigation of tilted Dirac and Weyl semimetals.
Furthermore, we show the importance of the specific choice of the energy-momentum tensor, which for non-zero tilt directly affects the computed response.
The ambiguity of the energy-momentum tensor is well known, however, our results show explicitly that in these types of systems, the choice is not only a conceptual formality but has qualitative consequences.

% LTeX: language=no-NO
\addchap*{Oppsummering}
Effektive Dirac likninger i topologisk faststoff-fysikk kan, i motsetning til høy-energi ekvivalentene, ha ledd som bryter Lorentz invarianse.
Flere slike systemer har blitt oppdaget, både teoretisk og eksperimentelt, deriblant skråstilte Dirac og Weyl semimetaller.
Det har tidligere blitt vist at ikke-skråstilte Dirac og Weyl semimetaller gir opphave til transversale ladningsstrømmer som respons på temperaturgradienter i magnetiske felt, en Nernst effekt.
Effektens opphav er den konforme anomaliteten, en kvante-anomalitet relatert kurvet romtid.
Effekten består, viktig nok, også ved null kjemisk potensiale og temperatur.
Vi har generalisert utregningen til skråstilte systemer, ved hjelp av Kubo-formalismen.

Gjennom Luttingers relasjon, introduserer vi et effektivt gravitasjonsfelt fra den termiske gradienten, som vekselvirker med energitettheten.
Ved bevaringsloven for energi, kan dette omformuleres som en respons på den deriverte til elementer av energi-impuls-tensoren.

Vi ser at effekten er justerbar ved retning og størrelse på skråstillingen i forhold til magnetfeltet.
Flere kandidater til eksperimentell signatur presenteres, og dette kan dermed være en mulighet for eksperimentell utforsking av skråstilte Dirac og Weyl semimetaller.
Videre viser vi viktigheten av valg av energi-impuls-tensoren, som for skråstilte systemer påvirker resultatet av beregningen.
Tvetydigheten rundt definisjonen av tensoren er vellkjent, men vårt resultat viser eksplisit at i denne typen systemer, så har valget ikke bare konseptuell betydning, men direkte kvalitative effekt.
% LTeX: language=en-US
