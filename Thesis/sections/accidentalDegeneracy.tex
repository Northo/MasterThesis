\section{Accidental degeneracy}
\label{sec:accidental-degeneracy}
In general, for a two level system depending on some parameter the energy levels of the two levels will not cross, i.e. be degenerate, unless there are symmetries in the system forcing them to be degenerate, as is the case in for example Kramer's degeneracy.
However, even without any symmetries
\footnote{There will always, for a degenerate system, be some symmetry, although it might be a \emph{hidden} symmetry. We here mean no a priori apparent symmetry.}
there will be so-called \emph{accidental degeneracies} if the parameter space is sufficiently large.
Consider a general two-level Hamiltonian
\begin{equation}
  H = f_1 \sigma_x + f_2 \sigma_y + f_3 \sigma_z,
\end{equation}
which will have an energy splitting between the two levels
\begin{equation}
  \Delta E = 2 \sqrt{
    f_1^2 + f_2^2 + f_3^2
  }.
\end{equation}
In general, we may solve $\Delta E = 0$ by tuning the three parameters simultaneously, and thus there must be degenerate points -- accidental degeneracies.
Supposing that the parameters $f_i$ can be expressed as functions of the momentum components, $f_i =f_i(p_i)$, this will correspond to degenerate points in momentum space.

If there are in addition some symmetry constraints on the system, the space of degenerate points may increase.
Suppose, for example, the system is time-reversal symmetric.
Recalling the time-reversal operator defined in \cref{eq:time-rev-def}
$$
\Theta = UK,
$$
with $U$ being a unitary operator and $K$ the complex conjugate, the imaginary Pauli matrix $\sigma_y$ must be excluded.
Thus, the solution to the closing of the band gap has a free parameter, and the degenerate space has dimension one.
