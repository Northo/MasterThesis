\section{Removing the explicit tilt from the Lagrangian by a non-flat metric}\label{sec:tilt-metric}
In the main text, we have used the Lagrangian
\begin{equation}
  L_s = i \phi^{\dagger} \tilde{\sigma} ^{\mu} \partial_{\mu} \phi,
\end{equation}
where we defined the \emph{modified} Pauli matrices \( \tilde{\sigma} ^{\mu}= \sigma^{\mu} + t^{\mu}, \; t^{\mu} = (0, \vec{t}) \).
We here present an alternative, where we instead consider moving the tilting into the metric, i.e. considering a non-tilted cone in curved spacetime.
In essence, we want
\[
g^{\mu \nu} \sigma_{\nu} p_{\mu}
\]
to give \( (\sigma^{\mu} + t^{\mu}) p_{\mu} \).
We see that this involves putting \( t^{\mu} \) on the 0-components of the metric.

Consider the metric
\begin{equation}
  g^{\mu \nu} = \eta^{\mu \nu } + t^{\mu} (\delta^{\mu}_0 + \delta^{\nu}_0) =
  \begin{pmatrix}
    1 & t^x & t^y & t^z\\
    t^x & -1 & 0 &0\\
    t^y & 0 & -1 &0\\
    t^z & 0 & 0 &-1
  \end{pmatrix}.
\end{equation}
The top row is however problematic, as it gives an unwanted
\[
g^{0 \nu} \sigma_{\nu} = \sigma^0 - \vec{t} \vec{\sigma}.
\]
Interestingly, the metric thus \emph{cannot} be symmetric!
We conclude that the appropriate choice is
\begin{equation}
  g^{\mu \nu} = \eta^{\mu \nu } + t^{\mu} \delta^{\nu}_0 =
  \begin{pmatrix}
    1 & 0 & 0 & 0\\
    t^x & -1 & 0 &0\\
    t^y & 0 & -1 &0\\
    t^z & 0 & 0 &-1
  \end{pmatrix}.
\end{equation}

Consider therefore the Lagrangian
\begin{equation}
  \mathcal{L} = \phi^{\dagger} g^{\mu \sigma} \sigma_{\sigma} \partial_{\mu} \phi.
\end{equation}
Using once again the canoncial defintion
\begin{align}
  T^{\mu \nu} &= \frac{\partial \mathcal{L}}{\partial(\partial_{\mu}\phi)} \partial^{\nu} \phi\\
  &= \phi^{\dagger} g^{\mu \sigma} \sigma_{\sigma} \partial^{\nu} \phi,
\end{align}
we find the component of the energy-momentum tensor
\begin{equation}
  T^{y 0} = \phi^{\dagger} g^{y \sigma} \sigma_{\sigma} \partial^0 \phi =
  \phi^{\dagger} (t^y \sigma_0 + \sigma_y) \partial^0 \phi.
\end{equation}

\begin{remark}
  The four-Pauli matrices are defined as
  \begin{equation}
    \sigma^{\mu} = (\sigma^0, \sigma^1, \sigma^2, \sigma^{3}) = (\sigma^0, \sigma_x, \sigma_y, \sigma_z),
  \end{equation}
  where the matrices with lower roman indices are the well-known Pauli matrices.
  Thus, the four-matrices with lowered inddex are
  \begin{equation}
    \sigma_{\mu} = g_{\mu \nu} \sigma^{\mu} = (\sigma^0, -\sigma^1, -\sigma^2, -\sigma^{3}).
  \end{equation}
\end{remark}
