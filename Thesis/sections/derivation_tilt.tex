\section{The response of a tilted cone}

\todo{Write this better

  Repeating the calculation of the response function is now straightforward, but rather tedious.
Due to the boost transformation, the elements of the spinor in the untilted system, Eq. \eqref{eq:47}, mix.
We thus have twice as many terms to keep track of.
}

\subsection{Explicit form of the matrix elements}
We will here find an explicit form of the matrix elements, starting with the charge current
\begin{equation}
  \label{eq:tilt:currentExpr}
  J_{\vec{k} m s; \vec{k}+\qvec{q} n s } (\vec{q}) = \int \mathrm{d}y e^{-i q_{y} y}
  s v_{F} e \phi ^{*}_{\vec{k} m s} (y) (\sigma^{x} + s t^s_x) \phi _{\vec{k} + \qvec{q} n s}(y),
\end{equation}
which we will split into two parts, \( J^{(1)}, J^{(2)} \), corresponding to the terms \( \sigma_x \) and \( s t^s_x \).
For the first part, we must find the matrix product \(\phi \sigma_{x} \phi \).
Recall from summary \ref{summary:llevels} that \(\phi = \sqrt{\alpha}  e^{\theta /2 \sigma _{x}} \tilde{\phi} \), and thus we must find
\[
  \phi ^{*} \sigma _{x} \phi
  = \alpha \tilde{\phi}^{*} e^{\theta /2 \sigma _{x}} \sigma _{x} e^{\theta /2 \sigma _{x}} \tilde{\phi}
  =  \alpha \tilde{\phi}^{*} \sigma _{x} e^{\theta \sigma _{x}} \tilde{\phi}.
\]
As defined in summary \ref{summary:llevels}
\[
  \tilde{\phi} = e^{-\frac{1}{2} \chi ^2}
  \begin{pmatrix}
    a_{\vec{k} m s} H_{M-1} (\chi)\\
    b_{\vec{k} m s} H_{M} (\chi)
  \end{pmatrix},\quad
  \chi = \sqrt{\alpha} \frac{ y - k_x l_B^2 }{l_{B}} + \frac{t_{\perp}^s l_B}{\sqrt{\alpha} v_{F}} E^0_{m, \alpha B}.
\]
With the previously found solution \(\theta = - \tanh ^{-1} t^s_{x}\), we get the rather simple form
\[
  e^{\theta \sigma _{x}} =
  \begin{pmatrix}
    1 & - s t^s_{x}\\
    -s t^s_{x} & 1
  \end{pmatrix}
  \frac{1}{\sqrt{1-t_{x}}}.
\]
Where we in the untilted case only have off-diagonal contributions from \( \sigma_x \), the hyperbolic rotation gives contributions on the diagonal as well.

First of all, let us consider the exponent of the product.
We want to complete the square similarly to what was done for the untilted cone in section \ref{sec:notilt:explicit}.
Due to the extra term in \(\chi\), this becomes more elaborate.
The exponent in the current matrix element Eq.~\eqref{eq:tilt:currentExpr} is of course
\begin{equation}
  \label{eq:99}
  \exp\{-i q_{y} y - \frac{1}{2} \chi_{\vec{k}} ^2 - \frac{1}{2} \chi _{\vec{k} + \qvec{q}}^2\}.
\end{equation}
A straightforward but tedious calculation shows that the argument of the exponent can be written as
\begin{equation}
  \label{eq:100}
  -\frac{\alpha}{l_{B}^2} \left(y + \frac{l_{B}^2}{2 \alpha } (i q_{y} - (q'_x + 2 k'_x))\right)^2
  -\frac{l_{B}^2}{4 \alpha } (q_{y}^2 + 2i (q'_x + 2 k'_x) q_{y} + ( q' _{x} )^2 ),
\end{equation}
where we have defined
\begin{align}
  \label{eq:qkprime}
  q' _x &= q_x \alpha  - \frac{\beta}{v_{F} }( E^0_{n,\alpha B} - E^0_{m, \alpha B} ),\\
  k' _x &= k_x\alpha - \frac{\beta}{v_F } E^0_{m, \alpha B}.
\end{align}
\todo{check sign of E above}
These must not be confused with the transformed momenta \( \tilde{k} \), which are similar in form.
Eq. \eqref{eq:100} is on the same for as in the untilted cone case, and we may thus proceed with the same method.
Make a change of variable
\[
\tilde{y} = \frac{\sqrt{\alpha }}{l_{B}} \left(y + \frac{l_{B}^2}{2\alpha } (iq_{y} - 2 k' _x - q' _x )\right),
\]
\todo{Follow up the substitution of the root in the integral. Consider moving the root into \(\Xi \)}
to get the exponent on the form \(e^{-\tilde{y}^2}\).
With this substitution,
\begin{align}
  \chi _{\vec{k}} &= \tilde{y} + \frac{l_{B}}{2 \sqrt{\alpha }} \left(q' _x - i q_{y}\right),\\
  \chi _{\vec{k} + \qvec{q}} &= \tilde{y} + \frac{l_{B}}{2 \sqrt{\alpha }} \left(- q' _x - i q_{y}\right).
\end{align}
The first part of the current matrix element, Eq.~\eqref{eq:tilt:currentExpr}, is thus
\begin{equation}
  \begin{split}
    J^{(1)}_{\vec{k}ms; \vec{k}+\qvec{q} ns}(\vec{q}) &=
    \frac{s v_F e}{\sqrt{\alpha }} \int \mathrm{d}\tilde{y} \: l_B
    \exp \left[
      -\frac{l_{B}^2}{4 \alpha } \left(q_{y}^2 + 2i (2 k' _x + q' _x) q_{y} + (q'_x)^2 \right)
    \right]
   \\
    e^{-\tilde{y}^2}
   &\left[
    a_{\vec{k}ms}b_{\vec{k} + \qvec{q} ns}
    H_{M-1} \left(  \chi_{\vec{k}} \right)
    H_N\left( \chi_{\vec{k} + \qvec{q}} \right) \right.\\
    &- s t_{x} a_{\vec{k}ms}a_{\vec{k} + \qvec{q} ns}
    H_{M-1} \left( \chi_{\vec{k}} \right)
    H_{N-1}\left( \chi_{\vec{k} + \qvec{q}} \right)\\
   & +
    b_{\vec{k}ms} a_{\vec{k} + \qvec{q} ns}
    H_M \left( \chi_{\vec{k}} \right)
    H_{N-1} \left( \chi_{\vec{k} + \qvec{q}} \right)\\
    &\left. - s t_{x}
    b_{\vec{k}ms} b_{\vec{k} + \qvec{q} ns}
    H_M \left( \chi_{\vec{k}} \right)
    H_{N} \left(  \chi_{\vec{k} + \qvec{q}}\right)
    \right].
  \end{split}
\end{equation}

Next consider the second term of the current operator,
\begin{equation}
  \label{eq:103}
  J_{\vec{k} m s; \vec{k} + \qvec{q} n s}^{(2)} (\vec{q}) =
  e v_{F} t^s_{x}
  \int \mathrm{d} y
  e^{-iq_{y} y}
  \phi ^{*}_{\vec{k} m s}(y)  \phi _{\vec{k} + \qvec{q} n s}(y).
\end{equation}
With a procedure similar to above, with the same substituion and completion of the square
\begin{multline}
  \label{eq:105}
  J_{\vec{k} m s; \vec{k} + \qvec{q} n s}^{(2)} (\vec{q}) =
  \frac{s v_F e t^s_x}{\sqrt{\alpha}}
  \int \mathrm{d} \tilde{y} l_B
    \exp \left[
      -\frac{l_{B}^2}{4 \alpha } (q_{y}^2 + 2i (2 k'_x + q' _x) q_{y}  + (q'_x)^2 )
    \right]\\
  e^{-\tilde{y}^2} \big[
    a_{\vec{k} m s} H_{M-1}( \chi _{\vec{k}} ) \left(s a_{\vec{k} + \qvec{q} n s} H_{N-1}( \chi _{\vec{k} + \qvec{q}} ) - t^s_{x} b_{\vec{k} + \qvec{q} n s} H_{N}( \chi _{\vec{k} + \qvec{q}} ) \right)\\
   +
    b_{\vec{k} m s} H_{M}( \chi _{\vec{k}} ) \left(- t^s_{x} a_{\vec{k} + \qvec{q} n s} H_{N-1}( \chi _{\vec{k} + \qvec{q}} ) + s b_{\vec{k} + \qvec{q} n s} H_{N}( \chi _{\vec{k} + \qvec{q}} ) \right)
    \big].
\end{multline}
By inspection,  recalling \( \sqrt{1 - t_x^2} = \alpha \), we see
\begin{multline}
J_{\vec{k}ms; \vec{k}+\qvec{q} ns}(\vec{q}) =
  s v_F e \sqrt{\alpha} \int \mathrm{d}\tilde{y} \: l_B
  \exp \left[
    -\frac{l_{B}^2}{4 \alpha } \left(q_{y}^2 + 2i (2 k' _x + q' _x) q_{y} + (q'_x)^2 \right)
  \right]
  e^{-\tilde{y}^2}
  \\
  \times
  \Big[
  a_{\vec{k}ms}b_{\vec{k} + \qvec{q} ns}
  H_{M-1} \left(  \chi_{\vec{k}} \right)
  H_N\left( \chi_{\vec{k} + \qvec{q}} \right)
  +
  b_{\vec{k}ms} a_{\vec{k} + \qvec{q} ns}
  H_M \left( \chi_{\vec{k}} \right)
  H_{N-1} \left( \chi_{\vec{k} + \qvec{q}} \right)
  \Big].
\end{multline}

To perform the integration, we use the \emph{shifted orthogonality} relation for Hermite polynomials~\cite[Eq. (7.377)]{gradshteinTableIntegralsSeries2015}
\begin{equation}
  \label{eq:hermite-shift-ortho}
  \int\limits_{-\infty }^{\infty } \mathrm{d}x
  e^{-x^2} H_m(x+y) H_n(x+z)
  = 2^n \pi^{\frac{1}{2}} m! y^{n-m} L^{n-m}_m(-2yz), \quad m\leq n,
\end{equation}
where \(L^{a}_{b}\) are the \emph{generalized Laguerre polynomial} of order \(b\) and type \(a\).
Define the functions \( \Xi_1, \Xi_2 \) by
\begin{align}
  \label{eq:101}
  \frac{ \sqrt{\alpha } \alpha_{k_z m s} \Xi_1 ( \vec{q}, m, n, s )}{
    \sqrt{\alpha_{k_z m s}^2 + 1}
    \sqrt{\alpha_{k_z + \qvec{q}_z n s} ^2 + 1}
  }
  &=
  \int \mathrm{d} \tilde{y}
  ~e^{-\tilde{y}^2}
  l_{B}
  a_{\vec{k} m s} b_{\vec{k} + \qvec{q} ns}
  H_{M-1} (\chi_{\vec{k}})
  H_N ( \chi _{\vec{k} + \qvec{q}} ),\\
  \label{eq:102}
  \frac{ \sqrt{\alpha } \alpha_{k_z + \vec{q} n s} \Xi_2 ( \vec{q}, m, n, s )}{
    \sqrt{\alpha_{k_z m s}^2 + 1}
    \sqrt{\alpha_{k_z + \qvec{q}_z n s} ^2 + 1}
  }
  &=
  \int \mathrm{d} \tilde{y}
  ~e^{-\tilde{y}^2}
  l_{B}
  b_{\vec{k} m s} a_{\vec{k} + \qvec{q} ns}
  H_{M} (\chi_{\vec{k}})
  H_{N-1} ( \chi _{\vec{k} + \qvec{q}} ).
\end{align}
Using that
\begin{align}
  a_{\vec{k} m s} b_{\vec{k} +\qvec{q} ns}
  &=
    \frac{ \sqrt{\alpha} \alpha_{k_z m s}}{\sqrt{\alpha_{k_z m s} ^2 + 1} \sqrt{\alpha_{k_z + \qvec{q}_z ns} ^2  + 1}  }
    \left[
    2^{N+M-1} (M-1)! N! \pi l_{B}^2
    \right  ]^{-\frac{1}{2}},\\
  b_{\vec{k} m s} a_{\vec{k} +\qvec{q} ns}
  &=
    \frac{\sqrt{\alpha} \alpha_{k_z + \qvec{q}_z n s}}{\sqrt{\alpha_{k_z m s} ^2 + 1} \sqrt{\alpha_{k_z + \qvec{q}_z ns} ^2  + 1}  }
    \left[
    2^{N+M-1} (N-1)! M! \pi l_{B}^2
    \right  ]^{-\frac{1}{2}},
\end{align}
we use Eq.~\eqref{eq:hermite-shift-ortho} to find explicit expressions
\begin{subequations}
  \label{eq:xi1def}
  \begin{align}
    \Xi_1 ^{(1)}(\vec{q}, m, n, s) &= \sqrt{\frac{2^N (M-1)!}{2^{M-1} N!}}
                                     \left( \frac{q'_x - iq_y}{2 \sqrt{\alpha } } l_B \right)^{N-M + 1}
                                     L^{N-M+1}_{M-1} \left( \frac{\qvec{q}_y^2 l_B^2}{2 \alpha } \right),\\
                                     %%%
    \Xi_1 ^{(2)}(\vec{q}, m, n, s) &= \sqrt{\frac{2^{M-1} N!}{2^N (M-1)!}}
                                     \left( \frac{-q'_x - iq_y}{2 \sqrt{\alpha } } l_B \right)^{M-N - 1}
                                     L^{M - N - 1}_N \left( \frac{\qvec{q}_y^2 l_B^2}{2 \alpha } \right),\\
    \Xi_1(\vec{q}, m, n, s) &=
                              \begin{cases}
                                \Xi _1 ^{(1)} & \text{if } N \geq M-1\\
                                \Xi _1 ^{(2)} & \text{if } N \leq M-1
                              \end{cases} \text{ for } M>0, N \geq 0,
  \end{align}
\end{subequations}
\begin{subequations}
  \label{eq:xi2def}
  \begin{align}
    \Xi_2 ^{(1)}(\vec{q}, m, n, s) &= \sqrt{\frac{2^{N-1} M!}{2^M (N-1)!}}
                                     \left( \frac{q'_x - iq_y}{2 \sqrt{\alpha }} l_B \right)^{N-1 - M}
                                     L^{N-1 -M}_{M} \left( \frac{\qvec{q}_y^2 l_B^2}{2 \alpha } \right),\\
                                     %%%
    \Xi_2 ^{(2)}(\vec{q}, m, n, s) &= \sqrt{\frac{2^M (N-1)!}{2^{N-1} M!}}
                                     \left( \frac{-q'_x - iq_y}{2 \sqrt{\alpha }} l_B \right)^{M-N + 1}
                                     L^{M - N + 1}_{N-1} \left( \frac{\qvec{q}_y^2 l_B^2}{2 \alpha } \right),\\
    \Xi_2(\vec{q}, m, n, s) &=
                              \begin{cases}
                                \Xi _2 ^{(1)} & \text{if } N-1 \geq M\\
                                \Xi _2 ^{(2)} & \text{if } N-1 \leq M
                              \end{cases} \text{ for } M\geq 0, N > 0,
  \end{align}
\end{subequations}
Here, \( \qvec{q}_y = (q'_x, q_y) \).

Thus, the current matrix element in terms of  the functions \( \Xi_i \) is
\begin{multline}
  J_{\vec{k} m s; \vec{k} + \qvec{q} n s} (\vec{q}) =
  e v_F s \alpha^2
  \frac{
    \exp \left[
      -\frac{l_{B}^2}{4 \alpha } (q_{y}^2 + 2i (2 k' _x + q' _x) q_{y} + (q'_x)^2 )
    \right]
  }{
    \sqrt{\alpha_{k_z m s}^2 + 1} \sqrt{\alpha_{k_z + \qvec{q}_z n s}^2 + 1}
  }\\
 \times \left[
     \alpha_{k_z m s} \Xi_1(\vec{q}, m, n, s) + \alpha_{k_z + \qvec{q}_z n s} \Xi_2(\vec{q}, m, n, s)
  \right].
\end{multline}

\subsubsection{Energy-momentum tensor}
Consider now the energy-momentum tensor matrix element
\begin{equation}
  \label{eq:106}
  T_{\vec{k} + \qvec{q} n s, \vec{k} m s}^{0y} (\vec{q}) =
  \frac{1}{2}
  \int \mathrm{d} y
  e^{i q_{y} y}
  \phi ^{*}_{\vec{k} + \qvec{q} n s}(y) s \sigma ^{y}
  (E_{k_z m  s} + E_{k_z+\qvec{q}_z n s} - 2 \mu)
  \phi _{\vec{k} m s} (y).
\end{equation}
As
\begin{equation}
  \label{eq:107}
  % e^{\theta /2 \sigma _{x}} \sigma _{y} e^{\theta /2 \sigma _{x}} = 1
  \sigma _{y} e^{\theta /2 \sigma _{x}} = e^{-\theta /2 \sigma _{x}} \sigma _{y}
\end{equation}
we get the very fortunate result
\begin{equation}
  \label{eq:108}
  \phi^{*} \sigma _{y} \phi
  = \frac{1}{\mathcal{N}^{*} \mathcal{N}} \tilde{\phi}^{*} \sigma _{y} \tilde{\phi}
  = \alpha \tilde{\phi}^{*} \sigma _{y} \tilde{\phi}.
\end{equation}
The first term of the stress-energy tensor thus has the exact same form as the untilted case, however with a prefactor \( \alpha \) and using the transformed coordinates \( \chi \).
We thus get
\begin{multline}
  \label{eq:109}
  T_{\vec{k}+\qvec{q} ns, \vec{k} m s}^{0y (1)} (\vec{q}) =
  \frac{is \alpha}{2} (E_{k_z m s} + E_{k_z + \qvec{q}_z n s} - 2 \mu)
  \int \mathrm{d} y
  e^{i q_{y} y}
  e^{-\frac{1}{2} (\chi_{\vec{k}+\qvec{q}}^2 + \chi_{\vec{k}} ^2) }\\
  [
  -a_{\vec{k} + \qvec{q} ns} b_{\vec{k} m s} H_{N-1} (\chi_{\vec{k} + \qvec{q}}) H_{M} (\chi_{\vec{k}})
  + b_{\vec{k} + \qvec{q} n s} a_{\vec{k} ms} H_{N}(\chi_{\vec{k} + \qvec{q}}) H_{M-1} (\chi_{\vec{k}})
  ].
\end{multline}
We will perform once again the completion of the square and substituion of \(y\).
The exponent is the same as that which we found for the current operator case, Eq. \eqref{eq:100}, with the change \(q_{y} \to - q_{y}\).
We thus make the change of variables
\begin{equation}
  \label{eq:110}
  \tilde{y} = \frac{\sqrt{\alpha}}{l_{B}} \left(y  - \frac{l_{B}^2}{2 \alpha } (i q_{y} + (2k' _x + q' _x) )\right),
\end{equation}
giving
\begin{align}
  \chi _{\vec{k}} &= \tilde{y} + \frac{l_{B}}{2 \sqrt{\alpha }} \left( q'_x + i q_{y}\right),\\
  \chi _{\vec{k} + \qvec{q}} &= \tilde{y} + \frac{l_{B}}{2 \sqrt{\alpha }} \left( -q' _{x} + i q_{y}\right).
\end{align}
Thus, after inserting and employing the defining relations for the \( \Xi_i \) functions, the matrix element reads
\begin{align}
  T_{\vec{k} + \qvec{q} n s, \vec{k} m s}^{0y} (\vec{q}) &=
  \frac{is \alpha }{2 }
  \frac{E_{k_z m s} + E_{k_z + \qvec{q}_z n s} - 2 \mu}{
    \sqrt{\alpha_{k_z m s}^2 + 1}
    \sqrt{\alpha_{k_z + \qvec{q}_z n s}^2 + 1}
  }\\
  &\pe \exp \left[
    -\frac{l_{B}^2}{4 \alpha } (q_{y}^2 - 2i ( 2k' _x + q' _x ) q_{y} + (q' _x)^2)
  \right  ]\\
  &\pe (-\alpha_{k_z + \qvec{q}_z n s} \Xi_{2} (\bar{\vec{q}}, m, n, s) + \alpha_{k_z m s} \Xi_{1}(\bar{\vec{q}}, m, n, s)),
\end{align}
where \(\bar{\vec{q}} = (q_{x}, -q_{y}, q_{z})\).

\begin{summary}
  \label{summary:tilt:matrixelements}
  In summary we have
  \begin{align}
    \label{eq:112}
    J_{\vec{k} m s; \vec{k} + \qvec{q} n s} (\vec{q}) &=
                                                        v_{F} e s \alpha^2
                                                        \Gamma ^-_{\vec{k} \qvec{q} m n s}
                                                        % \sum\limits_{i=1}^{4} \chi_{i} (\vec{q}, m, n, s)
                                                        \left[ \alpha_{k_z m s} \Xi_{1} (\vec{q}, m, n, s)
                                                        + \alpha_{k_z + \qvec{q}_z n s} \Xi _{2} (\vec{q}, m, n, s) \right],\\
    T_{\vec{k} + \qvec{q} n s, \vec{k} m s}^{0y} (\vec{q}) &=
                                                             \frac{is \alpha}{2}
                                                             (E_{k_z m s} + E_{k_z + \qvec{q}_z n  s} - 2 \mu) \Gamma ^+_{\vec{k} \qvec{q} m n s} \\
    \nonumber &\times [- \alpha_{k_z + \qvec{q}_z n s} \Xi_{2} (\bar{\vec{q}}, m, n, s) + \alpha_{k_z m s} \Xi_{1}(\bar{\vec{q}}, m, n, s)],
  \end{align}
  with \( \bar{\vec{q}} = (q_x, -q_y, q_z) \) and
  \[
    \Gamma _{\vec{k} \qvec{q} m n s}^{\pm} =
    \frac{
      \exp
      \left[
        - \frac{l_{B}^2}{4 \alpha } (q_{y}^2 + (q' _x)^2) \pm  i q_y l_B^2 (k' _x + \frac{q'_x}{2} )
      \right]
    }{
      \left[(\alpha_{k_z m s}^2 +1) (\alpha_{k_z + \qvec{q}_z ns} ^2 + 1)\right]^{\frac{1}{2}}
    }.
  \]
\end{summary}
\subsection{Static limit and dimensionless form of the matrix elements}
We are interested in the response in the static limit \( \vec{q} \to  0 \).
We may use the property of limits that
\[
  \lim_{n \to a} A \cdot  B = \lim_{n \to a} A \cdot \lim_{n \to a} B.
\]
We may thus consider the limits of the current and energy-momentum matrix elements separately, which we will do here.
Furthermore, to facilitate for more easily solving the integration later, we will introduce dimensionless quantities.

Let the dimensionless energy and momentum \( \epsilon_{\kappa_z m s} = v_F \sqrt{2 e B} E_{k_{z} m s}, \; \kappa_z = \sqrt{2 e B} k_{z}  \).
Consider firstly the exponent in the \( \Gamma ^{\pm } \) factor from summary \ref{summary:tilt:matrixelements},
\[
  \Gamma^{\pm}_{\vec{k} \qvec{q} m n s} \propto
  \exp
  \left[
    - \frac{l_{B}^2}{4 \alpha } (q_{y}^2 + (q' _x)^2) \pm  i q_y l_B^2 (k' _x + \frac{q'_x}{2} )
  \right].
\]
Define
\begin{equation}
P = \lim_{\vec{q} \to 0} \frac{l_B q_x'}{\sqrt{2 \alpha } } = \frac{\beta}{\sqrt{\alpha}} (\epsilon^0_{n, \alpha B} - \epsilon^0_{m, \alpha B}),
\end{equation}
where \( q'_x \) was defined in Eq.~\eqref{eq:qkprime},
\begin{equation*}
  q' _x = q_x \alpha  - \frac{\beta}{v_{F} }( E^0_{n,\alpha B} - E^0_{m, \alpha B} ).
\end{equation*}
In the limit, the exponent is thus
\begin{equation}
  \lim_{\vec{q}\to 0} \Gamma_{\vec{k} \qvec{q} m n s} \propto
  \exp
  \left[
    - \frac{\beta ^2 }{2 \alpha} ( \epsilon ^0_{n,\alpha B} - \epsilon ^0_{m, \alpha B} )^2
  \right].
\end{equation}

The normalization factor \( \alpha_{k_z m s} \) is independent on \( \vec{q} \), and already dimensionless.
Explicity, it is given in dimensionless quantities as
\begin{equation}
  \alpha_{k_z m s} =
  -\frac{\sqrt{2 e \alpha B M}}{ \frac{E_{k_{z} m s} - t_{\parallel} v_F k_z}{v_{F} s \alpha } - k_z}
  = -\frac{\sqrt{\alpha M}}{s \epsilon ^{0}_{m, \alpha B} - \kappa }.
\end{equation}

In the tilted case, the \( \Xi_{i} \) functions, defined in Eqs.~(\ref{eq:xi1def}, \ref{eq:xi2def}), do not have a trivial form in the static limit, as was the case in the untilted case.
Expressed in the quantities introduced here, they simplify to
\begin{subequations}
  \begin{align}
    \Xi_1 ^{(1)}(\vec{q}, m, n, s) &= \sqrt{\frac{2^N (M-1)!}{2^{M-1} N!}}
                                     \left( \frac{P}{\sqrt{2}} \right)^{N-M + 1}
                                     L^{N-M+1}_{M-1} \left( P^2 \right),\\
                                     %%%
    \Xi_1 ^{(2)}(\vec{q}, m, n, s) &= \sqrt{\frac{2^{M-1} N!}{2^N (M-1)!}}
                                     \left( -\frac{P}{\sqrt{2}} \right)^{M-N - 1}
                                     L^{M - N - 1}_N \left( P^2 \right),
    % \Xi_1(\vec{q}, m, n, s) &=
    %                           \begin{cases}
    %                             \Xi _1 ^{(1)} & \text{if } N \geq M-1\\
    %                             \Xi _1 ^{(2)} & \text{if } N \leq M-1
    %                           \end{cases} \text{ for } M>0, N \geq 0,
  \end{align}
\end{subequations}
\begin{subequations}
  \begin{align}
    \Xi_2 ^{(1)}(\vec{q}, m, n, s) &= \sqrt{\frac{2^{N-1} M!}{2^M (N-1)!}}
                                     \left( \frac{P}{\sqrt{2}} \right)^{N-1 - M}
                                     L^{N-1 -M}_{M} \left( P^2 \right),\\
                                     %%%
    \Xi_2 ^{(2)}(\vec{q}, m, n, s) &= \sqrt{\frac{2^M (N-1)!}{2^{N-1} M!}}
                                     \left( -\frac{P}{\sqrt{2}} \right)^{M-N + 1}
                                     L^{M - N + 1}_{N-1} \left( P^2 \right),
    % \Xi_2(\vec{q}, m, n, s) &=
    %                           \begin{cases}
    %                             \Xi _2 ^{(1)} & \text{if } N-1 \geq M\\
    %                             \Xi _2 ^{(2)} & \text{if } N-1 \leq M
    %                           \end{cases} \text{ for } M \geq 0, N > 0.
  \end{align}
\end{subequations}
Lastly, notice that in the static limit, the entire expression of the response function is independen of \( k_x \), and so the same procedure as was done for the untilted cone in section \ref{sec:response_notilt} is valid for the tilted cone, replacing the \( \vec{k} \) sum with an integral over \( k_z \) and a degeneracy factor
\begin{equation}
  \label{eq:113}
  \sum\limits_{\vec{k}}^{} \to \frac{\mathcal{V} e B}{(2\pi)^2 } \int \mathrm{d} k_z.
\end{equation}
Importantly, the degeneracy factor does \emph{not} depend on the renormalized magnetic field \( \alpha B \), but rather \( B \) itself.

\subsection{Perpendicular tilt}
\label{sec:perptiltsum}
We consider here the specialized situation where \( \vec{t} = t_x \hat{x} \), i.e. only tilt perpendicular to the magnetic field.
The response function
\begin{multline*}
  \lim_{\omega \to 0} \lim_{\vec{q} \to 0} \chi^{xy}(\omega, \vec{q}) = \lim_{\eta \to 0}
  \frac{e B i v_F}{(2 \pi)^2}
  \sum\limits_{mn}^{} \int \mathrm{d}k_z [n_{\vec{k} m s} - n_{\vec{k} n s}]\\
  \times \frac{
    J^x_{\vec{k} m s, \vec{k} ns} (\vec{q}\to 0) T^{y 0}_{\vec{k} n s, \vec{k}ms}(\vec{q} \to 0)
  }{
    (E_{k_z m s} - E_{k_z ns} + i \eta)(E_{k_z m s}-E_{k_z ns} + i \eta)
  }.
\end{multline*}
Writing out the matrix products we have
\begin{multline}
  J^x_{\vec{k} m s, \vec{k} ns} (\vec{q} \to 0) T^{y 0}_{\vec{k} n s, \vec{k}ms}(\vec{q} \to 0)
  =
  \frac{v_F e i \alpha^{3}}{2}
  e^{-P^2}\\
  \frac{
    (E_{k_{z} m s} + E_{k_z n s})
    (\alpha_{k_z m s}^2 \Xi_1(0, m,n, s)^2 - \alpha_{k_z n s}^2 \Xi_2(0, m,n, s)^2)
  }{
    (\alpha_{k_z m s}^2 + 1)(\alpha_{k_z n s}^2 + 1)
  }.
  \label{eq:114}
\end{multline}
And so, inserting into the response function
\begin{multline}
  \label{eq:115}
  \lim_{\omega \to 0} \lim_{\vec{q} \to 0} \chi^{xy}(\omega, \vec{q}) = \lim_{\eta \to 0}
  \frac{- e^2 \alpha^3 v_F B }{2 (2 \pi)^2 }
  \sum\limits_{mn}^{}
  \int \mathrm{d}\kappa_z
  e^{-P^2}\\
  \frac{
    [n_{\kappa_z m s} - n_{\kappa_z n s}]
    (\epsilon_{\kappa_z m s} + \epsilon_{\kappa_z n s})
    (\alpha_{\kappa_z m s}^2 \Xi_1(0, m,n, s)^2 - \alpha_{\kappa_z n s}^2 \Xi_2(0, m,n, s)^2)
  }{
    (\alpha_{\kappa_z m s}^2 + 1)(\alpha_{\kappa_z n s}^2 + 1)
    (\epsilon_{\kappa_z m s} - \epsilon_{\kappa_z ns} + i \eta)^2
  },
\end{multline}
where we also made a change of variables \( k_z = \sqrt{2 e B} \kappa_z \).

We make the observation that \( \Xi _1(m, n) = \Xi _2(n, m) \), where it is important to note that \( P \) changes sign under interchange of \( m,n \).
The rest of the factors are invariant under the interchange \( m \leftrightarrow n \), except for the step functions, which gives an overall sign change.
Thus, using \( \Xi _1(m, n) = \Xi _2(n, m) \) and relabelling the summation indices we may consider
\[
  \alpha_{\kappa_z m s}^2 \Xi_1^2 - \alpha_{\kappa_z n s}^2 \to 2 \alpha_{\kappa_z m s}^2 \Xi_1^2.
\]

We may also simplify the step function expression.
Physically, the step function term corresponds to only considering transitions between states with energies of opposite sign.
For Type-I systems, which we are restricted to here as we consider currently only perpendicular tilt, the energy of the state with quantum number \( n \) has the same sign as \( n \) itself, excluding of course the zeroth state.
For the zeroth state, the sign of the energy is \( \sign(-s \kappa_z) \).
Using these considerations, we may make certain seletion rules for the sum.
In the \( (m,n) \)-plane, the first and third quadrant give no contribution, as there \( m n > 0 \), i.e. they have the same sign.
Our sum is thus restricted to the second and fourth quadrant.
It is easy to show that
\begin{equation}
  \label{eq:116}
  n_{\vec{k} m s} - n_{\vec{k} + \qvec{q} n s} =
  \begin{cases}
    \phantom{-} 0 & m n > 0 \text{ or  } m,n = 0,\\
    - \sign(m) & m,n \neq 0,\\
    \phantom{-} \sign(n) \theta(\sign(n) s \kappa) & m = 0,\\
    -\sign(m) \theta(\sign(m) s \kappa) & n = 0.
  \end{cases}
\end{equation}

Furthermore, the contributions from the second and fourth quadrant are equal, which we will now show.
The mapping \( (m,n, \kappa_z) \mapsto (-m, -n, -\kappa_z) \), i.e. a \( \pi \) rotation, transforms points from the \( m < 0 \) half plane to the \( m > 0 \) half plane, including mapping the second quadrant to the fourth quadrant.
We want to consider how the integrand in question transforms under such a mapping.
Recall
\begin{align*}
  \alpha_{\kappa_z m s} &= - \frac{\sqrt{\alpha M} }{s \epsilon_{m, \alpha B}^0 - \kappa_z},\\
  \epsilon_{m, \alpha B}^0 &= \sign(m) \sqrt{\alpha M + \kappa_z^2}, \quad m \neq 0.
\end{align*}
Under the above mapping, we have the following relations
\begin{align}
  \label{eq:117}
  \epsilon_{m, \alpha B}^0 &\mapsto -\epsilon_{m, \alpha B}^0,\\
  \alpha_{\kappa_z m s} &\mapsto -\alpha_{\kappa_z m s},\\
  P &\mapsto -P.
\end{align}
The \( \Xi \) functions also acquries a sign for some values of \( m,n \), however, we only consdier \( \Xi^2 \).
The integrand in Eq.~\eqref{eq:115} is thus invariant under the transformation from the second to the fourth quadrant, and so we may consider only the fourth quadrant, adding a degeneracy factor 2.

Lastly, completely analogous to the untilted case, the integrand only depend on \( s \) and \( \kappa_z \) through their product \( s \kappa_z \), and thus is invariant under \( (s, \kappa_z) \mapsto (-s, -\kappa_z) \).
As the integral spans all of \( \kappa_z \), the contribution is independent of the chirality \( s \), and may be calcluated for a specific choice, which is here taken to be \( s=+1 \).

\todo{Make a note about \( M = N\) always giving zero contributinos.
Maybe also show in figure.
This is important wrt. saying that \( \gamma_0 \) is all contributison withtin square etc.}

\begin{summary}
  The response of a perpendicularly tilted cone is given by
  \begin{equation}
    \label{eq:118}
    \lim_{\omega \to 0} \lim_{\vec{q} \to 0} \chi^{xy}(\omega, \vec{q}) =
    \frac{e^2v_F B }{2 (2 \pi)^2 }
    \gamma^{t_x}_N,
  \end{equation}
  with
  \begin{equation}
    \label{eq:119}
    \gamma^{t_x}_N =
    2 \alpha^3
    \sum\limits_{mn}^{N}
    \int \mathrm{d}\kappa_z
    e^{-P^2}
    \frac{
      (\epsilon_{\kappa_z m s} + \epsilon_{\kappa_z n s})
      \alpha_{\kappa_z m s}^2 \Xi_1(0, m,n, s)^2
    }{
      (\alpha_{\kappa_z m s}^2 + 1)(\alpha_{\kappa_z n s}^2 + 1)
      (\epsilon_{\kappa_z m s} - \epsilon_{\kappa_z ns})^2
    },
  \end{equation}
  \footnote{note to self. \( \eta \) was taken to zero}
  where the summation goes over \( m > 0, n \leq 0 \), capped at the Landau level \( N \).
  The integration limits are \( (-\infty, \infty) \), except for \( n = 0 \), where they are \( [0, \infty) \)\footnote{Where we have used \( s=+1 \).}.
  \todo{Make sure numerical prefactors are correct. In particular, have we included the 2 from restricting to half plane?}
\end{summary}


\begin{figure}[ht]
  \centering
  \tikzsetnextfilename{mnregiontx}
  \begin{tikzpicture}
    \draw[step=1cm,gray,very thin, xshift=0.5cm, yshift=-0.5cm] (-0.9,0.9) grid (4.9,-3.9);
    \draw[thick,->] (0,0) -- (5.5,0) node[anchor=north west] {\( m \)};
    \draw[thick,->] (0,-4.3) -- (0,0.5) node[anchor=south east] {\( n \)};
    \foreach \x in {1,2,3,4,5}
    \draw (\x cm,1pt) -- (\x cm,-1pt) node[anchor=south] {$\x$};
    \foreach \y in {0, -1,-2,-3,-4}
    \draw (1pt,\y cm) -- (-1pt,\y cm) node[anchor=east] {$\y$};

    \foreach\x in {0,...,5} {
      \foreach \y in {0,-1,-2,-3,-4} {
        \fill (\x, \y) circle[radius=1pt];
      }
    }
    % \fill[blue!20, opacity=0.4] (-0.5, 0) -- (4, -4.5) -- (-0.5, -4.5) -- (-0.5, 0);
    % \fill[red!20, opacity=0.4] (0, -0.5) -- (4.5, -5) -- (4.5, -0.5) -- (0, -0.5);
    \fill[blue!20, opacity=0.4] (1, 0) -- (5, -4) -- (5, 0) -- (1, 0);
    \fill[red!20, opacity=0.4] (1,0) -- (5, -4) -- (1, -4) -- (1,-1);
    \draw (1, 0) -- (5, -4);
  \end{tikzpicture}
  \caption{The region of \( (m,n) \) to sum over for a Type-I perpendicularly tilted cone.
    The black line represents the combinations that give a finite contribution also in the untilted case.
    As the cone is tilted, this sharp line ``diffuse'' into the red and blue regions as well.
    Note that, as \( \Xi_1 \) defined only for \( M>0\), the region with \( m=0 \) gives no contribution.}
  \label{fig:nmregion}
\end{figure}

\begin{figure}[ht]
  \centering
  \pgfplotsset{
  % colormap={X}{ gray(0cm)=(1); gray(1cm)=(0);},
  % colormap={mycool}{ rgb255(0)=(255,0,255); rgb255(8)=(0,128,255); rgb255(10)=(255,255,255);},
  colormap={temp}{rgb255=(36,0,217) rgb255=(25,29,247) rgb255=(41,87,255)
    rgb255=(61,135,255) rgb255=(87,176,255) rgb255=(117,211,255)
    % rgb255=(153,235,255)
    rgb255=(255,255,255)
    % rgb255=(255,214,153)
    rgb255=(255,172,117) rgb255=(255,120,87) rgb255=(255,61,61)
    rgb255=(247,40,54) rgb255=(217,22,48) rgb255=(166,0,33)},
  % colormap={mycool}{ rgb255(-1)=(255,0,255); rgb255(-0.1)=(255,0,255); rgb255(0)=(255,255,255); rgb255(1)=(0,128,255)},
  % colormap={violet}{rgb255=(25,25,122) color=(white) rgb255=(238,140,238)},
  point meta min=-0.1,
  point meta max=0.1,
  xlabel=\( m \),
  ylabel=\( n \),
}
\newcounter{plotnum}
\tikzsetnextfilename{contribmn-P}
\begin{tikzpicture}
  \newcommand\myaxislimit{9.5}
  \begin{groupplot}[
    width=0.5\textwidth,
    unit vector ratio=1 1 1,unit rescale keep size=false, %% Do not enlarge limit, keep unit equal
    xmin=-\myaxislimit, xmax=\myaxislimit,
    ymin=-\myaxislimit, ymax=\myaxislimit,
    % title={\( t_x= \title \)},
    title style={ at={(0.75, 0.7)} },
    group style={
      group size=2 by 2,
      group name=contribs,
      x descriptions at=edge bottom,
      y descriptions at=edge left,
      horizontal sep=8pt,
      vertical sep=8pt,
    },
    ]
    \pgfplotsforeachungrouped \datafile/\title in {
      % contribTilttx0.PbrokenFalse.csv/0,
      % contribTilttx0.125PbrokenFalse.csv/0.16,
      % contribTilttx0.25PbrokenFalse.csv/0.25,
      % contribTilttx0.5PbrokenFalse.csv/0.5
      contribmnTilttx0..csv/0,
      contribmnTilttx0.125.csv/0.125,
      contribmnTilttx0.25.csv/0.25,
      contribmnTilttx0.5.csv/0.5
    }{
      \edef\tmp{
        \noexpand\ifthenelse{\theplotnum=1}{
          \noexpand\nextgroupplot[title={$t_x = \title$}, colorbar right,
          every colorbar/.append style={height=
            2*\noexpand\pgfkeysvalueof{/pgfplots/parent axis height}+8pt, ytick={-0.1,0,0.1}},
          colorbar to name={mycolorbar}
          ]
          % See  https://tex.stackexchange.com/questions/126177/common-colorbar-for-groupplot
        }{
          \noexpand\nextgroupplot[title={$t_x = \title$}]
        }
        % \noexpand\nextgroupplot[title={$t_x = \title$}, colorbar, colorbar horizontal, colorbar to name={mycolorbar}]
        \noexpand\addplot [
        matrix plot*,
        % mesh/cols=14,
        point meta=explicit,
        ] table[meta index=2, col sep=comma] {data/\datafile};
        \noexpand\addplot [
        matrix plot*,
        % mesh/cols=14,
        point meta=explicit,
        ] table[meta index=2, x expr=-\noexpand\thisrowno{0}, y expr=-\noexpand\thisrowno{1}, col sep=comma] {data/\datafile}; %% Other half plane
      }
      \tmp

      \ifthenelse{\theplotnum=0}{
        \draw[dotted, gray] (axis cs:-1.5,-1.5) rectangle (axis cs:1.5,1.5) node[pos=0, pin={[pin edge={solid}]-180:\( \gamma_0 \)}] {};
        \draw[dotted, gray] (axis cs:-2.5,-2.5) rectangle (axis cs:2.5,2.5) node[pos=0, pin={[pin edge={solid}]-110:\( \gamma_1 \)}] {};


        % \draw[dotted, gray] (axis cs:0.5,-0.5) rectangle (axis cs:1.5,0.5) node[pos=0, yshift=4pt, pin={[pin edge={solid}]-140:\( \gamma_0 \)}] {};
        % \draw[dotted, gray] (axis cs:0.5,-1.5) rectangle (axis cs:2.5,0.5) node[pos=0, pin={[pin edge={solid}]-140:\( \gamma_1 \)}] {};
        % \draw[dotted, gray] (axis cs:0.5,-3.5) rectangle (axis cs:4.5,0.5) node[pos=0, pin={[pin edge={solid}]-140:\( \gamma_3 \)}] {};
      }{}
      \stepcounter{plotnum}
    }
  \end{groupplot}
  %% Shfit half vertical sep
  \node[anchor=west, yshift=4pt] at (contribs c2r2.north east) {\ref{mycolorbar}};
\end{tikzpicture}

  \caption{Contributions to \( \gamma_N \) from \( m\to n \) transitions for different values of \( t_x \).
    In order to retain contrast, the color values are capped at \( 0.1 \), meaning that the \( \gamma_0 \) contributions are clipped.
    Note that all quadrants of the \( m,n \)-plane is shown, although as proven in the main text, only the fourth quadrant needs to be computed, as the second quadrant contributions are equal.
    \label{fig:contribs}
    }
\end{figure}

\subsection{Tilt parallell to the magnetic field}
Even though the treatment above for a general tilt is valid for parallel tilt, the response can be found more directly from the untilted case.
For \( \vec{t} = t_z \hat{z} \), the energy momentum tensor \( T^{y 0} \), charge current \( J^x \), and wave functions \( \phi(\vec{r}) \) are all independent of \( t_z \), and the only difference compared to the untilted system is a change in the energies of the Landau levels.
We may thus immediately use the result from the untilted case
\begin{equation}
  \label{eq:120}
  \lim_{\omega \to 0} \lim_{\vec{q} \to 0} \chi^{xy} =
  - \frac{e^2 v_F B}{2 (2\pi)^2}
  \sum\limits_{mn} \int \mathrm{d} \kappa_z
  \xi(\kappa_z)
  (\epsilon_{\kappa_z m s} + \epsilon_{\kappa_z n s})
  (\alpha_{\kappa_z m s}^2 \delta_{M-1, N} - \alpha_{\kappa_z n s}^2 \delta_{N-1, M}),
\end{equation}
with
\begin{align}
  \label{eq:121}
  \epsilon_{\kappa_z m s} &=
                          \begin{cases}
                            t_z^s \kappa_z + \sign{m} \sqrt{M + \kappa_z^2} & m \neq 0,\\
                            (t_z^s - s) \kappa_z & m = 0,
                          \end{cases}\\
  \alpha_{\kappa_z m s} &=
                          -s \frac{\sqrt{M}}{\epsilon_{\kappa_z  m s}^0 - s \kappa_z },\\
  \lim_{\omega \to 0} \lim_{\vec{q} \to 0}\xi(\kappa_z) &= \frac{[n_{\kappa ms} - n_{\kappa ns}]
  \left[ (\alpha_{\kappa ms}^2 + 1) (\alpha_{\kappa ns}^2 + 1) \right]^{-1}
  }{
    (\epsilon _{\kappa m s} - \epsilon _{\kappa n s})^2
  }.
\end{align}
In the untilted case we made several simplifications to this expression, especially with regards to limiting the summation domain.
We will here consider which of those simplifications apply also in the case of tilt \( t_z \).

Under the transformation \( (m,n,\kappa_z) \mapsto (-m, -n , -\kappa_z) \), \( \xi(\kappa_z), \epsilon_{\kappa_z m s}, \alpha_{\kappa_z m s} \) are all still odd, and so the integrand is invariant under such a transformation.
As the integral is over all \( \kappa_z \), we may therefore consider only half the \( m,n \) plane, as was the case in the untilted case.
However, in the untilted case the sum was in fact restricted to only one quadrant, as at \( T\to 0 \) the transitions must be between states with energy of opposite sign.
In the case of Type-II systems, this requirement does not restrict the sum to one quadrant.
It is thus convenient to consider Type-I and Type-II separately.

In the untitled system, the contributions from the two chiralities where the same, as \( \kappa_z \) and \( s \) always appeared in conjunction, \( \kappa_z s \).
In the case of \( t_z \) tilt, this is not the case.
The proof for the response from the two chiralities being the same in the untilted case was that \( s \) and \( \kappa_z \) appeared only through the product \( s \kappa_z \), and so the expression was invariant under \( (s, \kappa_z) \mapsto (-s, -\kappa_z) \).
As our integration spans all \( \kappa_z \), the total response is invariant under \( s \to -s \).
The tilt parameter enters the expression only through \( \epsilon_{\kappa_z m s} = \epsilon_{\kappa_z m s}^0 + \kappa_z t^s_z \), and in the inversion symmetric case, \( t^s_z = s t_z \), the argument still holds.
In the case of broken inversion symmetry, however, where \( t^s_z = t_z \), the argument fails.
A similar argument may, however, be made for the transformation \( (s, \kappa_z, t_z) \mapsto (-s, -\kappa_z, -t_z) \), for which the (inversion broken) system is invariant.
The response of a cone with chirality \( s = -1 \) is thus equal the response with \( s = +1 \) and \( t_z \to -t_z \).
We therefore compute all responses for \( s=+1 \);
for symmetric systems the response is equal for \( s=-1 \), while for broken inversion symmetry, the response is given at \( t_z \to -t_z \).

\subsubsection{Type-I}
In Type-I systems, the selection rules from the step functions are independent of \( t_z \), and the only difference from the untilted case is the term \( \epsilon_{\kappa_z m s} + \epsilon_{\kappa_z n s} = \epsilon^0_{\kappa_z m s} + \epsilon^0_{\kappa_z n s} + 2 \kappa_z t^s_z \).
The response is therefore
\begin{equation}
  \label{eq:122}
  \lim_{\omega \to 0} \lim_{\vec{q} \to 0} \chi^{xy} = \frac{e^2v_F B}{2 (2\pi)^2} (\gamma_N^0 + \gamma_{\text{div}, N}),
\end{equation}
where \( \gamma_N^0 \) is the prefactor of the untilted case, and according to Eq.~\eqref{eq:93}
\begin{equation}
  \label{eq:123}
  \gamma_{\text{div}, N} = -4 \sum\limits_{i=0}^{N} \int \mathrm{d} \kappa_z \xi(\kappa_z)
  2 \kappa_z t^s_z \alpha_{\kappa_z m s}^2 \Big|_{\overset{m=i+1}{n=-i}},
\end{equation}
which has an UV divergence.
Introduce the momentum cutoff \( \Lambda \), in which case the integral can be solved analytically, with the result\footnote{Note the minus sign introduced by the step function in \( \xi \).}
\begin{equation}
  \label{eq:124}
  \gamma_{\text{div}, 0} = 2 t_z \left(\Lambda  \left(\Lambda -\sqrt{\Lambda ^2+1}\right)+\sinh ^{-1}(\Lambda
   )\right)
\end{equation}
and the contribution from each term of the sum
\begin{multline}
  \gamma_{\text{div}, N} - \gamma_{\text{div}, N-1} =
  2 t_z
  \Bigg\{
    \Lambda\left(\sqrt{\Lambda^2 + N} - \sqrt{\Lambda^2 + N + 1}  \right)\\
    + (N + 1) \tanh^{-1}\left[\frac{\Lambda}{\sqrt{\Lambda^2 + N + 1} } \right]
    - N \tanh^{-1}\left[\frac{\Lambda}{\sqrt{\Lambda^2 + N}}\right]
    \Bigg\},
    \label{eq:125}
\end{multline}
where we used the selection rule of the sum \( N = M - 1 \) and \( m>0, n<0 \).
This contribution is shown in figure \ref{fig:divergent-factor}.
\todo{is it ok to write 'the contribution \eqref{eq:125}', or must it always be 'the contribution Eq. \eqref{eq:125}'?}
The contribution \eqref{eq:125} is odd in \( t_z \), and so for systems with broken inversion symmetry, the total contribution from two cones cancel.

Assuming \( \Lambda \gg 1 \) the expression is approximated by
\begin{equation}
  \label{eq:126}
  t_z
  \left(
    \left[
  -1 + N \log\left(\frac{N}{N+1}\right) - \log \frac{N+1}{4}
  \right]
 + 2 \log\Lambda
\right) + \mathcal{O}\left(\frac{1}{\Lambda^2}\right).
\end{equation}


\begin{figure}[htp]
  \centering
  \tikzsetnextfilename{divergentContribCutoff}
  \begin{tikzpicture}
    \begin{axis}[
      xlabel=\( \Lambda \),
      ylabel=Contribution,
      legend pos=south east,
      % cycle list name=exotic,
      cycle list={
        teal,every mark/.append style={fill=teal!80!black},mark=*\\
        orange, dashed, every mark/.append style={fill=orange!80!black},mark=square*\\
        cyan!60!black, densely dotted, every mark/.append style={fill=cyan!80!black},mark=otimes*\\
        red!70!white, dashdotted, mark=star\\
        lime!80!black, loosely  dashed, every mark/.append style={fill=lime},mark=diamond*\\
        red,densely dashed,every mark/.append style={solid,fill=red!80!black},mark=*\\
        yellow!60!black,densely dashed,
        every mark/.append style={solid,fill=yellow!80!black},mark=square*\\
        black,every mark/.append style={solid,fill=gray},mark=otimes*\\
        blue,densely dashed,mark=star,every mark/.append style=solid\\
        red,densely dashed,every mark/.append style={solid,fill=red!80!black},mark=diamond*\\
      },  %% exotic with various line types
      ]
      \foreach \i in {1,...,5} {
        \addplot+[mark=none] table[x index=0, y index=\i, col sep=comma] {data/divergentFactor.csv};
        \addlegendentryexpanded{\( m = \i \)};
      }
    \end{axis}
  \end{tikzpicture}
  \caption{The divergent factor \( \gamma_{\text{div}, N} / t_{z} \) for the first Landau levels, as a function of the momentum cutoff \( \Lambda \).
  \label{fig:divergent-factor}}
\end{figure}


\subsubsection{Type-II}
For Type-I semimetals, the sign of energy state \( m \neq 0 \) is given by the sign of \( m \) itself.
For \( m = 0 \) the sign of the energy is given by \( -s \sign{\kappa } \).
Due to this, the sum is restricted to \( n=M+1, m=-M \) and \( n=-M-1, m=M \).
In the case of Type-II, however, the situation is not so simple.
The energy bands cross the Fermi surface, and we must also include in our sum overlap between states of the same sign, i.e. \( n=M+1, m=M \) and \( n=-M-1, m=-M \), which is non-zero for certain intervals of \( \kappa  \).
See plot of the tilted Landau levels in figure~\ref{fig:llevelstilt}.

% Furthermore, the expression is no longer invariant under \( (s, \kappa_z) \mapsto (-s, -\kappa_z) \).
% For inversion symmetric systems, \( \vec{t^s} = s \vec{t} \), the symmetry is still conserved, however, for broken inversion symmetry, the extra term in the energies, \( t^s_z \kappa_z \) breaks the symmetry between the chiralitites.
% As the energies change between the chiralities, so does the step function, giving other selection rules.
% For inversion symmetric systems, we may thus simply choose a value of \( s \), for example \( s=1 \), and compute all results for that value.
% For broken inversion symmetry, however, the results must be computed separately for the two chiralities.
% \todo{Or maybe it is easier to do all calculations with \( t_x^s \) (and its sign) in focus, and then later apply inversion (broken) symmetry?}

In order to find explicitly the limits of integration for the Type-II case, we must find the roots of the energy levels.
The zeroth Landau level always has only one root, which is in the origin.
For the higher order Landau levels, we solve
\begin{equation}
  \label{eq:127}
  \epsilon_{\kappa_z m s } = t_z^s \kappa_z + \sign(m) \sqrt{M + \kappa_{z} ^2} = 0,
\end{equation}
whose solution is
\[
\kappa_z^2 = \frac{M}{t_{z}^2 - 1}.
\]
The actual roots of the energies are
\begin{equation}
  \label{eq:128}
  \kappa_z = -\sign(m t^s_x) \sqrt{\frac{M}{t_{z}^2 - 1}}.
\end{equation}
The integration limit for the \( 0 \to 1 \) transition is thus, for \( t_z > 1 \), \( [-\sqrt{t_z^2 - 1 }^{-1}, 0] \).
The \( 1\to 2 \) transition is \( [-\sqrt{2} /\sqrt{t_z^2 - 1}, -\sqrt{t_z^2 - 1 }^{-1}] \), and so forth.
The general \( n \to m \) transition has the integration limits
\[
  \left[-\sign(t_z) \sqrt{\frac{m}{t_{z}^2 - 1}}, -\sign(t_z n) \sqrt{\frac{-n}{t_{z}^2 -1}} \right].
\]

The \( 0\to 1 \) transitions was computed analytically, and found to be
\begin{equation}
  \label{eq:129}
  \gamma_0 =
  2 \sign(t_z)
  \left(
  |t_z| \sinh^{-1}\left(\frac{1}{\sqrt{t_{z}^2-1} }\right) -1
\right  ).
\end{equation}
For a general \( n \to m, \; N>0, M=N+1 \) transition, the contribution \( \gamma_N - \gamma_{N-1} \) was found to have very lengthy expressions.
Consult Table~\ref{tab:typeii-expressions} to find the appropriate expressions for positive and negative tilt, and interband and intraband transitions.

\begin{table}[ht]
  \centering
  \renewcommand\arraystretch{2.5}
  \begin{tabular}{ll | c | c |}
    &\multicolumn{1}{c}{}&\multicolumn{2}{c}{\textbf{Tilt direction}}\\[-2ex]
    &\multicolumn{1}{c}{}
                         &\multicolumn{1}{c}{\( t_x > 1 \)}&\multicolumn{1}{c}{\( t_x < -1 \)}\\
    \cline{3-4}
    \multirow{2}{*}{\rotatebox{90}{\textbf{Band type}}}
    &\( n < 0 \)& Lst.~\ref{lst:typeii-interband-tzpos} & Lst.~\ref{lst:typeii-interband-tzneg}\\
    \cline{3-4}
    &\( n > 0 \)& Lst.~\ref{lst:typeii-intraband-tzpos} & Lst.~\ref{lst:typeii-intraband-tzneg}\\
    \cline{3-4}
  \end{tabular}

  \begin{tabular}{l | c | c |}
    \multicolumn{1}{c}{}
                         &\multicolumn{1}{c}{\( t_x > 1 \)}&\multicolumn{1}{c}{\( t_x < -1 \)}\\
    \cline{2-3}
    \( n < 0 \)& Lst.~\ref{lst:typeii-interband-tzpos} & Lst.~\ref{lst:typeii-interband-tzneg}\\
    \cline{2-3}
    \( n > 0 \)& Lst.~\ref{lst:typeii-intraband-tzpos} & Lst.~\ref{lst:typeii-intraband-tzneg}\\
    \cline{2-3}
  \end{tabular}

  \caption{
    % The expression for the \( m\to n; N> 0, M=N+1 \) transition is given by the equation number at the various choices of \( t_z \) and \( \sign (n) \).
    Decision matrix for the expression of the \( m \to n; N > 0, M =N+1 \) transition over different regions.
    Expressions given in Mathematica code format.
    Code listings starting on page \pageref{lst:typeii-interband-tzpos} in \cref{cha:long_expression}.
    See main text for details.
    \todo{Make sure I am not hanged for having vertical lines}
  }
  \label{tab:typeii-expressions}
\end{table}


\section{Results}
As described above, the contribution from the cone with chirality \( s = -1 \) can be found from the result of the positive chirality cone.
In the case of perpendicular tilt, they are exactly the same.
In the case of parallel tilt, it depends on the symmetry of the tilt.
For systems with broken inversion symmetry, the response from the two cones are the same.
On the other hand, for inversion symmeric systems, the contribution form the cone with chirality \( s=-1 \) is the same as that of the \( s=+1 \) cone at the opposite tilt \( t_z \to - t_z \).
Therefore, it is useful to separate the contribution into even and odd components, for finding the total contribution from the two cones combined.
For some contribution \( \gamma(t_{x /z}) \), we define
\begin{align}
  \gamma_{\text{even}}(t_{x/z}) &= \frac{\gamma(t_{x/z}) + \gamma(-t_{x/z})}{2}\label{eq:130},\\
  \gamma_{\text{odd}}(t_{x/z}) &= \frac{\gamma(t_{x/z}) - \gamma(-t_{x/z})}{2}\label{eq:131}.
\end{align}
All results will be given in terms of these components, at \( t_{x /y} > 0 \).

We will here consider parallel and perpendicular tilt separately.

\subsection{Perpendicular tilt}
In the case of a tilt perpendicular to the magnetic field, we are, as previously explained, restricted to Type-I materials, as the Landau level description breaks down for Type-II perpendicular tilt.
Importantly, this does not generally mean that the effect is not present for Type-II systems, but simply that the Linear model Landau level description is not a good basis for the system.
The collapse of the Landau levels caused \textcite{soluyanovTypeIIWeylSemimetals2015} to errenously predict the collapse of the chiral anomaly in their now famous paper first describing Type-II Weyl semimetals.

As explained in section \ref{sec:perptiltsum}, the \( m,n \) summation is restricted to the fourth quadrant in the \( m,n \) plane.
In the case of no tilt, only contributions from \( M = N + 1 \) were non-zero;
we named the contribution from the \( 0\to 1 \) transition \( \gamma_0 \), the \( -1\to 2 \) transition \( \gamma_1 \) and so fourth.
Here, as there are contributions also away from the \( M=N + 1 \) line, we denote by \( \gamma_0 \) the contributions from inside the square of length 1 centered at the origin.
The \( \gamma_1 \) contributions are those inside the square of length 2, and so fourth.
This definition effectively sets a roof to which Landau levels we consider.
This is indicated in figure \ref{fig:contribs}.

\todo{Correct which values }
The integral was computed numerically for \( M,N \leq 6 \) over different values of \( t_x \) with \( t_z = 0 \), shown in figure \ref{fig:contribs}.
The total contribution \( \gamma_N \) as a function of \( N \) is shown in figure \ref{fig:total_contribs}.
The contribution is even in \( t_x \), and the two cones have the same contribution, as shown analytically in section \ref{sec:perptiltsum}.

\begin{figure}[ht]
  \centering
  \tikzsetnextfilename{contribtx}
  \begin{tikzpicture}
    \begin{axis}[
      legend columns=-1,
      legend to name=contribs,
      xlabel=Number of Landau levels,
      ylabel=\( \gamma_N \),
      width=12cm,
      height=8cm,
      name=myaxis,
      xtick distance=1,
      ]
      \newcommand\plotContrib[1]{
        \addplot table[col sep=comma] {data/contribSumTilttx#1PbrokenTrue.csv};
        % \addlegendentry{\( t_x = \num[minimum-decimal-digits=3]{#1} \)}  %% Make all same length
        \addlegendentry{\( t_x = \num[minimum-decimal-digits=1]{#1} \)}
      }
      \plotContrib{0.}
      \plotContrib{0.125}
      \plotContrib{0.25}
      \plotContrib{0.375}
      \plotContrib{0.5}
      % \addplot table[col sep=comma] {data/contribSumTilttx0.PbrokenTrue.csv};
      % \addplot table[col sep=comma] {data/contribSumTilttx-0.PbrokenTrue.csv};
      % \addplot table[col sep=comma] {data/contribSumTilttx-0.125PbrokenTrue.csv};
      % \addplot table[col sep=comma] {data/contribSumTilttx-0.25PbrokenTrue.csv};
      % \addplot table[col sep=comma] {data/contribSumTilttx-0.375PbrokenTrue.csv};
      % \addplot table[col sep=comma] {data/contribSumTilttx-0.5PbrokenTrue.csv};
    \end{axis}
    \node[anchor=south] at (myaxis.north) {\ref{contribs}};
  \end{tikzpicture}
  \caption{\label{fig:total_contribs} }
\end{figure}

\begin{figure}[htb]
  \centering
  \tikzsetnextfilename{contribtx-zerothll}
  \begin{tikzpicture}
    \begin{axis}[
      xlabel=\( t_x \),
      ylabel=\( \gamma_0 \),
      legend pos=south west,
      cycle multiindex* list={
        [samples of colormap=5 of viridis]\nextlist
        mark=o\\mark=oplus\\mark=square\\
      },
      mark repeat={4},
      ]
      \foreach \i in {1,3,5,7} {
      \addplot table[y index=\i, col sep=comma] {data/contribTilttx-zerothll.csv};
      \addlegendentryexpanded{\( N = \i \)};
      % \addplot[mark=o] table[x expr=-\thisrowno{0}, y index=\i, col sep=comma] {data/contribTilttx-zerothll.csv};
      }
    \end{axis}
  \end{tikzpicture}
  \caption{Numerically computed values of the prefactor \( \gamma_N \) with only the first Landau level included for perpendicular tilt \( t_x \).
    The contribution is even in \( t_x \), and vanish as \( |t_x| \to 1 \).
    \todo{This is only 0->1 transition, but for tx tilt, we also have 0-> higher order. Should we include these}
  }
\end{figure}


\subsection{Parallel tilt}
\todo{Should we also compute the momentum cutoff for nontilted terms?}
In the Type-I regime, the contributions differ from that of the untitled system by \( \gamma_{\text{div}, N} \), Eq. \eqref{eq:125}, dependent on a momentum cutoff \( \Lambda \).
The contribution is odd in \( t_z \), so for systems with broken inversion symmetry, the two chiralities cancel, and the response is equal to the untilted case.
In case of inversion symmetry, the contributions from the two chiralities are equal and add up.
In the large cutoff limit, the divergence goes like \( \log \Lambda \), where the dimensionfull cutoff \( k^{\text{cutoff}}_z = \sqrt{2 e B} \Lambda  \).

In the Type-II regime, the contributions have more complicated form.
Considering firstly only the lowest Landau level contribution, Eq. \eqref{eq:129},, which is odd in \( t_z \), the total contribution cancel between the chiralitites for broken inversion symmetry, while it adds up for inversion symmetric systems.
As \( |t_z| \to 1 \) from above, the contribution blows up.
This is to be expected as we move towards the Lifshitz transition, where we expect the linear model to perform poorly.
\footnote{As the Fermi surface of the linear model is vastly different from the Fermi surface of the tight binding model.
See \textcite{vanderwurffMagnetovorticalThermoelectricTransport2019}}
\todo{put this on more solid footing.
  Discuss how the Fermi surface is massively wrong there
}
The contribution goes to zero as \( t_z \to \infty \), shown in figure \ref{fig:contribtzII}.

Considering also higher Landau level contributions, both interband and intraband transitions must be included,
\footnote{By band we here refer to the ``conduction'' band and ``valence'' band}
meaning the summation is no longer restricted to a quadrant in the \( m,n \) plane, but rather to half the plane.
The contributions are shown in figure \ref{fig:contribtzII}.
These contributions are not odd in \( t_z \) -- they have a finite even component.
Due to this, the contribution does not cancel for inversion broken systems, however the contribution is small in magnitude compared to the other contributions.

A schematic plot of all the contributions of a parallel tilt is shown in figure \ref{fig:scetch}.


\begin{figure}[ht]
  \centering
  \subcaptionbox{Intraband contributions, \( -N \to N + 1 \).}{
  \tikzsetnextfilename{tzcontribtypeii}
  \begin{tikzpicture}
    \pgfkeys{/pgf/declare function={arcsinh(\x) = ln(x + sqrt(x^2+1));}}
    \begin{axis}[
      width=0.49\textwidth,
      domain=1:2,
      xlabel=\( t_z \),
      ylabel=Contribution,
      samples=100,
      legend entries={%
        \( \hphantom{-}0\rightarrow 1\),
        \( -1 \rightarrow 2\),
        \( -2\rightarrow 3 \),
        \( -3\rightarrow 4 \),
      }, %% There is a bug with \to
      % cycle list={[samples of colormap={4} of colormap/autumn]},
      % cycle list={[of colormap=colormap/autumn]},
      ]
      % \addplot[mark=none] {0.5 * (x * arcsinh(1/sqrt(x^2-1)) - 1)};
      % \addplot+[mark=none] {(-1 + x * ln((1+x)/sqrt(x^2-1)))}; %% Same as above, just more stable
      \addplot+[mark=none] table[col sep=comma, y=1] {data/contribtypeii_interband_odd.csv};
      \addplot+[mark=none] table[col sep=comma, y=2] {data/contribtypeii_interband_odd.csv};
      \addplot+[mark=none] table[col sep=comma, y=3] {data/contribtypeii_interband_odd.csv};
      \addplot+[mark=none] table[col sep=comma, y=4] {data/contribtypeii_interband_odd.csv};

      \pgfplotsset{cycle list shift=-3}  %% Change shift to number of transitions > 0
      \addplot+[mark=none, dashed] table[col sep=comma, y=2] {data/contribtypeii_interband_even.csv};
      \addplot+[mark=none, dashed] table[col sep=comma, y=3] {data/contribtypeii_interband_even.csv};
      \addplot+[mark=none, dashed] table[col sep=comma, y=4] {data/contribtypeii_interband_even.csv};
    \end{axis}
  \end{tikzpicture}
  } %% Subcaptionbox
  \subcaptionbox{Interband contributions, \( N \to N + 1 \).}{
  \tikzsetnextfilename{tzcontribtypeii_intraband}
  \begin{tikzpicture}
    \begin{axis}[
      width=0.49\textwidth,
      domain=1:2,
      xlabel=\( t_z \),
      % ylabel=Contribution,  %% Enough with one
      legend entries={%
        \( 1 \rightarrow 2\),
        \( 2\rightarrow 3 \),
      }, %% There is a bug with \to
      ]
      % \addplot[mark=none] {0.5 * (x * arcsinh(1/sqrt(x^2-1)) - 1)};
      \addplot+[mark=none] table[col sep=comma, y=2] {data/contribtypeii_intraband_odd.csv};
      \addplot+[mark=none] table[col sep=comma, y=3] {data/contribtypeii_intraband_odd.csv};

      \pgfplotsset{cycle list shift=-2}
      \addplot+[mark=none, dashed] table[col sep=comma, y=2] {data/contribtypeii_intraband_even.csv};
      \addplot+[mark=none, dashed] table[col sep=comma, y=3] {data/contribtypeii_intraband_even.csv};
    \end{axis}
  \end{tikzpicture}
  } %% subcaptionbox
  \caption{The contribution from \( n\to m \) transitions in a Type-II \(t_z\) tilted system.
    % Shown in dashed line is the difference between the contribution at \( t_z \) positive and negative.
    Shown in dashed line of corresponding color, is the even component of the contribution, i.e. \( [\text{contrib}(|t_z|) + \text{contrib}(-|t_z|)] /2 \).
  }
  \label{fig:contribtzII}
\end{figure}

\begin{figure}[p]
  \centering
  \tikzsetnextfilename{schematic_tz}
  \begin{tikzpicture}
    \begin{axis}[
      ymax=1,
      xmin=0, xmax=2,
      width=13cm, height=10cm,
      legend entries={Even,Odd},
      xtick={0,1,2},
      ytick=\empty,
      xlabel=\( t_z \),
      ]
      \draw[thin, gray, dashed](axis cs:1,-1)--(axis cs:1,1.2);
      \addplot[forget plot, thin] {0};

      \addplot[mark=none, domain=0:1] {0.5} node[midway, pin=above:Untilted contribution] {};
      \addplot[mark=none, dashed, domain=0:1] {0.2 * x} node[midway, pin={[pin edge=solid]above:\( \propto \log\Lambda \)}] {};
      \addplot[dashed, mark=none] table[col sep=comma, y=-1] {data/contribtypeii.csv};
      \addplot[mark=none] table[col sep=comma, y=-1] {data/contribtypeii_diffsigntz.csv};
    \end{axis}
  \end{tikzpicture}
  \caption{
    Schematic summary of the contribution for perpendicular tilt \( t_z \).
    Shown is the even (solid line) and odd (dashed line) parts as a function of \( t_z \).
    As explained in the main text, the total contribution for a pair of cones is given by the sum of the even and even part in inversion symmetric systems, and by the odd part for broken inversion symmetry.
    \label{fig:scetch}
  }
\end{figure}

\FloatBarrier
\subsection{Draft -- Platau}
In real materials, the Fermi level is close to, but not exactly at, the Dirac point.
\textcite{arjonaFingerprintsConformalAnomaly2019} investigated this, which is of great interest with regards to experimental observations, by extending the computation to finite chemical potential and temperature.
For sufficiently large magnetic field, only the zeroth Landau level is filled \cite{arjonaFingerprintsConformalAnomaly2019,vozmedianoTheoreticalPhysicsColloquium2021}, and the only transitions are the \( 0 \to \pm 1 \) transitions.
For a chemical potential \( \mu \) small enough to be contained between the \( \pm 1 \) Landau levels, i.e. \( |\mu| /(v_F \sqrt{2eB} ) < 1 \), the response function was found to be invariant.
Furthermore, for a finite temperature, it was found that thermally activated carriers increased the magnitude of the effect, with a stable platau around \( \mu = 0 \).
The width of the plateau is inversely proportional to the temperature.
See figure \ref{fig:platau}.

As tilt is introduced, the energy interval in which one only has the zeroth Landau level is reduced, and as \( t \to 1 \) the interval vanishes.
So as the system is tilted, the width of the platau is reduced.
\todo{Comment on we found this also for non-symmetric energy-momentum tensor?}
\todo{Do the computation for tz? In that case, compute separately the even and odd component}
\todo{See also \cite{chernodubThermalTransportGeometry2021} FIG. 9 and discussion}

\begin{figure}[htb]
  \centering
  \tikzsetnextfilename{lllevelspotentialwinset}
\begin{tikzpicture}
  \begin{axis}[
    width=0.9\textwidth,
    height=0.63\textwidth,
    domain=-3:4,
    xmin=-2.5,xmax=4.2,
    ymin=-3, ymax=3,
    samples=50,
    xlabel=\( \kappa_z \),
    ylabel=\( \epsilon_n \),
    ]
    \addplot[mark=none, samples=2] {-x};
    \foreach \n in {1,2,3}{
      \addplot[mark=none, red] {sqrt(\n + x^2)};
      \addplot[mark=none, blue] {-sqrt(\n + x^2)};
    }

    \addplot[dashed, domain=-3:1.6] {1};
    \addplot[dashed, domain=-3:1.6] {-1};
    \addplot[thin, gray, domain=-3:5] {0};
    \fill[white] (axis cs:1.3,-0.8) rectangle (axis cs:4,0.8); %% Pad label of inset
    \coordinate (insetPos) at (axis cs: 2.2,0);
  \end{axis}
  \begin{axis}[at={(insetPos)}, anchor=west, footnotesize, height=3.9cm, width=4cm,
    axis background/.style={fill=white},
    xlabel=\( \gamma_0 \),
    ylabel=\( \mu / (v_F \sqrt{2e B} ) \),
    xtick={0,1,2},
    ytick={-1,0,1},
    ymin=-1,ymax=1,
    xmin=0, xmax=2,
    ]
    \draw (axis cs:1,-1) -- (axis cs:1,1);
    \addplot[domain=-1:1, dashed] ({exp(-(x / 1.1)^6) + 0.13}, x);
  \end{axis}
\end{tikzpicture}

  \caption{The Landau level of an untilted Weyl cone.
    The inset shows the prefactor \( \gamma_0 \) of the response function for a small finite potential \( \mu \), within the energy interval indicated with dashed lines.
    In the inset, the solid line is computed at zero temperature, while the dashed line is computed at a small finite temperature.
    Figure inspired by \textcite{arjonaFingerprintsConformalAnomaly2019}.
    }
    \label{fig:platau}
\end{figure}


\begin{figure}[h]
  \centering
  \tikzsetnextfilename{symmetryspin}
  \begin{tikzpicture}
    \draw[->] (-4, 0) -- (4, 0) node[right] {\(k\)};
    \draw[->] (0, 0) -- (0, 4) node[right] {\(E\)};

    % vf = 1, v0 = 0.8
    \draw[blue] (-3.5, 0.7) -- (0, 0) -- (2, 3.6) node[right] {\(\ket{\uparrow}\)} coordinate[pos=0.7] (a);
    \draw[red] (3.5, 0.7) node[right] {\(\ket{\downarrow }\)} -- (0, 0) -- (-2, 3.6)
    coordinate[pos=0.7] (b);

    \draw[->] (a) -- ++(1, 0);
    \draw[->] (b) -- ++(1, 0);
  \end{tikzpicture}
  \caption{Time reversal breaking in tilted system.
    Cross section in the tilt direction shown, with blue showing one cone and red the other.
    Black arrows indicate spin direction, which for \(\ket{\uparrow {}}\) is proporitional to  \(k\) while for \(\ket{\downarrow {}}\) is proportional to \( -k \).
  }
\end{figure}
