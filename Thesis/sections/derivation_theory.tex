\subsection{Transport and magnetization}
Recall that we generally define the transport coeffieicents
\[
J^i = e^2 L^{ij}_{11} E_j + e L_{12}^{ij} \nabla_jT,
\]
where \( J^i \) is the electrical current.
In our work, we focus on the \( L_{12} \) coefficient, however the following discussion is valid also more generally.
The definition of transport currents becomes more subtle in systems with broken time-reversal symmetry\cite{vanderwurffMagnetovorticalThermoelectricTransport2019, chernodubThermalTransportGeometry2021}.
In such systems, unobservable, circulating \emph{magnetization} currents arise.
These currents do not contribute to transport, but the Kubo treatment derives the local current, which in general also includes non-transporting currents.
Let
\begin{equation}
  \label{eq:110}
  \vec{J} = \vec{J}_{\text{tr}} + \vec{J}_M,
\end{equation}
where \( \vec{J} \) is the total local current, \( \vec{J}_{\text{tr}} \) is the transport current, and \( \vec{J}_M \) is the circulating magnetization current.
While our response \( \chi \) relates to the total current, we are more interested in the experimentally measurable transport resonse \( L^{ij}_{12} \), related to our Kubo result as \cite{arjonaFingerprintsConformalAnomaly2019}
\todo{this might not be a first-hand source. See thermal transport...geometry chernodub eq. 62}
\begin{equation}
  \label{eq:111}
  L_{12}^{ij} = \chi^{12} - \epsilon^{ijl} M_l,
\end{equation}
with \( M_l \) the magnetization.
For zero chemical potential, however, these magnetization currents have been shown to go to zero as \( T \to 0 \).
