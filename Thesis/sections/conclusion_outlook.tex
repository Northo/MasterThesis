% \section{Conclusion and outlook}
\addchap{Conclusion and outlook}
We found the response function \( \chi \) to be tunable with the tilt parameter \( t \).
The behavior Landau levels of the tilted Weyl cone is massively different depending on the direction of the tilt with respect to the magnetic field.
We found the behavior of \( \chi \) to also be massively different depending on the direction.

The results also offer potential experimental signatures.
The response of parallel tilt is odd in the tilt \( t_\parallel \) and the response of perpendicular tilt is suppressed with increasing tilt \( t_\perp \).
We believe this opens up the possibility of various experimental designs.
One example is to rotate the sample in a magnetic field, such that the tilt goes from being parallel to perpendicular.
The response of perpendicular tilt discussed in \cref{sec:other-observe}, also shown in \cref{fig:0tontx}, is a good candidate for an experimental signature as well, however it requires some more work before it is sufficiently rigorous.

\todo{
- Dramatic change around the transition from type I to type II
- Opposed to what Sharma found in their semiclassical bolztman computation of chiral anomaly, where there is no abrubt change
}
\todo{
  Discuss whether Type-II is anomalous?
  We have DoS, so energy scale, and thus maybe not conformal contribution? (Maybe compare with no abrubt change for chiral anomaly from type i to type ii) i.e. we break the scale invariance when we tilt to type ii
  only the 0th that is truly conformal??
  This is probably more result than conclusion material...
}

The calculation has also demonstrated the importance of a correct treatment of the energy-momentum tensor.
Depending on the definition used, the resulting response is qualitatively different for tilted systems, as opposed to untilted systems.

After this work, we have several unanswered questions, and believe this to be a fruitful topic for the future.
Here we propose a selection of ideas and questions that are relevant, which are direct extensions of the work in this thesis.
\begin{description}
  \item[Tilt parallel to temperature gradient]
        Due to time constraints, we were not able to extend the calculation to tilt parallel to \( \nabla T \) in time for writing the thesis.
        This is a natural extension, and we hope to be able to do this for a manuscript currently being written.

  \item[The energy-momentum tensor]
        The ambiguity related to the energy-momentum tensor, discussed in \cref{sec:commen-T} is still an open question, which should be explored more.
        Much literature has been written on the topic, both in general~\cite{forgerCurrentsEnergyMomentumTensor2004} and specific to Dirac and Weyl semimetals~\cite{vanderwurffMagnetovorticalThermoelectricTransport2019,arjonaFingerprintsConformalAnomaly2019}.
        We have had discussions on the topic with María Vozmediano and Alberto Cortijo, and have several venues that we wish to explore further on this question.

        One of our current ideas involve a fully covariant calculation, absorbing the tilt directly in the metric instead of explicitly including it in the Lagrangian.
        That will involve combining the curvature of the tilt and Luttinger's perturbation in a Kubo calculation.
        It is of interest to see if this leads to the same expressions as we found from having explicit tilt in flat space.

        - Important symmetries, gmunu non-symmetric, related to torsion and stress tensors

  \item[Finite chemical potential and temperature]
        As was discussed in \cref{sec:other-observe}, it would be interesting to extend the calculation to finite potential and temperature.
        In the untitled case, the response has a stable plateau as a function of chemical potential even for finite temperature, related to the energy gap between the \( \pm 1 \) Landau levels, where there is only the zeroth Landau level.
        As the cone is tilted, the gap is reduced, and vanishes at the transition between Type-I and Type-II.
        An explicit calculation of this is a natural next steps, and requires only minor adaptions of the work done here.

\end{description}
