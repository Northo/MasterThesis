\chapter{Contributions from symmetric energy-momentum tensor}\label{sec:otherterm-appendix}
As noted in the main text, there are some subtlety in the definition of the energy-momentum tensor.
The \emph{canonical} definition, which we have used in the main text, is in general not symmetric.
% The tensor enter our calculation from the conservation law
% \[
% \partial_\mu T^{\mu \nu} = 0,
% \]
% which for \( \nu = 0 \) is nothing more than the conservation law of energy: \( \partial_t \epsilon - \vec{\nabla} \vec{J}_\epsilon = 0 \), where \( \epsilon \) is energy density and \( \vec{J}_\epsilon \) is the energy current.
In the calculation by~\textcite{arjonaFingerprintsConformalAnomaly2019}, the symmetrized\footnote{See \cref{sec:commen-T} for a more precise discussion on the symmetrization of the energy-momentum tensor.}
energy-momentum tensor
\[
T_S^{\mu \nu} = \frac{T^{\mu \nu} + T^{\nu \mu}}{2}
\]
was used.
In this appendix we show the contributions of the symmetric tensor.
The contributions from \( T^{\mu \nu} \) and \( T^{\nu \mu} \) is shown to be equal in the non-tilted case, while they differ in the tilted case.

In the main text we have already found the contributions from the canonical tensor, and so we focus here on the contributions from \( T_F^{\mu \nu } = T^{\nu \mu} \).
The relevant element is \( T_F^{y 0} = T^{0 y} \).

The tilted canonical energy-momentum tensor, \cref{eq:T-canon-tilt},
\[
  T^{\mu\nu} =
  \frac{i}{2} (
  \phi^{\dagger} \tilde{\sigma}_s ^{\mu} \partial_{\nu} \phi
  - \partial_{\nu} \phi^{\dagger} \tilde{\sigma}_s ^{\mu} \phi
  - \eta^{\mu \nu} \mathcal{L}
  ),
\]
and so the symmetric tensor is
\renewcommand\overleftrightarrow[1]{\overset{\tiny\leftrightarrow}{#1}}
\begin{equation}
  \label{eq:87}
  T_S^{\mu\nu} =
  \frac{i}{2} (
  \phi^{\dagger} \tilde{\sigma}_s ^{\mu} \overleftrightarrow{\partial}_{\nu} \phi
  +
  \phi^{\dagger} \tilde{\sigma}_s ^{\nu} \overleftrightarrow{\partial}_{\mu} \phi
  - \eta^{\mu \nu} \mathcal{L}
  ),
\end{equation}
where we used the notation \( \phi^{\dagger} \overleftrightarrow{\partial} \phi = (\phi^{\dagger} \partial \phi - (\partial \phi^{\dagger}) \phi) /2 \).
We split \( T_S^{y0} \) into three parts;
the first part corresponds to the canonical energy-momentum tensor, while the two latter correspond to the two terms of \( T_F^{y0} \).
Explicitly
\begin{equation}
  \label{eq:88}
  T_S^{y 0} =
  \underbrace{\frac{i}{2} \phi^{\dagger} \tilde{\sigma}_s^{y} \overleftrightarrow{\partial_0} \phi}_{T^{(1)}}
  +  \underbrace{ \frac{i}{4} \phi^{\dagger} \partial_y \phi}_{T^{(2)}}
  + \underbrace{ \frac{i}{4} \phi^{\dagger} \partial_y \phi}_{T^{(3)}}.
\end{equation}
In other words, the first part is half that found in the main text, while the two latter are unique to the symmetric tensor.
For convenience, we will for the rest of the appendix rename \( T^{\mu \nu} = T_S^{\mu \nu} \).

\section{No tilt}\label{sec:otherstressterm:notilt}
Begin by considering the matrix elements
\begin{align}
  \label{eq:144}
  T ^{0y\; (2)} _{\vec{k}+\qvec{q}ns, \vec{k}ms} (\vec{q}) &= +
                                                            \frac{1}{4}
                                                            \int \mathrm{d}y
                                                            e^{iq_y y} v_F
                                                            \phi ^{*}_{\vec{k}+\qvec{q} ns}(y)
                                                            p_y \phi _{\vec{k}ms} (y),\\
  \label{eq:145}
  T ^{0y\; (3)} _{\vec{k}+\qvec{q}ns, \vec{k}ms} (\vec{q}) &= -
                                                            \frac{1}{4}
                                                            \int \mathrm{d}y
                                                            e^{iq_y y} v_F
                                                            \left( p_y\phi ^{*}_{\vec{k}+\qvec{q} ns}(y) \right)
                                                            \phi _{\vec{k}ms}(y).
\end{align}
Recall that $\phi_{\vec{k}ms} (y)$, defined in \cref{eq:72}, consists of two $y$-dependent factors:
\(
  \exp \left[-\frac{(y-k_x l_B^2)^2}{2 l_{B}^2 } \right]
\)
and the Hermite polynomials.
The operator $p_y$ thus produces two terms when operating on $\phi $.
The first term, coming from the exponent, is proportional to $y-k_xl_B^2$.
The operator in \cref{eq:144,eq:145} acts on $\phi $ with the quantum  number $\vec{k}$ and $\vec{k} + \qvec{q}$, respectively;
when summing the two contributions, everything thus cancels except for a term proportional to $q_x$, which vanishes in the local limit.

It remains to consider the result of $p_y$ operating on the Hermite polynomials.
Let $\tilde{p}_y$ indicate the $p_y$ operator acting only on the Hermite polynomial part of $\phi $, and use the property of Hermite polynomials $\partial _x H_n(x) = 2 n H_{n-1}(x)$~\cite[Eq.~18.9.25]{NIST:DLMF}.
\begin{multline}
      \phi ^{*}_{\vec{k}+\qvec{q}ns}(y) \tilde{p}_y \phi _{\vec{k}ms} =
    -i \hbar \exp \left\{
      - \frac{(y-k_xl_B^2)^2 + (y-(k_x + q_x) l_B^2)^2}{2 l_B^2}
    \right\}\\
    \frac{2}{l_B}\Bigg\{
      (M-1) a _{\vec{k} ms}a _{\vec{k}+\qvec{q} ns} H_{M-2}\left( \frac{y-k_x l_B^2}{l_B} \right) H_{N-1} \left( \frac{y - (k_x+q_x)l_B^2}{l_B} \right)\\
      + M b _{\vec{k} ms} b _{\vec{k}+\qvec{q} ns} H_{M-1} \left( \frac{y - k_xl_B^2}{l_B} \right) H_N \left( \frac{y - (k_x+q_x)l_B^2}{l_B} \right)
    \Bigg\}.
\end{multline}
Completing the square, we get
\begin{multline}
  \int \mathrm{d}y e^{iq_y y} \phi ^{*}_{\vec{k}+\qvec{q} ns}(y) \tilde{p}_y \phi _{\vec{k} ms}(y) =
  -i\hbar
  \exp \left[ -\frac{l_B^2}{4} \left\{ \qvec{q}_y^2 - 2iq_y ( 2k_x + q_x) \right\} \right]\\
  \int \mathrm{d}y
  \exp \left[
    - \left\{
      y + \frac{l_B^2}{2} \left( -iq_y - 2k_x - q_x \right)
    \right\} ^2
    / l_B^2
  \right]\\
  \frac{2}{l_B}\Bigg\{
  (M-1) a _{\vec{k}ms}a _{\vec{k}+\qvec{q} ns} H_{M-2}\left( \frac{y-k_x l_B^2}{l_B} \right) H_{N-1} \left( \frac{y - (k_x+q_x)l_B^2}{l_B} \right)\\
  + M b _{\vec{k} ms} b _{\vec{k}+\qvec{q} ns} H_{M-1} \left( \frac{y - k_xl_B^2}{l_B} \right) H_N \left( \frac{y - (k_x+q_x)l_B^2}{l_B} \right)
  \Bigg\}.
\end{multline}
Upon introducing $\tilde{y} = \frac{y}{l_B} + l_B( -iq_y - q_x - 2k_x) / 2$, as was also done in the main text, the expression reduces to
\begin{multline}
  \int \mathrm{d}y e^{iq_y y} \phi ^{*}_{\vec{k}+\qvec{q} ns}(y) \tilde{p}_y \phi _{\vec{k} ms}(y) =
  -i\hbar
  \exp \left[ -\frac{l_B^2}{4} \left\{ q_x^2 + q_y^2 - 2iq_y ( 2k_x + q_x) \right\} \right]\\
  \int \mathrm{d}\tilde{y} l_B
  \exp \left[
    - \tilde{y}^2
  \right]\\
  \frac{2}{l_B}\Bigg\{
  (M-1) a _{\vec{k}ms}a _{\vec{k}+\qvec{q} ns}
  H_{M-2}\left( \tilde{y} + \frac{l_B}{2}(iq_y + q_x) \right)
  H_{N-1} \left( \tilde{y} + \frac{l_B}{2}(iq_y - q_x) \right)\\
  + M b _{\vec{k} ms} b _{\vec{k}+\qvec{q} ns}
  H_{M-1} \left( \tilde{y} + \frac{l_B}{2}(iq_y + q_x)\right)
  H_N \left( \tilde{y} + \frac{l_B}{2}(iq_y - q_x) \right)
  \Bigg\}.
\end{multline}

Considering now the local limit \( \vec{q} \to 0 \), the expression greatly simplifies, and we may use the orthogonality relation for the Hermite polynomials \cref{eq:hermite-ortho}
\[
  \int\limits_{-\infty}^{\infty} \mathrm{d}x e^{-x^2} H_n(x)H_m(x) = \sqrt{\pi} 2^{n} n! \delta_{n,m}
\]
to evaulate the integral.
\begin{equation}
  \label{eq:146}
  \lim_{\vec{q} \to 0} \int \mathrm{d}y e^{iq_y y} \phi ^{*}_{\vec{k}+\qvec{q} ns}(y) \tilde{p}_y \phi _{\vec{k} ms}(y) =
  -i \hbar \sqrt{2}
  \frac{
    \alpha_{k m s} \alpha_{k n s} \sqrt{M-1}
    +
    \sqrt{M}
  }{
    {l_{B} \sqrt{\alpha_{k m s}^2 + 1}\sqrt{\alpha_{k n s} ^2 + 1}}
  }
  \delta_{N, M-1}.
\end{equation}

Similarly, for $T^{0y\; (3)}_{\vec{k}+\qvec{q}ns, \vec{k}ms} (\vec{q})$, one has
\begin{multline}
  \left( \tilde{p}_y \phi ^{*}_{\vec{k}+\qvec{q} ns}(y) \right)
  \phi _{\vec{k} m s}(y) =
  -i \hbar \exp \left\{
    - \frac{(y-k_xl_B^2)^2 + (y-(k_x + q_x) l_B^2)^2}{2 l_B^2}
  \right\}\\
  \frac{2}{l_{B}} \left\{
    (N-1) a_{\vec{k} m s} a_{\vec{k} + \qvec{q} n s} H_{M-1} \left( \frac{y-k_xl_B^2}{l_B} \right)H_{N-2} \left( \frac{y - (k_x + q_x) l_B^2}{l_B} \right) \right.\\
  \left. +
    N b_{\vec{k} m s} b_{\vec{k}+\qvec{q} n s} H_M \left( \frac{y - k_xl_B^2}{l_B} \right) H_{N-1}\left( \frac{y-(k_x+q_x) l_B^2}{l_B} \right)
  \right\}
\end{multline}
which with the same procedure as above gives
\begin{equation}
 \lim_{\vec{q}\to 0} \int \mathrm{d}y e^{iq_y y}
  \left( \tilde{p}_y \phi ^{*}_{\vec{k}+\qvec{q} ns}(y) \right)
  \phi _{\vec{k} m s}(y)
  =
  -i \hbar \sqrt{2}
  \frac{
    \alpha_{k m s} \alpha_{k n s} \sqrt{N-1}
    +
    \sqrt{N}
  }{
    {l_{B} \sqrt{\alpha_{k m s}^2 + 1}\sqrt{\alpha_{k n s} ^2 + 1}}
  }
  \delta_{M, N-1}.
\end{equation}

\begin{summary}\label{summary:notilt-otherterm}
  In the untilted case, we have
  \begin{align}
    \label{eq:71}
    \lim_{\vec{q} \to 0} T^{y0\; (2)}_{\vec{k} n s, \vec{k} m s} &= -\frac{i \hbar \sqrt{2}}{4}
        \frac{
          \alpha_{k m s} \alpha_{k n s} \sqrt{M-1}
          +
          \sqrt{M}
        }{
          {l_{B} \sqrt{\alpha_{k m s}^2 + 1}\sqrt{\alpha_{k n s} ^2 + 1}}
        }
                                                                   \delta_{N, M-1},\\
    \lim_{\vec{q} \to 0} T^{y0\; (3)}_{\vec{k} n s, \vec{k} m s} &= \frac{i \hbar \sqrt{2}}{4}
        \frac{
          \alpha_{k m s} \alpha_{k n s} \sqrt{N-1}
          +
          \sqrt{N}
        }{
          {l_{B} \sqrt{\alpha_{k m s}^2 + 1}\sqrt{\alpha_{k n s} ^2 + 1}}
        }
                                                                   \delta_{M, N-1}.
  \end{align}
\end{summary}

\section{With tilt}
In the tilted case, we have shown in the main text that \todo{insert ref}
\[
  T^{\mu 0} = \frac{i}{2}
  \left[\partial_i \bar{\psi} \Gamma^j \gamma^0 \Gamma^{\mu} \psi
    - \bar{\psi} \Gamma^{\mu} \gamma^0 \Gamma^j \partial_j \psi
  \right].
\]
Swapping the indices, we have for \( \mu \neq 0 \)~\cite{vanderwurffMagnetovorticalThermoelectricTransport2019}
\[
  T^{0 i} = \frac{i}{2} [\bar{\psi} \gamma^0 \partial^{\mu } \psi
  - \partial^{\mu }\bar{\psi} \gamma^{0 } \psi ].
\]
In our work, we have considered only tilt perpendiculat to the thermal gradient,  so the component of the energy-momentum tensor of interest are not affected by the tilt.

or
\begin{align}
  T_{\vec{k} + \qvec{q} ns, \vec{k} ms}^{0y (2)} (\vec{q}) &=
                                                             + \frac{1}{4} \int \mathrm{d} y
                                                             e^{iq_{y} y} v_{F}
                                                             \phi ^{*}_{\vec{k}+\qvec{q} ns} (y) p_{y} \phi _{\vec{k} m s} (y),\\
  T_{\vec{k} + \qvec{q} ns, \vec{k} ms}^{0y (3)} (\vec{q}) &=
                                                             - \frac{1}{4} \int \mathrm{d} y
                                                             e^{iq_{y} y} v_{F}
                                                             (p_{y} \phi ^{*}_{\vec{k}+\qvec{q} ns} (y))  \phi _{\vec{k} m s} (y).
\end{align}
Firstly, we note that
\[
  [p_{y} , e^{\theta /2 \sigma _{x}}] = 0.
\]
Furthermore, exactly as for the untilted case, the momentum operator acting on the exponential prefactor of \(\phi \) gives contributions proportional to \(q_{x}\).
In the local limit \(q\to  0\) this term vanishes, and we need only consider the effect of the momentum operator acting on the Hermite polynomials.

Denote by \(\tilde{p}_{y}\) the momentum operator \(p_{y}\) acting only on the Hermite polynomial part of \(\phi \).
Furthermore, we will use the property of Hermite polynomials \(\partial _{x} H_{n} (x) = 2 n H_{n-1} (x)\)~\cite[Eq.~18.9.25]{NIST:DLMF}.
\begin{align}
  \tilde{p}_{y} \phi _{\vec{k} ms} &=
  -i \hbar
  e^{\theta /2 \sigma _{x}}
  e^{-\frac{1}{2} \chi ^2}
  \partial _{y}
  \begin{pmatrix}
    a_{\vec{k} m s} H_{M-1} (\chi) \\
    b_{\vec{k} ms} H_{M} (\chi)
  \end{pmatrix}\\
                                   &=
                                     -i \hbar
                                     e^{\theta /2 \sigma _{x}}
                                     e^{-\frac{1}{2} \chi ^2}
                                     2 \frac{\partial \chi }{\partial y}
                                     \begin{pmatrix}
                                       a_{\vec{k} m s} (M-1) H_{M-2} (\chi) \\
                                       b_{\vec{k} ms} (M) H_{M-1} (\chi)
                                     \end{pmatrix}\\
                                   &=
                                     -i \hbar
                                     e^{\theta /2 \sigma _{x}}
                                     e^{-\frac{1}{2} \chi ^2}
                                     \frac{2 \sqrt{\alpha}}{ l_{B} }
                                     \begin{pmatrix}
                                       a_{\vec{k} m s} (M-1) H_{M-2} (\chi) \\
                                       b_{\vec{k} ms} (M) H_{M-1} (\chi)
                                     \end{pmatrix}.
\end{align}
And thus, recalling that
\[
  e^{\theta \sigma _{x}} =
  \begin{pmatrix}
    1 & -t_{x}\\
    -t_{x} & 1
  \end{pmatrix}
  \frac{1}{\sqrt{1-t_{x}^2}},
\]
we find the product
\begin{multline}
  \phi ^{*} _{\vec{k} + \qvec{q} ns} (y) \tilde{p}_{y} \phi _{\vec{k} m s}
  % = -\frac{i\hbar 2 \sqrt{\alpha } }{ l_{B} }
  % e^{-\frac{1}{2} \chi _{\vec{k}}^2 - \frac{1}{2} \chi _{\vec{k} + \qvec{q}} ^2}
  % \phi ^{*}_{\vec{k} + \qvec{q} ns}(y)
  % e^{\theta \sigma _{x}}
  % \phi _{\vec{k} ms} (y)\\
  % %
  = -\frac{i\hbar 2 \sqrt{\alpha } }{l_{B} \sqrt{1 - t_{x}^2} }
  e^{-\frac{1}{2} \chi _{\vec{k}}^2 - \frac{1}{2} \chi _{\vec{k} + \qvec{q}} ^2}\\
  \Big[
  a_{\vec{k} + \qvec{q} ns} H_{N-1}(\chi_{\vec{k}+\qvec{q}})
  \left\{a_{\vec{k} ms} (M-1) H_{M-2} (\chi_{\vec{k}}) - t_{x} b_{\vec{k} ms} M H_{M-1} (\chi_{\vec{k}})\right\}\\
  +
  b_{\vec{k} + \qvec{q} ns} H_{N} (\chi_{\vec{k} + \qvec{q}})
  \left\{-t_{x} a_{\vec{k} ms} (M-1) H_{M-2} (\chi_{\vec{k}}) + b_{\vec{k} ms} M H_{M-1} (\chi_{\vec{k}})\right\}
  \Big].
\end{multline}
Completing the square and substituting
\[
  \tilde{y} = \frac{\sqrt{ \alpha  }}{l_{B}}
  \left(y - \frac{l_{B}^2}{2 \alpha } (i q_{y} + (2 k'_x + q' _x) ) \right)
\]
gives
\begin{multline}
  \int \mathrm{d}y
  e^{i q_{y}}
  \phi ^{*}_{\vec{k} + \qvec{q} ns}(y) \tilde{p}_{y}
  \phi_{\vec{k} ms} (y)
  =
  \exp
  \left[
    - \frac{l_{B}^2}{4 \alpha } (q_{y}^2 - 2 i (2k' _x + q' _x) q_{y} + (q' _x)^2 )
  \right  ]\\
\times\frac{ - i\hbar 2 \sqrt{\alpha} }{l_{B} \sqrt{1 - t_{x}^2} }
  \int \mathrm{d} \tilde{y} \frac{l_{B}}{\sqrt{\alpha } }\\
\times  \Big[
  a_{\vec{k} + \qvec{q} ns} H_{N-1}( \chi _{\vec{k} + \qvec{q}} )
  \left\{
    a_{\vec{k} ms} (M-1) H_{M-2} ( \chi _{\vec{k}} )
    - t_{x} b_{\vec{k} ms} M H_{M-1} ( \chi _{\vec{k}} ) \right\}\\
  +
  b_{\vec{k} + \qvec{q} ns} H_{N} ( \chi _{\vec{k} + \qvec{q}} )
  \left\{
    -t_{x} a_{\vec{k} ms} (M-1) H_{M-2} ( \chi _{\vec{k}} )
    + b_{\vec{k} ms} M H_{M-1} ( \chi _{\vec{k}} )
  \right\}
  \Big].
\end{multline}

We must now evaluate the integral, and express the result in the \( \Xi \)-functions, defined in \cref{eq:xi1def,eq:xi2def} of the main text.
\[
    \begin{pmatrix}
      a_{\vec{k} + \qvec{q} n s} H_{N-1} ( \chi _{\vec{k} + \qvec{q}} )\\
      b_{\vec{k} + \qvec{q} ns} H_N ( \chi _{\vec{k} + \qvec{q}} )
    \end{pmatrix}^{T}
  \underbrace{
    \begin{pmatrix}
      1 & -t_x\\
      -t_x & 1
    \end{pmatrix}
    }_{T}
    \begin{pmatrix}
      a_{\vec{k} m s} (M-1) H_{M-2} (\chi_{\vec{k}})\\
      b_{\vec{k} m s} M H_{M-1} ( \chi _{\vec{k}} )
    \end{pmatrix}
\]
For each of the entries in \( T \), we get a product of Hermite polynomials.
Where the untilted cone had two such terms, the tilt parameter \( t _x \) now gives two extra products, which we must evaluate.
Let \( M^{(2)}_{ij} \) be the product corresponding to \( T_{ij} \), i.e.
\begin{align}
  M^{(2)}_{11} &= \phantom{-t_x} a_{\vec{k} + \qvec{q} ns} a_{\vec{k} ms} (M-1) H_{N-1} (\chi_{\vec{k} + \qvec{q}}) H_{M-2} (\chi_{\vec{k}}),\\
  M^{(2)}_{12} &= -t_x a_{\vec{k} + \qvec{q} ns} b_{\vec{k} ms} M H_{N-1} (\chi_{\vec{k} + \qvec{q}}) H_{M-1} (\chi_{\vec{k}}),\\
  M^{(2)}_{21} &= -t_x b_{\vec{k} + \qvec{q} ns} a_{\vec{k} ms} ( M-1 ) H_{N} (\chi_{\vec{k} + \qvec{q}}) H_{M-2} (\chi_{\vec{k}}),\\
  M^{(2)}_{22} &= \phantom{-t_x} b_{\vec{k} + \qvec{q} ns} b_{\vec{k} ms} M H_N (\chi_{\vec{k} + \qvec{q}}) H_{M-1} (\chi_{\vec{k}}).
\end{align}
We want to evaluate
\begin{equation}
  \label{eq:147}
  F^{(2)}_{ij} =
  \left[(\alpha_{k_z m s}^2 + 1) ( \alpha_{k_z + \qvec{q}_z n s}^2 + 1 )\right]^{\frac{1}{2}}
  \int \mathrm{d} \tilde{y}
  e^{-\tilde{y}^2}
  M^{(2)}_{ij},
\end{equation}
with the prefactor introduced for later convenience.

Notice that
\todo{Verify \( l_B \) in this section}
\begin{equation}
  F^{(2)}_{12}
  = -t_x \sqrt{\alpha}  \sqrt{\frac{M}{2}} \alpha_{k+q, n} \Xi_2(\bar{\vec{q}}, m\mp 1, n).
\end{equation}
and
\begin{equation}
  \label{eq:148}
  F^{(2)}_{21}
  =
  -t_{x} \sqrt{\alpha} \sqrt{\frac{M-1}{2}} \frac{a_{\vec{k} m s}^2}{l_B a_{\vec{k} m \mp 1 s}}
  \Xi_{1} (\bar{\vec{q}}, m \mp 1, n, s).
\end{equation}
\( F^{(2)}_{11} \) and \( F^{(2)}_{22} \) are the same as for the untilted case:
\begin{equation}
  \label{eq:149}
  F^{(2)}_{11} = \sqrt{\alpha}  \frac{\alpha_{k_z m s} \alpha_{k_z + \qvec{q}_z ns} \sqrt{M-1} }{ l_B \sqrt{2} }
  \Xi_{1} (\bar{\vec{q}}, m\mp 1, n\mp 1, s),
\end{equation}
and
\begin{equation}
  \label{eq:150}
  F^{(2)}_{22} =
  \sqrt{\alpha }
  \frac{\sqrt{M} }{l_B \sqrt{2} }
  \Xi_{1} ( \bar{\vec{q}}, m, n, s ).
\end{equation}
In summary we have
\begin{align}
  \label{eq:151}
  T^{0y~(2)}_{\vec{k} + \qvec{q} ns, \vec{k} ms} (\vec{q}) &= + \frac{v_F}{4} \int \mathrm{d} y
  e^{i q_y q} \phi _{\vec{k} + \qvec{q} ns}^{*} (y) p_{y} \phi _{\vec{k} ms} (y)\\
&= -\frac{ i \hbar v_{F} }{2}
                                                                                     \Gamma _{\vec{k} \qvec{q} m n s}^+
\sum\limits_{i,j}^{} F^{(2)}_{ij},
\end{align}
where
\[
  \Gamma _{\vec{k} \qvec{q} m n s}^+ =
  \frac{
  \exp
  \left[
    - \frac{l_{B}^2}{4 \alpha } (q_{y}^2 - 2 i (2k' _x + q' _x) q_{y} + (q' _x)^2 )
  \right  ]
}{
  \left[(\alpha_{k_z m s}^2 +1) (\alpha_{k_z + \qvec{q}_z ns} ^2 + 1)\right]^{\frac{1}{2}}
  \sqrt{1 - t_x^2 }}
\]

In a similar procedure, we find \( T^{0y~(2)}_{\vec{k}+\qvec{q} ns, \vec{k} ms}(\vec{q}) \).
\begin{equation}
  \tilde{p}_y \phi^{*}_{\vec{k}+\qvec{q} m s} = \frac{-i \hbar \sqrt{\alpha } }{l_{B}}  e^{-\frac{1}{2} \chi ^2}
                                   \begin{pmatrix}
                                     a_{\vec{k}+\qvec{q} m s} (M-1) H_{M-2} (\chi) \\
                                     b_{\vec{k}+\qvec{q} m s} (M) H_{M-1} (\chi)
                                   \end{pmatrix}.
\end{equation}
And thus,
\begin{multline}
  \left( \tilde{p}_y \phi ^{*} _{\vec{k} + \qvec{q} ns} (y) \right) \phi _{\vec{k} m s}
  = -\frac{i\hbar 2 \sqrt{\alpha } }{l_{B} \sqrt{1 - t_{x}^2} }
  e^{-\frac{1}{2} \chi _{\vec{k}}^2 - \frac{1}{2} \chi _{\vec{k} + \qvec{q}} ^2}\\
  \Big[
  a_{\vec{k} + \qvec{q} ns} (N-1) H_{N-2}(\chi_{\vec{k}+\qvec{q}})
  \left\{a_{\vec{k} ms} H_{M-1} (\chi_{\vec{k}}) - t_{x} b_{\vec{k} ms} H_{M} (\chi_{\vec{k}})\right\}\\
  +
  b_{\vec{k} + \qvec{q} ns} N H_{N-1} (\chi_{\vec{k} + \qvec{q}})
  \left\{-t_{x} a_{\vec{k} ms}  H_{M-1} (\chi_{\vec{k}}) + b_{\vec{k} ms} H_{M} (\chi_{\vec{k}})\right\}
  \Big].
\end{multline}
With the now well-known completion of the square and substitution, we have
\begin{multline}
  \int \mathrm{d}y
  e^{i q_{y}}
  \left[\tilde{p}_y \phi ^{*}_{\vec{k} + \qvec{q} ns}(y) \right]
  \phi_{\vec{k} ms} (y)
  =
  \exp
  \left[
    - \frac{l_{B}^2}{4 \alpha } (q_{y}^2 - 2 i (2k' _x + q' _x) q_{y} + (q' _x)^2 )
  \right  ]\\
  \times \frac{ - i\hbar 2 \sqrt{\alpha} }{l_{B} \sqrt{1 - t_{x}^2} }
  \int \mathrm{d} \tilde{y} \frac{l_{B}}{\sqrt{\alpha } }\\
  \times \Big[
  a_{\vec{k} + \qvec{q} ns} (N-1) H_{N-2}(\chi_{\vec{k}+\qvec{q}})
  \left\{a_{\vec{k} ms} H_{M-1} (\chi_{\vec{k}}) - t_{x} b_{\vec{k} ms} H_{M} (\chi_{\vec{k}})\right\}\\
  +
  b_{\vec{k} + \qvec{q} ns} N H_{N-1} (\chi_{\vec{k} + \qvec{q}})
  \left\{-t_{x} a_{\vec{k} ms}  H_{M-1} (\chi_{\vec{k}}) + b_{\vec{k} ms} H_{M} (\chi_{\vec{k}})\right\}
  \Big].
\end{multline}
Denote the terms of the integrand by
\begin{align}
  M^{(3)}_{11} &= \phantom{-t_x} a_{\vec{k} + \qvec{q} ns} a_{\vec{k} ms} (N-1) H_{N-2} (\chi_{\vec{k} + \qvec{q}}) H_{M-1} (\chi_{\vec{k}}),\\
  M^{(3)}_{12} &= -t_x a_{\vec{k} + \qvec{q} ns} b_{\vec{k} ms} (N-1) H_{N-2} (\chi_{\vec{k} + \qvec{q}}) H_{M} (\chi_{\vec{k}}),\\
  M^{(3)}_{21} &= -t_x b_{\vec{k} + \qvec{q} ns} a_{\vec{k} ms} N H_{N-1} (\chi_{\vec{k} + \qvec{q}}) H_{M-1} (\chi_{\vec{k}}),\\
  M^{(3)}_{22} &= \phantom{-t_x} b_{\vec{k} + \qvec{q} ns} b_{\vec{k} ms} N H_{N-1} (\chi_{\vec{k} + \qvec{q}}) H_{M} (\chi_{\vec{k}}).
\end{align}
We must evaluate
\begin{equation}
  \label{eq:152}
  F^{(3)}_{ij} = \left[(\alpha_{k_z m s}^2  + 1) ( \alpha_{k_z + \qvec{q}_z ns}^2 + 1 )\right]^{\frac{1}{2}} \int \mathrm{d} \tilde{y} e^{-\tilde{y}^2} M^{(3)}_{ij}.
\end{equation}
From the untilted case we know
\begin{align}
  F^{(3)}_{11} &= \sqrt{\frac{N-1}{2}}
                 \frac{\alpha_{k_z m s} \alpha_{k_z + \qvec{q}_z n s}}{l_{B} \alpha_{k_z + \qvec{q}_z n \mp 1 s}}
                 \Xi_2( \bar{\vec{q}}, m\mp 1, n\mp 1, s ),\\
  F^{(3)}_{22} &= \sqrt{\frac{N}{2}}
                 \frac{1}{l_{B} \alpha_{k_z + \qvec{q}_z n s}}
                 \Xi_{2} (\bar{\vec{q}}, m, n, s).
\end{align}
Furthermore,
\begin{align}
  F^{(3)}_{12} &= -t_x \frac{\alpha_{k_z + \qvec{q}_z n}}{\alpha_{k_z + \qvec{q}_z n \mp 1} l_{B}}
                 \sqrt{\frac{N-1}{2}}
                 \Xi_{2} (\bar{\vec{q}}, m, n \mp 1, s),\\
  F^{(3)}_{21} &= -\frac{t_x}{l_{B}}
                 \sqrt{\frac{N}{2}}
                 \frac{\alpha_{k_z m}}{\alpha_{k_z + \qvec{q}_z n}}
                 \Xi_2 ( \bar{\vec{q}}, m \mp 1, n, s ).
\end{align}
We thus have
\begin{align}
  \label{eq:153}
  T^{0y~(3)}_{\vec{k} + \qvec{q} ns, \vec{k} ms} (\vec{q})
  &= - \frac{v_F}{4}
    \int \mathrm{d} y
    e^{i q_y y}
    \left( p_y \phi ^{*}_{\vec{k} + \qvec{q} ns}(y) \right) \phi _{\vec{k} m s}(y)\\
  &= \frac{i \hbar v_F}{2}
    \Gamma ^{+}_{\vec{k} \qvec{q} m n s}
    \sum\limits_{ij}^{} F^{(3)}_{ij}.
\end{align}

\begin{summary}
The non-canonical part of the energy-momentum tensor \( T_F^{\mu \nu} = T^{\nu \mu} \) in a tilted system have the matrix elements
  \begin{align}
    T^{0y~(2)}_{\vec{k} + \qvec{q} ns, \vec{k} ms} (\vec{q})
    &= -\frac{ i \hbar v_{F} }{2}
      \Gamma _{\vec{k} \qvec{q} m n s}^+
      \sum\limits_{i,j}^{} F^{(2)}_{ij},\\
    T^{0y~(3)}_{\vec{k} + \qvec{q} ns, \vec{k} ms} (\vec{q})
    &= \frac{i \hbar v_F}{2}
      \Gamma ^{+}_{\vec{k} \qvec{q} m n s}
      \sum\limits_{ij}^{} F^{(3)}_{ij}.
  \end{align}
  with
  \[
    \Gamma _{\vec{k} \qvec{q} m n s}^{+} =
    \frac{
      \exp
      \left[
        - \frac{l_{B}^2}{4 \alpha } (q_{y}^2 + (q' _x)^2) +  i q_y l_B^2 (k' _x + \frac{q'_x}{2} )  )
      \right  ]
    }{
      \left[(\alpha_{k_z m s}^2 +1) (\alpha_{k_z + \qvec{q}_z ns} ^2 + 1)\right]^{\frac{1}{2}}
    }
  \]
  and where the factors \( F_{ij}^{(n)} \) where found to be
  \begin{align}
    F^{(2)}_{12}
    &= -t_x \sqrt{\alpha}  \sqrt{\frac{M}{2}} \alpha_{k+q, n} \Xi_2(\bar{\vec{q}}, m\mp 1, n),\\
    F^{(2)}_{21}
    &=
      -t_{x} \sqrt{\alpha} \sqrt{\frac{M-1}{2}} \frac{a_{\vec{k} m s}^2}{l_B a_{\vec{k} m \mp 1 s}}
      \Xi_{1} (\bar{\vec{q}}, m \mp 1, n, s),\\
    F^{(2)}_{11} &= \sqrt{\alpha}  \frac{\alpha_{k_z m s} \alpha_{k_z + \qvec{q}_z ns} \sqrt{M-1} }{ l_B \sqrt{2} }
                   \Xi_{1} (\bar{\vec{q}}, m\mp 1, n\mp 1, s),\\
    F^{(2)}_{22} &=
                   \sqrt{\alpha }
                   \frac{\sqrt{M} }{l_B \sqrt{2} }
                   \Xi_{1} ( \bar{\vec{q}}, m, n, s ),\\
    F^{(3)}_{11} &= \sqrt{\frac{N-1}{2}}
                   \frac{\alpha_{k_z m s} \alpha_{k_z + \qvec{q}_z n s}}{l_{B} \alpha_{k_z + \qvec{q}_z n \mp 1 s}}
                   \Xi_2( \bar{\vec{q}}, m\mp 1, n\mp 1, s ),\\
    F^{(3)}_{22} &= \sqrt{\frac{N}{2}}
                   \frac{1}{l_{B} \alpha_{k_z + \qvec{q}_z n s}}
                   \Xi_{2} (\bar{\vec{q}}, m, n, s),\\
    F^{(3)}_{12} &= -t_x \frac{\alpha_{k_z + \qvec{q}_z n}}{\alpha_{k_z + \qvec{q}_z n \mp 1} l_{B}}
                   \sqrt{\frac{N-1}{2}}
                   \Xi_{2} (\bar{\vec{q}}, m, n \mp 1, s),\\
    F^{(3)}_{21} &= -\frac{t_x}{l_{B}}
                   \sqrt{\frac{N}{2}}
                 \frac{\alpha_{k_z m}}{\alpha_{k_z + \qvec{q}_z n}}
                 \Xi_2 ( \bar{\vec{q}}, m \mp 1, n, s ).
\end{align}
\end{summary}

\subsection{Parallel tilt}
The procedure greatly simplifies in the case of parallel tilt.
As noted in the main text, parallel tilt only rescales the energies Landau levels, while the wave functions and operators stay invariant.
The procedure for the untilted cone, done in \cref{sec:otherstressterm:notilt}, is thus relevant here as well, with an interchange of the energy levels where relevant.

The \( T^{(2)} \) and \( T^{(3)} \) parts of the energy-momentum tensor for parallel tilt is therefore the same as the result without tilt, found in summary~\ref{summary:notilt-otherterm}.
In the main text we showed a simplification procedure for terms of the form
\begin{equation}
 \label{eq:86}
  \alpha_{\kappa_z m s}^2 \delta_{M-1, N} - \alpha_{\kappa_z n s}^2 \delta_{N-1,M}
\end{equation}
in the total response function.
The outline of the idea was to note that we sum over all \( m,n \), and by certain symmetries of the terms under interchange of \( m\leftrightarrow n \), we could rename summation indices and replace
\begin{equation}
  \alpha_{\kappa_z m s}^2 \delta_{M-1, N} - \alpha_{\kappa_z n s}^2 \delta_{N-1,M}
  \to
  2 \alpha_{\kappa_z m s}^2 \delta_{M-1, N}.
\end{equation}
For details on the procedure see \cref{sec:response_notilt} of the main text.
By simply inserting \( T^{(2)}, T^{(3)} \) in the response function, one may easily show that the resulting term is on the form \cref{eq:86}, with the first term corresponding to \( T^{(3)} \) and the second to \( T^{(2)} \).
The response from \( T^{(2)} \) and \( T^{(3)} \) is thus equal.

By the procedure explained in \cref{sec:commen-T}, the response of \( T^{(2)} + T^{(3)} \) may be rewritten as the response of \( T^{(1)} \), which contains the factor \( E_{k_z m s} + E_{k_z n s} \), with the energies replaced with the untilted energies.
In other words, using the energy momentum tensor \( T^{\mu \nu}_F \), the response is the same as the response found for parallel tilt in the main text, \cref{eq:120},
\begin{multline*}
  \lim_{\omega \to 0} \lim_{\vec{q} \to 0} \chi^{xy} =
  - \frac{e^2 v_F B}{2 (2\pi)^2}
  \sum\limits_{mn} \int \mathrm{d} \kappa_z
  \xi(\kappa_z)
  (\epsilon_{\kappa_z m s} + \epsilon_{\kappa_z n s})\\
  \times(\alpha_{\kappa_z m s}^2 \delta_{M-1, N} - \alpha_{\kappa_z n s}^2 \delta_{N-1, M}),
\end{multline*}
with the term \( \epsilon_{\kappa_z m s} + \epsilon_{\kappa_z n s} \) replaced with the untilted energies \( \epsilon^0_{\kappa_z m s} + \epsilon^0_{\kappa_z m s}\).
The response from the \( T^{\mu \nu}_F \) tensor is therefore the exact same as that of the untilted cone, as long as one stays in Type-I.
It differs from the response found in the main text by the divergent prefactor \( \gamma_{\text{div}, N} \).
