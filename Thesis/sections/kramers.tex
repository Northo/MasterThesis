\section{Kramer's degeneracy}

%% Some of the material is inspired from
%% Sakurai and Napolitano

Kramer's degeneracy states that for any half-integer system that is time-reversal symmetric, energy levels are at least two-fold degenerate.
The proof of this is simple, and uses the fact that for any half-integer spin system, $\mathcal{T}^2 = -1$.
A heuristic way to see this is the fact that spin is odd under time-reversal, and for half-integer systems there is no zero-spin state, so reversing the spin cannot result in the same state.
\begin{Proof}
  Assume
  $$ [H, \mathcal{T}] = 0 $$
  and that $\ket{n}$ is an eigenstate of the system
  $$ H \ket{n} = E_n \ket{n}.$$\\
  Then
  $$
  H \mathcal{T} \ket{n} = \mathcal{T} H \ket{n} = \mathcal{T} E_n \ket{n} = E_n \mathcal{T} \ket{n}
  $$
  and so $\mathcal{T} \ket{n}$ is also an eigenstate with the eigenvalue $E_n$.
  To assert that the eigenvalue is in fact degenerate, one must also show that the two states are not the same ray.
  That is $\mathcal{T} \ket{n} \neq e^{i \delta} \ket{n}$, where $\delta$ is some phase.
  Suppose that the above is \emph{not} true, $\mathcal{T} \ket{n} = e^{i \delta} \ket{n}$.
  Then,
  $$
  \mathcal{T}^2 \ket{n} = \mathcal{T} e^{i\delta} \ket{n} = e^{-i\delta} \mathcal{T} \ket{n} = + \ket{n}.
  $$
  However, as was stated above, $\mathcal{T}^2 = -1$ for all half-integer systems.
  The assumption must therefore be wrong, and the eigenvalue is degenerate.
\end{Proof}
The two states, $\ket{n}$ and $\mathcal{T} \ket{n}$, are often referred to as Kramer's doublet.
Note that the two states have opposite spin.

\subsection{Generalization to time and parity symmetry}
Consider now a time-reversal and parity symmetric system, $[H, P \mathcal{T}] = 0$.
This will, similarly to the case for time-reversal, make the energy levels at least two-fold degenerate.
\begin{Proof}
  Assume
  $$
  [H,P \mathcal{T}] = 0
  $$
  and that $\ket{n}$ is an eigenstate of the system
  $$
  H\ket{n} = E_n\ket{n}.
  $$
  Then
  $$
  H P \mathcal{T}\ket{n} =
  P\mathcal{T} H \ket{n} =
  P \mathcal{T} E_n \ket{n} =
  E_n P\mathcal{T} \ket{n}.
  $$

  Assume now that $P\mathcal{T} \ket{n} = e^{i\delta} \ket{n}$, which we will prove to be false.
  That would lead to
  $$
  (P\mathcal{T})^2 \ket{n} = P\mathcal{T} e^{i\delta} \ket{n}
  = \ket{n}.
  $$
  However, as $[P, \mathcal{T}] = 0$, we have
  $$
  (P\mathcal{T})^2 =
  P\mathcal{T} P \mathcal{T}=
  P\mathcal{T}^2 P=
  -1
  $$
  as $P^2 = 1$.
  As above, the states are thus different, and the eigenvalue is degenerate.
\end{Proof}
