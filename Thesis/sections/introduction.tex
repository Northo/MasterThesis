\addchap{Introduction}
% What has been done
% What will we do
% Why is it nice?
%
% Topological materials
% Analogous to high energy
% Chiral and gravitational-chiral anomaly
% Conformal anomaly
Topological materials have been of central interest in contemporary condensed matter physics~\cite{fruchartIntroductionTopologicalInsulators2013}, with the first topological phases arising in the context of integer quantum Hall effect~\cites{klitzingNewMethodHighAccuracy1980}[as cited in][]{fruchartIntroductionTopologicalInsulators2013}.
A solid understanding of the topological theory behind this has been developed during the last decade and a half~\cite{fruchartIntroductionTopologicalInsulators2013, bernevigTopologicalInsulatorsTopological2013}, with the Nobel Prize in Physics 2016 awarded for theoretical work on topological matter~\cite{royalswedishacademyofsciencesNobelPrizePhysics2016}.
% An excellent review of topological materials is given in~\cite{fruchartIntroductionTopologicalInsulators2013}, and more directly relevant for this thesis is~\cite{armitageWeylDiracSemimetals2018}.
Two excellent reviews of topological materials are~\cite{fruchartIntroductionTopologicalInsulators2013} and, most directly relevant for this thesis,~\cite{armitageWeylDiracSemimetals2018}.

One interesting phenomenon in topological materials is the emergence of quantum anomalies and the emergent particles' analogy to fundamental particles of \gls{qft}.
% One interesting phenomenon in topological materials is the emergence of quantum anomalies; the emergent particles follow in wonderful mathematical analogy to fundamental particles of quantum field theory.
% One interesting phenomenon in topological materials is the emergence of effective particles with strong analogy to the fundamental particles of \gls{qft}, even housing quantum anomaly effects.
Noether's theorem says that for any continuous symmetry of the action of a system, there is an accompanying conserved current.
This explains, for example, the conservation of momentum and energy as a result of the position and time independence of our universe.
In a quantum mechanical treatment, however, the symmetry of the classical theory may be broken, which gives rise to an \emph{anomaly}.
The chiral anomaly, for example, has been of great interest in condensed matter research in recent years~\cite{arjonaFingerprintsConformalAnomaly2019}.
The chiral anomaly explains the non-conservation of the axial current~\cite{zeeQuantumFieldTheory2010}, and gives rise to exotic transport phenomena in condensed matter systems~\cite{burkovChiralAnomalyTransport2015, wehlingDiracMaterials2014, burkovTopologicalSemimetals2016}.
A less investigated anomaly is the conformal anomaly, the appearance of a non-vanishing trace of the energy-momentum tensor in a conformally scaled metric.
Transport from the conformal anomaly has recently been investigated and shown in Weyl and Dirac semimetals~\cite{chernodubAnomalousTransportDue2016, chernodubGenerationNernstCurrent2018, arjonaFingerprintsConformalAnomaly2019,arjonaromanoNovelThermoelectricElastic2019}.

\textcite{arjonaFingerprintsConformalAnomaly2019} derived, using the Kubo formalism, the charge current response to thermal perturbations in Weyl and Dirac semimetals.
This response, a contribution to the Nernst current, has its origin in the conformal anomaly.
In this thesis, we extend the calculation \emph{tilted} Dirac and Weyl semimetals -- Lorentz breaking extensions of the Dirac equation.
% In this thesis, we extend the Kubo calculation done by~\textcite{arjonaFingerprintsConformalAnomaly2019} to find the charge current response to thermal perturbations in \emph{tilted} Dirac and Weyl semimetals -- Lorentz breaking extensions of the Dirac equation.
The work combines important theory and concepts from both high-energy and condensed matter physics.

The thesis consists of four chapters;
the three first chapters introduce concepts central to the main derivation of the thesis, while the fourth chapter represents the bulk work of the thesis, namely the derivation itself.
Note that some sections of the first and second chapter, and most of the third chapter, started as parts of a specialization project report written in the fall of 2021.

The first chapter gives an overview of concepts important to topological materials, starting with symmetries and ending with a more in-depth discussion of Weyl and Dirac semimetals, with a special focus on the tilted type.
In the second chapter, linear responses theory is introduced in light of the Kubo formalism and the Luttinger formalism of thermal transport.
In chapter three, we introduce anomalies of \gls{qft} in the context of high-energy physics.
The thesis also contains three appendices;
\cref{cha:long_expression} contains long expressions not included in the main text, \cref{sec:otherterm-appendix} contains a lengthy calculation that runs somewhat in parallel to the main text, for an alternative choice of the energy-momentum tensor (details discussed in the main text), and lastly, \cref{sec:aux-appendix} contains several minor results the author finds interesting, that are only tangentially relevant to the main work.


% The thesis consists of four chapters;
% the first three chapters introduce central concepts to the derivation, namely topological materials, with a focus on Weyl and Dirac semimetals; linear response theory; and anomalies in quantum field theory.
% The fourth chapter is the derivation of the response function, combining all the previous concepts.
% Note that some sections of the first and second chapter, and most of the third chapter, were initially written for a specialization project report.
