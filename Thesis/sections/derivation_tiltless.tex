\section{Analytical expressions for the operators}
We will here find analytical expressions for the current operator $J^i(\omega, \vec{q})$ and stress-energy tensor $T^{0j}(\omega, \vec{q})$, needed to calculate the correlation function.
The fields are given, in the position basis, by
\begin{align}
  \psi &= \sum\limits_{\vec{k}n}^{}\braket{\vec{r} | \vec{k} n s} a_{\vec{k}ns}(t) = \sum\limits_{\vec{k}n}^{} \phi_{\vec{k} n s} (\vec{r}) a_{\vec{k}n s}(t),\\
  \psi^{\dagger} &= \sum\limits_{\vec{k}n}^{}
                   \braket{\vec{k} ns | \vec{r} }
                   a^{\dagger}_{\vec{k}ns}(t)
                   =\sum\limits_{\vec{k}n}^{} \phi^{*}_{\vec{k} n s} (\vec{r}) a^{\dagger}_{\vec{k}n s}(t).
\end{align}
Here $a_{\lambda }^{\dagger} (t) = \exp(iE_{\lambda } t / \hbar) a_{\lambda }^{\dagger}$ and $a_{\lambda }^{\dagger}, a_{\lambda }$ are the creation and annihilation operators of the state with quantum numbers $\lambda $.
The current operator $\hat{\vec{J}} = e \hat{\vec{v}}$, where $\hat{\vec{v}}$ is the velocity operator.
Using the relation of Heisenberg operators $\dot{A} = [A, H] / i\hbar $~\cite{sakuraiModernQuantumMechanics2017}, for the operator $A$ and Hamiltonian $H$, the operator
\begin{align}
  \vec{v} = \dot{\vec{r}} &= \frac{1}{i \hbar } \left[ \vec{\vec{r}}, H \right]\\
              &= \frac{sv_F \sigma ^i}{i \hbar } \left[ \vec{r}, p_i + e A_i \right]\\
              &= \frac{s v_F \sigma^i }{i \hbar } \left( i\hbar + e[\vec{r}, A_i] \right)\\
              &=s v_F \sigma ^i,
\end{align}
and thus
\begin{equation}
  J^x = \psi ^{\dagger} \hat{J}^x \psi = sv_F e \sum\limits_{\vec{k}m, \vec{l}n}^{}
  \phi _{\vec{k}ms}^{*}(\vec{r}) \sigma ^x \phi _{\vec{l}ns}(\vec{r})
  a_{\vec{k}ms}^{\dagger}(t)
  a_{\vec{l}ns}(t).
\end{equation}

\subsection{The energy momentum tensor}
The \emph{canonical} energy-momentum tensor is generally defined by
\begin{equation}
  \label{eq:66}
  T^{\mu \nu} =  \frac{\delta \mathcal{L}}{\delta(\partial_{\mu} \phi_{i})} \partial_{\nu} \phi_i - \eta^{\mu \nu} \mathcal{L},
\end{equation}
where the index \( i \) runs over the types of fields.
This definition is correct for commuting fields, however, for non-commuting fields like ours, this formula is slightly wrong.
This is often overlooked in many textbooks and papers, so we will here elucidate the issue to some degree.
While a proper derivation requires the use of Grassman variables and defining left and right derivation, which we will not do here, some simple considerations help in understanding the issue.
In the standard text book derivation of then canonical energy-momentum tensor, one expands the total derivative of the Lagrangian \( \mathcal{L}(\psi_i, \partial \psi_i)\) in terms of the fields
\begin{equation}
  \label{eq:67}
  \frac{\mathrm{d} \mathcal{L}(\psi_{i}, \partial \psi_i)}{\mathrm{d} x_{\nu }} \equiv \mathrm{d}^{\nu } \mathcal{L}
  = \frac{\partial \mathcal{L}}{\partial (\partial_{\mu } \phi_i)} \frac{\partial(\partial _{\mu } \psi_i)}{\partial x_{\nu }}
  + \frac{\partial \mathcal{L}}{\partial \psi _{i}} \frac{\partial \psi_i}{\partial x_{\nu }}.
\end{equation}
This expansion, however, ignores the non-commutative nature of the fields.
For concreteness, consider \( \psi _i = \bar{\psi} \).
Heuristically, the correct expression would be obtained by reordering the factors in the two terms.
By naively employing Eq. \eqref{eq:66}, the resulting canonical energy-momentum tensor of the Dirac theory would be
\begin{equation}
  T^{\mu \nu} = \frac{\delta \mathcal{L}}{\delta (\partial_{\mu} \bar{\psi})} \partial^{\nu} \bar{\psi}  + \frac{\delta \mathcal{L}}{\delta (\partial_{\mu} \psi)} \partial^{\nu} \psi - \eta^{\mu \nu} \mathcal{L},
\end{equation}
while the correct form is \cite[Eq.~3-153]{itzyksonQuantumFieldTheory1980}
\begin{equation}
  \label{eq:68}
  T^{\mu \nu} = \partial^{\nu} \bar{\psi} \frac{\delta \mathcal{L}}{\delta (\partial_{\mu} \bar{\psi})} + \frac{\delta \mathcal{L}}{\delta (\partial_{\mu} \psi)} \partial^{\nu} \psi - \eta^{\mu \nu} \mathcal{L}.
\end{equation}

Our Hamiltonian
\[
H_s = s \sigma^i k_i
\]
may of course be considered as a Weyl decomposition of a full massless Dirac equation.

\todo{Why do we have to consider 4x4? Is the definitions not also valid for 2x2?}
\todo{Regarding the non-symmetry of the stress tensor, see keichelriess eq 5.16 with discussion}

The Hamiltonian
\[
H_s = s \sigma^i k_i
\]
can be considered the Hamiltonian of one part of a Weyl decomposition of a Dirac system.
The Weyl field has the Lagrangian density \cite{kachelriessQuantumFieldsHubble2018}
\begin{equation}
  \label{eq:69}
  \mathcal{L} = i \phi^{\dagger} \sigma^{\mu} \partial_{\mu} \phi,
\end{equation}
which may be seen directly from the Dirac Lagrangian \( i \bar{\psi} \slashed{\partial } \psi  \) by taking \( \psi = (\phi_L, \phi_R)^T \) and set, for example, \( \phi _R = 0 \).
Symmetrizing in daggered and undaggered fields
\footnote{The Lagrangian itself is unphysical, and we may transform it in any way that leaves the action \( \int \mathcal{L} \) invariant.}
\todo{Alternatively argue by directly showning that this does not affect the action by doing an integration by parts}
\[
  \mathcal{L} = \frac{i}{2} \left(\phi^{\dagger} \sigma^{\mu} \partial_{\mu} \phi - \partial_{\mu} \phi^{\dagger} \sigma^{\mu} \phi \right),
\]
which will prove to be more convenient to work with.
Adapting the definition Eq. \eqref{eq:68} the energy-momentum tensor for the untilted Dirac cone is thus
\begin{equation}
  \label{eq:70}
  T^{\mu \nu} =
  \frac{i}{2} (
  \phi^{\dagger} \sigma^{\mu} \partial_{\nu} \phi
  - \sigma^{\mu} \phi \partial_{\nu} \phi^{\dagger}
  - \eta^{\mu \nu} \mathcal{L}
  ).
\end{equation}

Moving now to the tilted case, the 4x4 Lagrangian becomes \cite{vanderwurffMagnetovorticalThermoelectricTransport2019}
\todo{check sign compared to action in stoof}
\begin{equation}
  \label{eq:71}
  \mathcal{L}_{\text{tilt}} = i \bar{\psi} \Gamma ^{\mu }\partial_{\mu } \psi ,
\end{equation}
where we have introduced modified gamma matrices
\begin{equation}
  \label{eq:72}
  \Gamma ^{\mu } =
  \begin{cases}
    \gamma ^{\mu } + t^{\mu} \gamma ^0& \text{ inversion symmetry broken },\\
    \gamma^{\mu} + t^{\mu} \gamma^0 \gamma^5 & \text{ inversion symmetric },
  \end{cases}
\end{equation}
with \( t^{\mu } = (0, \vec{t}) \).
\begin{equation}
  \label{eq:73}
  T^{\mu \nu} =
  \frac{i}{2} (
  \phi^{\dagger} \tilde{\sigma} ^{\mu} \partial_{\nu} \phi
  - \tilde{\sigma} ^{\mu} \phi \partial_{\nu} \phi^{\dagger}
  - \eta^{\mu \nu} \mathcal{L}
  ),
\end{equation}
where we defined the modified Pauli matrices
\begin{equation}
  \label{eq:74}
  \tilde{\sigma}^{\mu} =
  \begin{cases}
    \sigma^{\mu} + t^{\mu} & \text{ inversion symmetry broken },\\
    \sigma^{\mu} + s t^{\mu} & \text{ inversion symmetric }.
  \end{cases}
\end{equation}
Similarly, the $T^{0y}$ component of the stress-energy tensor of the theory is given by~\cite{arjonaFingerprintsConformalAnomaly2019}
\begin{equation}
  \begin{split}
    T^{0y}(t, \vec{r}) &=
    \sum\limits_{\vec{k} m, \vec{l} n}^{}
    \frac{1}{4}
    \bigg\{
    \left[
      v_F \phi ^{*}_{\vec{k} m s}(\vec{r}) p_y \phi _{\vec{l} n s}(\vec{r})
      -v_F \left( p_y \phi ^{*}_{\vec{k} m s} \right) \phi _{\vec{l} ns}
    \right] a^{\dagger}_{\vec{k} m s}(t) a_{\vec{l} n s}(t)\\
    &+ \phi ^{*}_{\vec{k} m s}(\vec{r}) s \sigma ^y \phi _{\vec{l} n s }(\vec{r})
    \left[
      a^{\dagger}_{\vec{k} m s}(t) i\hbar \partial _0  a_{\vec{l} n s}(t)
      -
      i\hbar \left(\partial _0 a^{\dagger}_{\vec{k} ms }(t) \right) a_{\vec{l} n s}(t)
    \right]\\
    &+ \phi ^{*}_{\vec{k} m s}(\vec{r}) s \sigma ^y (2\mu ) \phi _{\vec{l} n s}(\vec{r}) a^{\dagger}_{\vec{k} m s}(t) a_{\vec{l} n s}(t)
    \bigg\}.
  \end{split}
\end{equation}
Here, also a non-zero potential $\mu $ is included.
Our final result will be given at zero potential, however it is included in the calculations as it might be of interest to consider finite potential in later work.
Recalling the time dependence of $a(t), a^{\dagger}(t)$ we have that
\[
  i\hbar \partial _0 a_{\lambda }(t) = E_{\lambda }a_{\lambda },
  \quad
  i\hbar \partial _0 a^{\dagger}_{\lambda }(t) = -E_{\lambda }a^{\dagger}_{\lambda },
\]
which further simplifies the expression.

\begin{comment}
  The stress-energy tensor of the massless QED
  \begin{equation}
    \label{eq:42}
    \mathcal{L} = -\frac{1}{4} F^{\mu \nu }F_{\mu \nu } + \overline{\psi} i \slashed{D} \psi
  \end{equation}
  is given by~\cite{chernodubGenerationNernstCurrent2018}
  \begin{equation}
    T^{\mu \nu } = -F^{\mu \nu } F_{\mu \nu } + \frac{1}{4} \eta ^{\mu \nu } F_{\alpha \beta } F^{\alpha \beta } + \frac{i}{2} \overline{\psi}
    \left( \gamma ^{\mu } D^{\nu } + \gamma ^{\nu } D^{\mu } \right) \psi
    - \eta ^{\mu \nu } \overline{\psi} i \slashed{D} \psi .
  \end{equation}
  Specializing to the Weyl Hamiltonian we may drop the terms originating with the $F$ field self energy, and also we will consider only one Weyl spinor part of the Dirac four spinor.
  Thus, the stress-energy tensor is given by
  \begin{equation}
    T^{\mu \nu } = \frac{i}{2} \psi ^{\dagger} \left( \sigma ^{\mu } D^{\nu } + \sigma ^{\nu } D^{\mu } \right) \psi  - \eta ^{\mu \nu } \psi ^{\dagger} i \sigma ^{\mu } D_{\mu } \psi ,
    \label{eq:43}
  \end{equation}
  where $\psi $ is to be understood as the solutions found above, $D_{\mu }=\partial _{\mu }  - i e A_{\mu }$ is the covariant derivative, and $\sigma ^{\mu } = (I, \sigma ^i)$
  In our calculations we will require the $T^{0y}$ component, which we will now find.

  By using Eq. (\ref{eq:43}) directly
  \begin{align}
    T^{0y} &= \frac{i}{2} \psi ^{\dagger} \left( D^{y} + \sigma ^{y} D^{0} \right)\psi \\
           &= \frac{i}{2} \psi ^{\dagger} \left( \partial ^{y} + \sigma ^{y} \partial ^{0} \right)\psi \\
           &= \frac{i}{2} \left[
             \phi ^{*} \sigma ^y \phi  a^{\dagger} \partial ^0 a + \phi ^{*} \left( \partial ^y\phi  \right) a^{\dagger}a
             \right].
  \end{align}
  The stress-energy tensor is obviously real, thus
  \[
    T^{0y} = \frac{1}{2} \left( T^{0y} + \left(  T^{0y}\right)^{\dagger} \right),
  \]
  which after evaluation gives
  \begin{equation}
    T^{0y} =  \frac{i}{4} \left[
      \phi ^{*} \sigma ^y \phi \left( a^{\dagger} \partial ^0 a - (\partial ^0a)^{\dagger} a \right)
      +
      \left(
        \phi ^{*} \partial ^y \phi  - \left( \partial ^y \phi  \right)^{\dagger} \phi 
      \right)
      a^{\dagger}a
    \right].
  \end{equation}
  Now, recovering our dimensionfull quantities by letting $i\partial_{\mu }  \to v_F p_{\mu }$ \todo{what happens to s?}, which we see from comparing the QED Lagrangian in Eq. (\ref{eq:42}) to our system.
  The four momentum is given as usual, with $c \to v_F$, by $p_{\mu } = \left( \frac{E}{v_{F}}, -\vec{p} \right) =  (i\hbar\frac{\partial_0}{v_{F}}, -p _i)$.
  This gives the final expression 
  \begin{equation}
    T^{0y} =  \frac{1}{4} \left[
      \phi ^{*} \sigma ^y \phi \left( a^{\dagger} i\hbar \partial ^0 a - (i\hbar \partial ^0a)^{\dagger} a \right)
      - v_{F}
      \left(
        \phi ^{*} p^y \phi  - \left( p^y \phi  \right)^{\dagger} \phi 
      \right)
      a^{\dagger}a
    \right].
  \end{equation}
\end{comment}
\todo{If time, derive T}

Fourier transforming the position gives
\begin{align}
  \label{eq:44}
  J^x(t, \vec{q}) &= \sum\limits_{\vec{k}m, \vec{l}n}
                    J^x_{\vec{k}ms, \vec{l}ns}(\vec{q})
                    a^{\dagger}_{\vec{k}ms}(t)
                    a_{\vec{l} ns}(t),\\
  \label{eq:45}
  T^{0y}(t, -\vec{q}) &= \sum\limits_{\vec{k}m, \vec{l}n}^{}
                    T^{0y}_{\vec{k}m s, \vec{l}n s}(\vec{q})
                    a^{\dagger}_{\vec{k}m s}(t)
                    a_{\vec{l} n s}(t),
\end{align}
where the matrix elements in momentum space are given by
\begin{align}
  J^x_{\vec{k}ms, \vec{l}ns}(\vec{q}) &=  \int \mathrm{d} \vec{r} e^{-i \vec{q} \vec{r}} s v_F e \phi ^{*}_{\vec{k}ms} (\vec{r}) \sigma ^x \phi _{\vec{l}ns}(\vec{r}),\\
  %%
  \label{eq:46}
  T^{0y}_{\vec{k}m s, \vec{l} n s}(\vec{q}) &= \frac{1}{4} \int \mathrm{d}\vec{r} e^{i\vec{q}\vec{r}} \left[
                                                  v_F \phi ^{*}_{\vec{k} m s}(\vec{r})  p_y \phi _{\vec{l} ns} (\vec{r})
                                                  - v_F (p_y \phi ^{*}_{\vec{k}m s}) \phi _{\vec{l} ns }(\vec{r})
                                                  \right]\\
                                             \nonumber &+ \frac{1}{4}
                                                \int \mathrm{d}\vec{r} e^{i\vec{q}\vec{r}}
                                                \phi ^{*}_{\vec{k}m s}(\vec{r}) s \sigma ^y
                                                (E_{\vec{k}_z m s} + E_{\vec{l}_z n  s} - 2 \mu ) \phi _{\vec{l} n  s}(\vec{r}).
\end{align}
Note that as $T^{0y}(t, -\vec{q})$ will be used later, we here for convenience included the sign into the definition of the matrix element  $T^{0y}_{\vec{k}ms, \vec{l}ns}$, as is reflected in the sign of the exponent of Eq. (\ref{eq:46}).

As was noted earlier, the eigenvectors are plane waves in the $x, z$-directions, and the non-trivial part is the $y$-dependent $\phi (y)$.
Thus, we want to express these matrix elements in terms of $\phi (y)$.
The sum over $\vec{l}$ in Eq. (\ref{eq:44}) can be replaced by an integration, as it is a good quantum number.
As usual, the measure in the integration is given by the density of states in momentum space, the well known $L_{i} /2\pi $, with $L_i$ being the length of the system in the $i$-direction.
\begin{align}
  J^x(t, \vec{q}) &= \sum\limits_{\vec{k}m, n}^{} \int \mathrm{d}l_x \mathrm{d}l_z \frac{L_xL_z}{4 \pi ^2}
                    J^x_{\vec{k}ms, \vec{l}ns} (\vec{q}) a^{\dagger}_{\vec{k} ms} (t) a_{\vec{l} ns}(t)\\
  \nonumber &= \int \mathrm{d}l_x \mathrm{d} l_{z} \int \mathrm{d} y e^{-i q_y y}
                    \delta (l_x - k_x - q_x) \delta (l_z - k_z -  q_z)
                    sv_F e \phi ^{*}_{\vec{k} ms}(y) \sigma ^x \phi _{\vec{l}ns}(y).
\end{align}
The Dirac delta functions appeared from taking the integrals from the matrix element over $x$ and $z$, as the integrand in these variables was only plane waves.
The exact same procedure may be done for the stress-energy tensor in Eq. (\ref{eq:45}).
Eliminating $l$ by doing the integrals yields
\begin{align}
  J^x(t, \vec{q}) &= \sum\limits_{\vec{k}, mn}^{}
                    J^x_{\vec{k}ms, \vec{k}+\qvec{q} ns}(\vec{q}) a^{\dagger}_{\vec{k} ms}(t) a_{\vec{k}+\qvec{q} ns}(t),\\
  T^{0y}(t, -\vec{q}) &= \sum\limits_{\vec{\kappa}, \mu  \nu }^{} T^{0y}_{\vec{\kappa } \mu  s, \vec{\kappa } - \qvec{q}, \nu  s}(\vec{q}) a^{\dagger}_{\vec{\kappa } \mu   s}(t) a_{\vec{\kappa } - \qvec{q} \nu  s}(t),
\end{align}
where ${\qvec{q}} = (q_x, q_z)$.
Keeping in mind that $a_{\lambda }^{\dagger} (t) = e^{i E_{\lambda } t / \hbar }a_{\lambda }^{\dagger}$, and that
\begin{equation}
  \Braket{\left[
a^{\dagger}_{\vec{k}ms} a_{\vec{k}+\qvec{q} ns}, a^{\dagger}_{\vec{\kappa}\mu s} a_{\vec{\kappa}-\qvec{q} \nu  s}
\right]}
=
\delta_{\vec{k}, \vec{\kappa}-\qvec{q}}
\delta _{m, \nu }
\delta _{\vec{k}+\qvec{q}, \vec{\kappa}}
\delta _{n, \mu }
\left[ n_{\vec{k}ms}- n_{\vec{k}+\qvec{q} ns} \right],
\end{equation}
the correlation function is given by
\begin{multline}
  \Braket{\left[ J^x(t, \vec{q}), T^{0y}(t', -\vec{q}) \right]}
  =
  \sum\limits_{\vec{k} mn}^{}
  e^{\frac{i}{\hbar }( E_{k_zms} - E_{k_z+\qvec{q}_z ns} )t}
  e^{\frac{i}{\hbar }( E_{k_z+\qvec{q}_z ns} - E_{k_z ms} ) t'}\\
  \times
  J^x_{\vec{k}ms, \vec{k}+\qvec{q}ns}(\vec{q})
  T^{0y}_{\vec{k}+\qvec{q}ns, \vec{k}ms}(\vec{q})
  \left[ n_{\vec{k}ms}- n_{\vec{k}+\qvec{q} ns} \right].
\end{multline}

We are now ready to find the correlation function $\chi ^{xy}$ given in Eq. (\ref{eq:12})
\begin{equation}
  \label{eq:47}
  \chi ^{xy}(\omega, \vec{q}) =
  \frac{-i v_F}{\mathcal{V} \hbar } 
  \int \mathrm{d}t e^{i \omega t} \int\limits_{-\infty }^0 \mathrm{d}t'
  \Theta (t)
  \Braket{\left[
J^x(t, \vec{q}), T^{0y}(t', -\vec{q})
    \right]}.
\end{equation}
Introduce as usual a decay factor $e^{-\eta (t-t')}$ to ensure convergence in the time integrals, and make a change of variables $t' \to -t'	$.
The integral part of Eq. (\ref{eq:47}), ignoring everything without time dependence for clarity, is then
\begin{multline}
  \lim_{\eta \to 0}
  \int\limits_{0}^{\infty } \mathrm{d}t \mathrm{d}t'
    \exp \left[ \frac{i}{\hbar } \left(
        E_{k_z m s} - E_{k_z+\qvec{q}_z ns} + \omega \hbar  + i \eta \hbar
      \right) t \right]
    \exp \left[ \frac{i}{\hbar } \left(
        E_{k_z m s} - E_{k_z+\qvec{q}_z ns} + i \eta \hbar
      \right) t' \right]\\
  =
  \lim_{\eta \to 0}\frac{\hbar}{i} \left[ E_{k_zms} - E_{k_z+\qvec{q}_z ns} + \omega \hbar  + i\eta \hbar  \right]^{-1}
\frac{\hbar}{i} \left[ E_{k_zms} - E_{k_z+\qvec{q}_z ns} + i \eta \hbar  \right]^{-1}.
\end{multline}
The response function then reads
\begin{multline}
  \chi ^{xy}(\omega , \vec{q}) =
  \frac{i v_F \hbar }{\mathcal{V} }
  \lim_{\eta \to 0}
  \sum\limits_{\vec{k}mn}^{}
  J^x_{\vec{k}ms, \vec{k}+\qvec{q}ns}(\vec{q})
  T^{0y}_{\vec{k}+\qvec{q}ns, \vec{k}ms}(\vec{q})
  \left[ n_{\vec{k}ms}- n_{\vec{k}+\qvec{q} ns} \right] \\
  \left[ E_{k_zms} - E_{k_z+\qvec{q}_z ns} + \omega \hbar  + i\eta \hbar  \right]^{-1}
  \left[ E_{k_zms} - E_{k_z+\qvec{q}_z ns} + i \eta \hbar  \right]^{-1},
\end{multline}
where the matrix elements are
\begin{align}\label{eq:48}
  J^x_{\vec{k}ms, \vec{k}+\qvec{q}ns}(\vec{q}) &= \int \mathrm{d}y
                                                e^{-i q_y y}
                                                s v_F e\phi ^{*}_{\vec{k}m s}(y) \sigma^x
                                                \phi _{\vec{k}+\qvec{q}n s}(y),\\
  T^{0y}_{\vec{k} m s, \vec{k}-\qvec{q} n s}(\vec{q}) &= \frac{1}{4} \label{eq:49}
                                                           \int \mathrm{d}y
                                                           e^{iq_y y}
                                                           \left[
                                                           v_F \phi ^{*}_{\vec{k}m s}(y) p_y
                                                           \phi _{\vec{k}-\qvec{q}n s}(y)
                                                           -
                                                           v_Fp_y \phi ^{*}_{\vec{k}m s}(y)
                                                           \phi _{\vec{k}-\qvec{q}n s}(y)
                                                           \right]\\
  \nonumber &+ \frac{1}{4} 
              \int \mathrm{d}y
              e^{iq_y y}
              \phi ^{*}_{\vec{k}m s}(y)
              s\sigma ^y
              \left(
              E_{k_zm s} + E_{k_z+\qvec{q}_z n s} - 2 \mu
              \right)
              \phi _{\vec{k}-\qvec{q} n s}(y).
\end{align}

We will consider the response function in the static limit \( \lim_{\omega \to 0} \lim_{\vec{q} \to 0} \).
We may use the property of the limit of a product of functions \( \lim A\cdot B = \lim A \cdot \lim B \) to write
\begin{equation}
  \lim_{\omega \to 0} \lim_{\vec{q} \to 0} \chi^{xy}(\omega, \vec{q}) = \frac{i v_F \hbar}{\mathcal{V}} \sum\limits_{\vec{k} m n}^{}
  \frac{
    J^x_{\vec{k} m s, \vec{k} n s} T^{0y}_{\vec{k} n s, \vec{k} m s} [n_{\vec{k} m s} - n_{\vec{k} n s}]
  }{
    (E_{k_z m s} - E_{k_z n s}) (E_{k_z m s}- E_{k_z n s})
  },
\end{equation}
where the current and energy-momentum tensor matrix elements are the expression given in Eqs.~\eqref{eq:48} and \eqref{eq:49} taken in the limit.


\section{Response of an untilted cone}
\subsection{Explicit form of the matrix elements}
Compared to the procedure used by \citeauthor{arjonaFingerprintsConformalAnomaly2019}\cite{arjonaFingerprintsConformalAnomaly2019}, taking the limit of each matrix element by itself greatly simplifies the calculation.

Let
\begin{equation}
  \label{eq:50}
  \phi _{\vec{k}ms}(y)
  = e^{-\frac{(y-k_xl_B^2)^2}{2 l_{B}^2}}
  \begin{pmatrix}
    a_{k_zms} H_{M-1} \left( \frac{y - k_xl_B^2}{l_B} \right)\\
    b_{k_zms} H_M \left( \frac{y - k_xl_B^2}{l_B} \right)
  \end{pmatrix},
\end{equation}
thus implicitly defining the prefactors $a_{k_z ms}, b_{k_z ms}$.


\subsubsection{The current operator}
The matrix element
\begin{align}
  &J_{\vec{k}ms; \vec{k}+\qvec{q} ns}(\vec{q})\\
  \nonumber &=  \int \mathrm{d}y
    e^{-iq_yy} sv_Fe \phi ^{*}_{\vec{k} ms}(y)
    \sigma ^x
    \phi _{\vec{k} +\qvec{q} ns}(y)\\
  &= s v_F e \int \mathrm{d}y
    \exp \left\{
    - i q_yy - \frac{(y-k_xl_B^2)^2 + (y-(k_x + q_x) l_B^2)^2}{2 l_B^2}
    \right\}\\
  \nonumber &\pe \left[
    a_{k_zms}b_{k_z + \qvec{q}_z ns} H_{M-1} \left( \frac{y - k_xl_B^2}{l_B} \right) H_N\left( \frac{y - (k_x + q_x) l_B^2}{l_B} \right)\right.\\
  \nonumber &+
    \left.  b_{k_zms} a_{k_z + \qvec{q}_z ns}
    H_M \left( \frac{y - k_xl_B^2}{l_B} \right)
    H_{N-1} \left( \frac{y - (k_x+ q_x) l_B^2}{l_B} \right)
    \right]\\
    %%% 
  &= s v_F e \int \mathrm{d}y
    \exp \left[-\left\{
    y+ \frac{l_B^2}{2} \left( iq_y - 2k_x - q_x \right)
    \right\}^2 /l_B^2\right]\\
  \nonumber &\pe \exp \left[- \frac{1}{4} l_B^2 \left\{
    \qvec{q}_y^2 + 2 i (2k_x + q_x) q_y 
    \right\}\right]\\
  \nonumber &\pe \left[
    a_{k_zms}b_{k_z + \qvec{q}_z ns} H_{M-1} \left( \frac{y - k_xl_B^2}{l_B} \right) H_N\left( \frac{y - (k_x + q_x) l_B^2}{l_B} \right) \right.\\
  \nonumber &\left. +
    b_{k_zms} a_{k_z + \qvec{q}_z ns}
    H_M \left( \frac{y - k_xl_B^2}{l_B} \right)
    H_{N-1} \left( \frac{y - (k_x + q_x) l_B^2}{l_B} \right)
    \right],
\end{align}
where we completed the square in the exponent, to get the form $e^{-a(y + b)^2}$.
Also, $\qvec{q}_y=(q_x, q_y)$, was introduced, not to be confused with $\qvec{q} = (q_x, q_z)$.
By introducing $\tilde{y} = \frac{y}{l_{B}} + l_B(iq_y - q_x - 2 k_x) / 2$ the matrix element may be rewritten
\begin{equation}
  \label{eq:51}
  \begin{split}
    J_{\vec{k}ms; \vec{k}+\qvec{q} ns}(\vec{q}) &=
    s v_F e \int \mathrm{d}\tilde{y} \: l_B
\exp \left[- \frac{1}{4} l_B^2 \left\{
    \qvec{q}_y^2 + 2 i (2k_x + q_x) q_y 
    \right\}\right]\\
  e^{-\tilde{y}^2}
   &\left[
    a_{k_zms}b_{k_z + \qvec{q}_z ns}
    H_{M-1} \left( \tilde{y} + \frac{l_B}{2}(q_x - iq_y) \right)
    H_N\left( \tilde{y} + \frac{l_B}{2}(-q_x - iq_y) \right) \right.\\
   &\left. +
    b_{k_zms} a_{k_z + \qvec{q}_z ns}
    H_M \left( \tilde{y} + \frac{l_B}{2}(q_x - iq_y) \right)
    H_{N-1} \left( \tilde{y} +  \frac{l_B}{2}(-q_x - iq_y)\right)
    \right].
  \end{split}
\end{equation}
Taking the limit we find the simple form
\begin{equation}
  \label{eq:52}
  J_{\vec{k} m s; \vec{k} n s} = J_{k_z m n s} =
  s v_F e l_B \int \mathrm{d}\tilde{y} e^{-\tilde{y} }
  \left[
    a_{k_zms} b_{k_z n s} H_{M-1}(\tilde{y} )H_N(\tilde{y} )
    + m \leftrightarrow n
  \right],
\end{equation}
where \( m \leftrightarrow n \) are the repetition of the previous term under the interchange of \( m, n \).
We employ now the orthogonality relation of the Hermite polynomials \cite[Table~18.3.1]{NIST:DLMF}
\begin{equation}
  \label{eq:hermite-ortho}
  \int\limits_{-\infty}^{\infty} \mathrm{d}x e^{-x^2} H_n(x)H_m(x) = \sqrt{\pi} 2^{n} n! \delta_{n,m}
\end{equation}
to write
\begin{equation}
  J_{\vec{k} m s, \vec{k} n s} = J_{k_z m n s}
  = s v_F e l_B \sqrt{\pi} (a_{k_z ms} b_{k_z n s} \delta_{M-1, N} 2^N N! + m \leftrightarrow n).
\end{equation}
With
\begin{align}
  a_{\vec{k}ms}b_{\vec{k}ns} &= 
  \frac{\alpha_{k_z ms} }{
    \sqrt{\alpha _{k_z ms}^2 +1}
    \sqrt{\alpha _{k_z ns}^2 + 1}
  }
  \left[ 2^{N+M-1} (M-1)! N! \pi l_B^2 \right]^{-\frac{1}{2}},\\
  b_{\vec{k}ms}a_{\vec{k} ns} &= 
  \frac{\alpha_{k_z ns} }{
    \sqrt{\alpha _{k_z ms}^2 +1}
    \sqrt{\alpha _{k_z ns}^2 + 1}
  }
  \left[ 2^{N+M-1} (N-1)! M! \pi l_B^2 \right]^{-\frac{1}{2}}.
\end{align}
we find explicitly
\begin{equation}
  J_{\vec{k} m s, \vec{k} n s} = J_{k_z m n s} =
  s v_F e
  \frac{\alpha_{k_z m s} \delta_{M-1, N} + \alpha_{k_z n s} \delta_{M, N-1}}{\sqrt{\alpha _{k_z m s}^2 + 1} \sqrt{\alpha _{k_z n s}^2 + 1}}.
\end{equation}


\subsubsection{The stress-energy tensor operator}
Consider the first part of the stress-energy matrix element
\begin{equation}
\label{eq:53}
T^{0y \;(1)}_{\vec{k}+\qvec{q}ns, \vec{k}ms} (\vec{q})
=
\frac{1}{4}
\int \mathrm{d}y e^{iq_y y}
\phi ^* _{\vec{k}+\qvec{q} ns}(y) s \sigma ^y (E_{k_z m s} + E_{k_z+\qvec{q}_z n s}- 2\mu )
\phi _{\vec{k} ms}(y).
\end{equation}
Recall that 
\begin{equation}
  \phi _{\vec{k}ms}(y) =
  e^{- \frac{(y - k_x l_B ^2)^2}{2 l_B^2}}
  \begin{pmatrix}
    a_{k_z ms} H_{M-1} \left( \frac{y-k_x l_B^2}{l_B} \right)\\
    b_{k_z ms} H_M \left( \frac{y - k_x l_B^2}{l_B} \right)
  \end{pmatrix}.
\end{equation}
The form of the integrand is very similar to the current matrix case, with the exchange of the Pauli matrix $\sigma ^x \to \sigma ^y$, thus giving an additional $i$ and a negative sign to the first term.
\begin{align}
&T^{0y \;(1)}_{\vec{k}+\qvec{q}ns, \vec{k}ms} (\vec{q})\\
  \nonumber &= \frac{i s}{4}
    (E_{k_z m s} + E_{k_z+\qvec{q}_z n s}- 2\mu )
    \int \mathrm{d}y e^{iq_y y} e^{- \frac{(y-k_xl_B^2)^2 + (y-(k_x + q_x) l_B^2)^2}{2 l_B^2}}\\
    \nonumber & \phantom{=} \left[
    - a_{k_z+\qvec{q}_z ns} b_{k_z ms} H_{N-1} (\dots ) H_M(\dots )
    + b_{k_z+\qvec{q}_z ns} a_{k_zms} H_N(\dots ) H_{M-1}(\dots )
    \right].
\end{align}
Taking care to note that the factor from the Fourier transform, that was $e^{-iq_y y}$ in the current matrix element is here $e^{+ i q_y y}$, a similar completion of the square is done 
\begin{equation}
  \begin{split}
    &T^{0y \;(1)}_{\vec{k}+\qvec{q}ns, \vec{k}ms} (\vec{q})\\
    &=
    \frac{i s}{4}
    (E_{k_z m s} + E_{k_z+\qvec{q}_z n s}- 2\mu )
    \exp \left[
      -\frac{l_B^2}{4} \left\{ \qvec{q}_y^2 - 2 i q_y (2 k_x + q_x) \right\}
    \right]\\
    &\phantom{=} \int \mathrm{d}y
    \exp \left[
      -\left\{ y + \frac{l_B^2}{2} (-iq_y - 2 k_x - q_x) \right\}^2 / l_B^2
    \right]\\
    &\phantom{=} \left[
      - a_{k_z+\qvec{q}_z ns} b_{k_z ms} H_{N-1} (\dots ) H_M(\dots )
      + b_{k_z+\qvec{q}_z ns} a_{k_zms} H_N(\dots ) H_{M-1}(\dots )
    \right].
  \end{split}
\end{equation}
The arguments of the Hermite polynomials have been dropped for brevity of notation.
As before make a change of variables to get the integral on the form of the shifted orthogonality relation for the Hermite polynomials Eq.~(\ref{eq:hermite-shift-ortho}).
Upon introducing $\tilde{y} = \frac{y}{l_{B}} + l_B( -iq_y - q_x - 2k_x) / 2$ the shifted orthogonality relation is used on the expression
\begin{equation}
  \label{eq:54}
  \begin{split}
    T^{0y \;(1)}_{\vec{k}+\qvec{q}ns, \vec{k}ms} (\vec{q})
    &= \frac{i s}{4}
    (E_{k \mu s} + E_{\lambda \nu s}- 2\mu )
    \exp \left[
      -\frac{l_B^2}{4} \left\{ \qvec{q}_y^2- 2 i q_y (2 k_x + q_x) \right\}
    \right]
    \int \mathrm{d}\tilde{y} \; l_B
    e^{-\tilde{y}^2}\\
    % \exp \left[
    %   -\left\{ y + \frac{l_B^2}{2} (-iq_y - 2 k_x - q_x) \right\}^2 / l_B^2
    % \right]\\
    &\phantom{=} \left[
      - a_{\vec{k}+\qvec{q} ns} b_{\vec{k} ms}
      H_{N-1} \left( \tilde{y} + \frac{l_B}{2} ( iq_y - q_x) \right)
      H_M \left( \tilde{y} + \frac{l_B}{2} (iq_y + q_x) \right)\right.\\
      &\phantom{=\big[} \left.+ b_{\vec{k}+\qvec{q} ns} a_{\vec{k}ms}
      H_N \left( \tilde{y} + \frac{l_B}{2} ( iq_y - q_x) \right)
      H_{M-1} \left( \tilde{y} + \frac{l_B}{2} (iq_y + q_x) \right)
    \right].
  \end{split}
\end{equation}
The terms in the integrand are exactly the same as in the current matrix element case, just in the reverse order and with $q_y \to -q_y$.
\begin{equation}
  \label{eq:55}
  T^{0y \;(1)}_{\vec{k} ns, \vec{k}ms} (\vec{q})
  = \frac{i s}{4}
  \frac{
    (E_{k_z m s} + E_{k_z n s}- 2\mu )
  }{
    \sqrt{\alpha _{k_zms}^2 +1}
    \sqrt{\alpha _{k_zns}^2 + 1}
  }
  \left(
    \alpha _{k_z m s}
    \delta_{M-1, N}
    -
    \alpha _{k_z n s}
    \delta_{M, N-1}
  \right).
\end{equation}

\begin{summary}
  For a untilted case, in the local limit \( q\to 0 \), we have the matrix elements
  \begin{align}
    J_{\vec{k} ms; \vec{k} ns}&=
                                \Gamma_{k_z m n s}
                                sv_F e
                                \left(
                                \alpha_{\vec{k} m s} \delta _{M-1, N}
                                + m\leftrightarrow n
                                \right),\\
                                %%
    T^{0y\; (1)}_{\vec{k} ns, \vec{k}ms} &=
                                           \frac{is \Gamma_{k_z m n s}}{4}
                                           \left(E_{k_z m s} + E_{k_z n s}- 2\mu \right)
                                           \left(
                                           \alpha_{k_z m s} \delta_{M-1, N}
                                           -
                                           m\leftrightarrow n
                                           \right).
  \end{align}
where \( m \leftrightarrow n \) represent the preceding term under the interchange of \( m, n \) and where we have defined
$
\Gamma_{k_z m n s} =
\left[(\alpha _{k_zm s}^2 + 1) (\alpha _{k_z n s}^2 + 1) \right]^{-\frac{1}{2}}
$.
\end{summary}

\subsection{Comment on the energy-momentum tensor}
There is some ambiguity with regards to the definition of th energy-momentum tensor \todo{cite}.
The \emph{canonical} energy-momentum tensor, derived from Lagrangian mechanics, is defined as
\begin{equation}
	T^{\mu \nu } = \frac{\partial \mathcal{L}}{\partial \partial _{\mu } \psi_i } \partial ^{\nu } \psi_i - \eta^{\mu \nu } \mathcal{L}.
\end{equation}
On the other hand, from general relativity, the (\emph{dynamical}?) energy-momentum tensor is defined by the variation of the action with respsect to the metric
\begin{equation}
	T^{\mu \nu} = something something \frac{\delta S}{\delta g_{\mu \nu}}.
\end{equation}
Immediately, we see that the first definition is in general not symmetric, while the latter is, as the metric is always symmetric \footnote{something with torsion never}.
...
Something about the non-defitness of the tensor, we may add some total derivative or something.

We may of course symmetrize the energy-momentum tensor.
Denote by \( T^{\mu \nu } \) the \emph{canonical} tensor, and let the symmetrized tensor
\begin{equation}
  \label{eq:symstress}
  T_S^{\mu \nu} = \frac{T^{\mu \nu} + T^{\nu \mu}}{2}.
\end{equation}

In the case of our untilted system, the Weyl Lagrangian, the components of interest are
\begin{align}
  \label{eq:56}
  T^{0 y} &= \frac{v_F}{4}
  \left[
  \phi^{\dagger} p_y \phi - p_y \phi^{\dagger} \phi
  \right],\\
  T^{y 0} &= \frac{s i}{4}
  \left[
  \phi^{\dagger} \sigma_y \partial_{0} \phi - \partial_0 \phi^{\dagger} \sigma_y \phi
  \right].\\
\end{align}


The symmetric form of the energy-momentum tensor, used by \citeauthor{arjonaFingerprintsConformalAnomaly2019}, gives additional contributions to the energy-momentum matrix element.
We will here show that in the case of no tilt, these contributions are identical to those of the non-symmetric tensor.
In the tilted case, however, the contributions differ.

The first other contribution is
\todo{take care of prefactors}
\begin{equation}
  \label{eq:57}
  \left(\frac{\sqrt{M}}{\alpha_{\vec{k} m s}} + \sqrt{(M-1)} \alpha_{\vec{k} n s}\right) \alpha_{\vec{k} m s} \delta_{M-1, N}.
\end{equation}
The normalization factor, given in dimensionless quantities is,
\[
\alpha_{k_z m s} = - \frac{s \sqrt{M}}{\epsilon_{m} - s \kappa}.
\]
Inserting this, and using the explicit form of the energy for \( m \neq 0 \)
\[
\epsilon_n = \sign(m) \sqrt{M + \kappa^2},
\]
for the case \( N > 0 \) the contribution can be shown to be
\begin{equation}
  \label{eq:58}
  -s (\epsilon_m + \epsilon_n)\alpha_{\vec{k} m s} \delta_{M-1, N}.
\end{equation}
For \( n = 0 \), the second term of Eq. \eqref{eq:57} is zero, and we have
\todo{mising s?}
\begin{equation}
  \label{eq:59}
  -(\epsilon_m - s \kappa) \alpha_{k m s} \delta_{M-1, N},
\end{equation}
and by identifying \( \epsilon_0 = - s \kappa \) this has the same form as Eq. \eqref{eq:58}.

In the case of tilt, however, the contribution can be shown to be
\begin{equation}
  \label{eq:60}
  -\frac{s}{\sqrt{\alpha}} (\epsilon_{m, \alpha B}^0 + \epsilon_{n, \alpha B}^0),
\end{equation}
where \( \epsilon_{m, \alpha B}^0 = \sign(m) \sqrt{\alpha M + \kappa^2}  \) and we used
\[
\alpha_{\vec{k} m s} = - \frac{\sqrt{\alpha M}}{s \epsilon^{0}_{m, \alpha B} - \kappa}
\]
in the tilted case.
Thus, we see that in the case of tilt perpendicular to the \( B \)-field, the contribution is scaled compared to the non-symmetric term.
In the case of tilt parallel to the \( B \)-field, one gets an additional term proportional to \( t_{\parallel} \kappa \).

\subsection{Computing the reponse function}
It is now finally possible to write out the entire response function.
The response function will be split into three parts, $\chi ^{xy\; (i)},\: i = 1,2,3$ corresponding to the three parts of the stress-energy tensor.
Also, the sum over the $\vec{k}$ values will be replaced by an integral.
Firstly, we will show that the sum over $k_x$ is restricted;
recall that the eigenfunctions are exponentially centered around $y_0 = k_x l_B^2$, which for a finite sample we expect to be restricted to $0 \leq y_0 \leq L_y$.
This restricts the $k_x$ sum to $0 \leq k_x \leq L_y / l_B^2 = L_ye B /\hbar $, resulting in the $k_x$ summation giving a finite degeneracy contribution~\cites[Ch.~1.4.1]{tongGaugeTheoryLecture}{linderIntermediateQuantumMechanics2017}.
\begin{align}
  \sum\limits_{\vec{k}}^{} = \sum\limits_{k_x = 0}^{L_y eB / \hbar } \sum\limits_{k_z}^{} &\to 
                                                                                            \frac{L_xL_z}{(2\pi )^2} \int\limits_0^{L_y e B /\hbar } \mathrm{d}k_x \int\mathrm{d}k_z \\
  &= \frac{\mathcal{V} e B}{(2 \pi)^2 \hbar } \int \mathrm{d}k_{z}.
\end{align}

Recall the response function
\begin{equation}
  \chi ^{xy} (\omega , \vec{q}) =
  \lim_{\eta \to 0}
  \sum\limits_{\vec{k}, mn}^{}
  \frac{1}{\mathcal{V}}
  \frac{
    i v_F \hbar J^x_{\vec{k} m s, \vec{k}+\qvec{q} n s}(\vec{q})
    T^{0y}_{\vec{k}+\qvec{q} ns, \vec{k}ms}(\vec{q})
    \;
    [n_{\vec{k} m s}- n_{\vec{k}+\qvec{q} ns }]
  }{
    (E_{k_z m s} - E_{k_z + \qvec{q}_z ns} + i \hbar  \eta )
    (E_{k_z m s} - E_{k_z + \qvec{q}_z ns} + \hbar \omega + i \hbar  \eta )
  }
  .
\end{equation}
Firstly, introduce the dimensionless quantities \( \kappa_z \sqrt{2 eB} = k_z, \epsilon_{k_z m s} v_F \sqrt{2 e B} = E_{k_z m s}  \), in order to facilitate solving the integral over \( k_z \).
Collecting dimensionfull quantites, the response function reads
\begin{multline}
  \label{eq:61}
\lim_{\omega \to 0} \lim_{\vec{q} \to 0} \chi^{xy} =
  -\frac{e^2 v_F B }{4 (2 \pi)^2}
  \sum\limits_{m n}^{}
  \int \mathrm{d} \kappa_z
  [n_{\kappa_z m s} - n_{\kappa_z n s}]
  [(\alpha_{\kappa_z m s}^2 + 1) (\alpha_{k_z n s}^2 + 1)]^{-1}\\
  \times
  \frac{
    (\epsilon_{\kappa_z m s} + \epsilon_{\kappa_z n s})
    (\alpha_{\kappa_z m s}^2 \delta_{M-1, N} - \alpha_{\kappa_z n s}^2 \delta_{N-1, M})
  }{
    (\epsilon_{\kappa_z m s} - \epsilon_{\kappa_z n s} + i \eta)^2
  }.
\end{multline}

Let us now define
\begin{equation}
  \xi(\kappa_z) = \frac{[n_{\kappa ms} - n_{\kappa + \qvec{q} ns}]
  \left[ (\alpha _{\kappa ms}^2 + 1) (\alpha _{\kappa +\qvec{q} ns}^2 + 1) \right]^{-1}
  }{
    (\epsilon _{\kappa m s} - \epsilon _{\kappa +\qvec{q}ns} + i \frac{\hbar  \eta }{v_F\sqrt{2 e B \hbar }})
    (\epsilon _{\kappa m s} - \epsilon _{\kappa +\qvec{q}ns} + \frac{\hbar \omega }{v_F \sqrt{2 e B \hbar }} + i \frac{\hbar  \eta }{v_F\sqrt{2 e B \hbar }})
  },
\end{equation}
which is odd under interchange of \( m,n \), i.e. \( \xi_{m,n} = -\xi _{n,m} \). \footnote{The impurity factor \( \eta \) also acquires a sign, but this is not an issue.}
Using this, we may simplify our expressions some.
In the last term of Eq. \eqref{eq:61}, relabel the summation indices \( m \leftrightarrow n \), and then use that \( \xi \) is odd under interchange of \( m,n \).
This renders the two terms equal, and we may consider
\[
\alpha_{\kappa_z m s}^2 \delta_{M-1,N} - \alpha_{\kappa_z n s}^2 \delta_{N-1, M} \to 2 \alpha_{\kappa_z m s}^2 \delta_{M-1, N}.
\]
The final expression is then
\begin{equation}
  \label{eq:62}
  \lim_{\omega \to 0} \lim_{\vec{q} \to 0} \chi^{xy} =
  -\frac{e^2 v_F B}{2 (2 \pi)^2} \sum\limits_{\underset{N=M-1}{mn}}
  \int \mathrm{d}\kappa_z \xi(\kappa_z)
  (\epsilon_{\kappa_z m s} + \epsilon_{\kappa_z n s} - 2 \mu) \alpha_{\kappa_z m s}^2.
\end{equation}

Before solving the integral, we note that in addition to the \todo{say the word diatomic?} \( N=M-1 \) selection rule of the sum, the factor with the distributions \( n_{\kappa_z m s} - n_{\kappa_z n s} \) impose further restrictions on which transitions are energetically allowed.
We consider the limit \( T \to 0 \) \todo{something about the Luttinger in this limit? I.e. the fact we get finite result in T-> 0 is the interesting thing about this result}, where the distributions take the form of step functions, \( n_{\kappa_z m s} \to \theta(-\epsilon_{\kappa_z m s}) \).
As the sign of energy level \( m \), for \( m \neq 0 \), is given by the sign of \( m \) itself, this gives a rather simple restriction on the sum.
For the zeroth energy level, the sign of the energy is given by \( \sign(-s\kappa_z) \).
The distribution factor is
\begin{equation}
  n_{\vec{k} m s} - n_{\vec{k} n s} =
  \begin{cases}
    \phantom{-} 0 & m n > 0 \text{ or  } m,n = 0,\\
    - \sign(m) & m,n \neq 0,\\
    -\sign(m) \theta\left[\sign(m) s \kappa_z\right] & n = 0.
  \end{cases}
\end{equation}
Combining this with the selection rule \( N=M-1 \), we see that the only allowed transitions are
\[ M \to -N = -(M-1), -M \to N = (M-1). \]

The last simplification we will make, is to note that the step function is odd under \( (m, n, \kappa_z) \to (-m, -n, -\kappa_z) \), and likewise with \( \epsilon_{\kappa_z m s} - \epsilon_{\kappa_z n s} \).
In the case of zero chemical potential, the expression may be simplified further, by considering only \( -N \to M = N + 1\) transitions, adding a factor 2.

For zero chemical potential, the response function is
\begin{equation}
  \label{eq:63}
  \lim_{\omega \to 0} \lim_{\vec{q} \to 0} \chi^{xy} =
  -\frac{e^2 v_F B}{(2 \pi)^2}
  \sum\limits_{N=0} \int \mathrm{d}\kappa_z
  \xi(\kappa_z) (\epsilon_{\kappa_z m s} + \epsilon_{\kappa_z n s})
  \alpha_{\kappa_z m s}^2
  \big|_{\underset{n=-N}{m=N+1}},
\end{equation}
where the integration limits are \( (-\infty, \infty) \) for \( N \neq 0 \), \( (-\infty, 0) \) for \( N = 0, s=-1 \), and \( (0, \infty) \) for \( N=0, s=1 \).

Including only the first term of the sum, we find
\begin{equation}
  \label{eq:64}
  \lim_{\omega \to 0} \lim_{\vec{q} \to 0} \chi^{xy} = \frac{e^2 v_F B}{(2 \pi)^2}.
\end{equation}
Including contributions from the \( N \) lowest landau levels, one acquire additional numerical prefactors,
\begin{equation}
  \label{eq:65}
  \lim_{\omega \to 0} \lim_{\vec{q} \to 0} \chi^{xy} = \gamma_N \frac{e^2 v_F B}{(2 \pi)^2},
\end{equation}
where the factor by analytical integration was found to be \( \gamma_0 = 1, \gamma_{20} \approx 2 \).
Furthermore, \( \gamma_N \) goes like \( \log N \).
The first 300 contributions are shown in Figure \ref{fig:contributions}.

Solving the integral analytically, we obtained the contribution from each term
\[
  \gamma_N - \gamma_{N-1} = \frac{1}{4} \left[
  1 + 2 N \left\{1 - (1+N) \log(1 + \frac{1}{N}) \right\} 
\right], \quad N > 0.
\]
The sum can be shown to equal the rather nasty expression
\begin{equation}
  \begin{split}
  \gamma_N = \gamma_0 + \frac{1}{12} &\Big(
    6 \zeta ^{(1,0)}(-2,N+1)
    -6 \zeta ^{(1,0)}(-2,N+2)
    +6 \zeta^{(1,0)}(-1,N+1)\\
    &\hphantom{\Big(} \mathllap{+} 6 \zeta^{(1,0)}(-1,N+2)
    +12 \log (\xi)
    +3 N^2
    +6 N
    -1
  \Big),
  \end{split}
\end{equation}
where \( \xi \approx  1.28243 \) is Glaisher's constant.
This expression goes like \( \log N \).

\begin{figure}[ht]
  \centering
  \newcommand\datafiles{data/notilt_contrib2.csv, data/notilt_contrib.csv}
  \newcommand\legendentries{\( \vec{t} = 0 \), \( t_x = 0.1 \)}
  \newcommand\plottitle{Yupp}  % Leave empty to get default
  % Assumes \datafiles is set
\begin{tikzpicture}
  \begin{axis}[
    title=\ifthenelse{\equal{\plottitle}{}}{Convergence Plot}{\plottitle},
    xlabel={Number of terms $N$},
    ylabel={$\gamma_N$},
    grid=major,
    width=\textwidth,
    height=.62\textwidth,
    % width=14cm,
    % height=8cm,
    % ymode=log,
    legend pos=south east,
    ]
    \ifthenelse{\isundefined\prefix}{\newcommand\prefix{t_x=}}{}
    \foreach \datafile/\entry in \datafiles {
      \addplot[mark=o] table[col sep=comma] {data/\datafile};
      \edef\tmp{\noexpand\addlegendentry{$ \prefix \entry $}}
      \tmp
    }
  \end{axis}
\end{tikzpicture}

  \caption{Prefactor \( \gamma_N \) as a function of the number of inlcuded terms \( N \).}
\end{figure}


% \todo{Make sure have notes that consider zero T for the distribution}
% In the static limit $\omega \to 0$, with potential $\mu =0$ \todo{explain}, the integral was solved in a CAS \todo{how to formulate this?}, and found to be $\frac{1}{2}$.
% Thus, considering only the lowest Landau level contributions, in which case the sum goes over $n=\pm 1, m=0$ only,
% \begin{equation}
%   \lim_{\omega \to  0} \lim_{\vec{q}\to 0}
%   \chi ^{xy \; (1)}
%   = \frac{e^2B v_F}{4 (2\pi )^2 \hbar }.
% \end{equation}
% Once again, in  the limit the integral is evaluated and found to be $\mp \frac{1}{2}$ for $s=\pm 1$, or rather $- s \frac{1}{2}$.
% Thus, summing over only the two main contributions
% \begin{equation}
%   \lim_{\omega \to  0} \lim_{\vec{q}\to 0}
%   \chi ^{xy \; (3)}
%   = \frac{e^2B s^2 v_F}{4 (2\pi )^2 \hbar }.
% \end{equation}
% The total transverse response function is therefore
% \begin{equation}
%   \lim_{\omega \to  0} \lim_{\vec{q}\to 0}
%   \chi ^{xy}
%   = \frac{e^2B v_F}{2 (2\pi )^2 \hbar }.
% \end{equation}

% \begin{figure}[h]
%   \centering
%   \input{figures/convergence_chi}
%   \caption{Prefactor $\gamma_N$ of $\chi$ vs. number of terms $N$ included in sum.}
%   \label{fig:convergence_chi}
% \end{figure}


