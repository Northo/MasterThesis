\usepackage{mathtools}
\usepackage[final]{pdfpages}
\usepackage{amsfonts}
\usepackage{biblatex}
\usepackage{xcolor}
\usepackage{graphicx}
\usepackage{bm}
\usepackage{braket}
\usepackage{amsthm}
\usepackage{microtype}
\usepackage{thmtools}
\usepackage{siunitx}
\usepackage{subcaption}
\usepackage{tikz}
\usepackage{pgfplots}
\usepackage{pgfplotstable}
\usepackage{hyperref}
\usepackage{float}
\usepackage{etoolbox}
\usepackage{booktabs}
\usepackage{multirow}
\usepackage[final]{listings}
\usepackage{cleveref}
\usepackage{comment}
\usepackage{placeins}
\usepackage{slashed}  % Gives the /slashed{A} macro. Consider in the future to switch with manually greated macro /slashed made from /not operator, as read in forum (supposedly looks better)
\usepackage[acronym]{glossaries}
\usepackage[compat=1.1.0]{tikz-feynman}
\usepackage{hyperref}
\hypersetup{
colorlinks=true,
allcolors=blue!80!red!60!black,
final=true,
}
\usepgfplotslibrary{colormaps}
\pgfplotsset{compat=1.18}
\usepgfplotslibrary{groupplots}
\usepgfplotslibrary{external}
\tikzexternalize
\tikzsetexternalprefix{figures/external/}
% \pgfkeys{/pgf/images/include external/.code={\href{file:#1}{\pgfimage{#1}}}} %% for draft mode. See pgf manual
\pgfkeys{/pgf/images/include external/.code={\href{file:#1}{\includegraphics[draft=false]{#1}}}} %% for draft mode. See pgf manual

\setacronymstyle{short-long}

\addbibresource{Master.bib}
\addbibresource{manual.bib}
\renewcommand\multicitedelim{\addsemicolon\space}


% \usepackage{fancyhdr}

%%% Header %%%%
% Layout 0
% \pagestyle{fancy}
% \addtolength{\headwidth}{\marginparsep}
% \addtolength{\headwidth}{\marginparwidth}
% \renewcommand{\chaptermark}[1]{\markboth{#1}{}}
% \renewcommand{\sectionmark}[1]{\markright{\thesection\ #1}}
% \fancyhf{}
% \fancyhead[LE,RO]{\textbf{\thepage}}
% \fancyhead[LO]{\textbf{\rightmark}}
% \fancyhead[RE]{\textbf{\leftmark}}
% \fancypagestyle{plain}{%
% \fancyhead{} % get rid of headers
% \renewcommand{\headrulewidth}{0pt} % and the line
% }
% Layout 1
% \pagestyle{fancy}
% % \addtolength{\headwidth}{\marginparsep}
% \addtolength{\headwidth}{\marginparwidth}
% \renewcommand{\chaptermark}[1]{\markboth{\MakeUppercase{#1}}{}}
% \renewcommand{\sectionmark}[1]{\markright{\thesection.\ #1}}
% % \fancyhead[LE,RO]{\thepage}
% \fancyhead{}
% \fancyhead[RO]{\rightmark}
% \fancyhead[LE]{\leftmark}
% \fancyfoot{}
% \fancyfoot[LE,RO]{\thepage}

% % Layout 2 %%
% \pagestyle{fancy}
% \renewcommand{\chaptermark}[1]%
% {\markboth{\MakeUppercase{\thechapter.\ #1}}{}}
% \renewcommand{\sectionmark}[1]%
% {\markright{\MakeUppercase{\thesection.\ #1}}}
% \renewcommand{\headrulewidth}{0.5pt}
% \renewcommand{\footrulewidth}{0pt}
% \newcommand{\helv}{%
% \fontfamily{phv}\fontseries{b}\fontsize{9}{11}\selectfont}
% \fancyhf{}
% \fancyhead[LE,RO]{\helv \thepage}
% \fancyhead[LO]{\helv \rightmark}
% \fancyhead[RE]{\helv \leftmark}
