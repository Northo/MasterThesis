\newcommand{\cbox}[2][yellow]{%
  \colorbox{#1}{\parbox{\dimexpr\linewidth-2\fboxsep}{\strut #2\strut}}%
}
\newcommand{\todo}[2][orange]{\hfill\break{\bfseries\cbox[#1]{#2}}\break}

\newcommand{\pe}{\phantom{=}}
% \newcommand{\sign}{\operatorname{sign}}
\DeclareMathOperator\sign{sign}
\DeclareMathOperator\arctanh{arctanh}
\renewcommand{\vec}{\bm}
\renewcommand{\Im}{Im}
\DeclareMathOperator{\Tr}{Tr}
\newcommand{\qvec}[1]{\mathfrak{#1}} % Used for two-dim vec

\declaretheorem[thmbox=S]{Proposition}
\declaretheorem[thmbox=M, style=remark]{summary}
\declaretheoremstyle[
    spaceabove=6pt,
    spacebelow=6pt,
    headfont=\normalfont\itshape,
    bodyfont = \normalfont,
    postheadspace=1em,
    qed=\( \square \),
    headpunct={:}]{myproofstyle}
\declaretheorem[style=myproofstyle, unnumbered]{Proof}
\declaretheorem[name=Theorem]{theorem}

% \newcommand{\gls}[1]{#1}  % Hack until glossaries work
\newacronym{qed}{QED}{quantum electrodynamics}
\newacronym{qft}{QFT}{quantum field theory}
\newacronym{trim}{TRIM}{time reversal independent momenta}
\makeglossaries
