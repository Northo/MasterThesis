\addchap{Conventions and Symbols}
\begin{itemize}
\item $e$ is the fundamental charge, i.e. $e = |e|$.
\item $l_B$ is the characteristic length of a $B$-field, given as $l_B= \sqrt{\hbar /eB}$, with $e$ the fundamental charge defined above.
\item The signature of the Minkowski metric is taken to be $-2$, i.e. the metric tensor $\eta _{\mu \nu } = \operatorname{diag} (+1, -1,-1,-1)$.
\item The Fourier transform will be taken to be on the asymmetric form, with opposite sign of the temporal and spatial part, thus treating it as a four vector.
  For a $3+1$ dimensional case ($\vec{q}$ three dimensional), the Fourier transform is defined as
  \begin{equation}
    \label{eq:define-fourier}
    A(\vec{q}, \omega )\! =\!\!
    \iint \mathrm{d}t \mathrm{d} \vec{r}
    e^{i(\omega  t - \vec{q} \vec{r} )}
    A(\vec{r}, t),
    \;
    A(\vec{r}, t) =\!\!
    \iint 
    \frac{\mathrm{d}\omega  \mathrm{d} \vec{q}}{(2\pi )^4}
    e^{-i(\omega  t - \vec{q} \vec{r} )}
    A(\vec{q}, \omega).
  \end{equation}
  For other dimensionalities, the exponent of the $2\pi $ factor must be chosen accordingly.

\item Vectors will be written in bold font, $\vec{v} = (v_1, v_2, v_3)$, and with Roman indices, $i, j, k$, for two and three-dimensional vectors.
  Four dimensional vectors will be typed in normal weight, $v$, with Greek indices, $\mu ,  \nu , \lambda $, and upper and lower indices indicating contravariant and covariant quantities
\item Natural units \( \hbar = c = 1 \) will be used in parts of the thesis, for more clear notation and in order to make it easier for the reader to recognize the similarities with high energy physics literature.
\item For spin degrees of freedom, the Pauli matrices
  % \begin{equation}
  %   \sigma _1 =
  %   \begin{pmatrix}
  %     0 & 1\\ 1 & 0
  %   \end{pmatrix},
  %   \quad
  %   %
  %   \sigma _2 =
  %   \begin{pmatrix}
  %     0 & -i\\ i & 0
  %   \end{pmatrix},
  %   \quad
  %   %
  %   \sigma _3 =
  %   \begin{pmatrix}
  %     1 & 0\\ 0 & -1
  %   \end{pmatrix},
  % \end{equation}
  will be denoted by $\sigma $ for real spin and $\tau $ for pseudo-spin.
  
\item Operators will in general be typed with as normal quantities: $O$ for scalar operators and $\vec{O}$ or $O$ for vector operators, depending on their dimensions.
  The hat symbol, $\hat{O}$, will not be used unless not including a hat will be confusing.

\item In Chapter \ref{ch:charge-current}, we will use capital letters \( M, N \) to indicate the absolute value of the corresponding quantity, \( M = |m| \).
\end{itemize}
